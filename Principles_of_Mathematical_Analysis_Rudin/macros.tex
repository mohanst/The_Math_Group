% --- macros.tex ---
% Generated by Gemini (Google AI) on 2025-04-04.
% Contains package inclusions and theorem/environment definitions
% for the Rudin Chapter 1 extracted items document.

% --- Packages ---
\usepackage[utf8]{inputenc}
\usepackage[T1]{fontenc}
\usepackage{amsmath}        % AMS math enhancements
\usepackage{amssymb}        % AMS symbols (for R, Q etc.)
\usepackage{amsthm}         % Theorem environments
\usepackage{hyperref}       % Create hyperlinks (optional)

% --- Theorem Styles & Definitions ---
% Using standard amsthm numbering linked to the chapter counter (requires book class)
\theoremstyle{definition}
\newtheorem{definition}{Definition}[chapter] % Numbered within chapter (e.g., 1.1)
\newtheorem{example}[definition]{Example} % Shares counter with definition
\newtheorem{remark}[definition]{Remark}   % Shares counter with definition

\theoremstyle{plain}
\newtheorem{theorem}[definition]{Theorem}         % Shares counter
\newtheorem{proposition}[definition]{Proposition} % Shares counter
\newtheorem{corollary}[definition]{Corollary}     % Shares counter
\newtheorem{lemma}[definition]{Lemma}             % Shares counter

% --- Hyperref Customization (Optional but improves autoref) ---
\hypersetup{
    colorlinks=true,
    linkcolor=blue,
    citecolor=green,
    urlcolor=magenta
}
% Aliases for \autoref compatibility with amsthm environments
\newcommand{\definitionautorefname}{Definition}
\newcommand{\exampleautorefname}{Example}
\newcommand{\remarkautorefname}{Remark}
% \newcommand{\theoremautorefname}{Theorem}
\newcommand{\propositionautorefname}{Proposition}
\newcommand{\corollaryautorefname}{Corollary}
\newcommand{\lemmaautorefname}{Lemma}

% --- End of macros.tex ---

% Add other packages and custom commands you need below...

% --- Custom Macros Suggestions (Add to macros.tex) ---

% Standard Sets (already using amssymb, but explicit macros are common)
\newcommand{\R}{\mathbb{R}} % Real numbers
\newcommand{\Q}{\mathbb{Q}} % Rational numbers
\newcommand{\C}{\mathbb{C}} % Complex numbers (though not used heavily in Ch1)
\newcommand{\Z}{\mathbb{Z}} % Integers (useful generally)
\newcommand{\N}{\mathbb{N}} % Natural numbers (useful generally, often {1, 2,...})

% Vectors (using bold as in Rudin)
\newcommand{\vect}[1]{\mathbf{#1}} % Usage: \vect{x}, \vect{y}

% Absolute Value / Norm (using lvert/rvert,lVert/rVert for better spacing)
% Rudin uses |.| for both, so you might prefer just |...| in math mode.
\newcommand{\abs}[1]{\left\lvert #1 \right\rvert} % Usage: \abs{z}
\newcommand{\norm}[1]{\left\lVert #1 \right\rVert} % Usage: \norm{\vect{x}}

% Real and Imaginary Parts (requires amsmath)
\DeclareMathOperator{\re}{Re} % Usage: \re(z)
\DeclareMathOperator{\im}{Im} % Usage: \im(z)

% Set Builder Notation
\newcommand{\set}[2]{\left\{ #1 \;\middle|\; #2 \right\}} % Usage: \set{x \in S}{P(x)}

% Inner Product (optional, \cdot is standard)
\newcommand{\innerprod}[2]{\left\langle #1, #2 \right\rangle} % Alternative notation <x,y>
%\newcommand{\innerprod}[2]{#1 \cdot #2} % If you prefer the dot

% --- End of Custom Macros Suggestions ---

% --- Suggested Additions/Modifications for macros.tex ---

% Function mapping notation
\newcommand{\map}[3]{#1: #2 \to #3} % Usage: \map{f}{A}{B}

% Big Union / Intersection with index specification
\newcommand{\bigunion}[1]{\bigcup_{#1}}     % Usage: \bigunion{\alpha \in A} E_\alpha
\newcommand{\bigintersect}[1]{\bigcap_{#1}} % Usage: \bigintersect{n=1}^\infty E_n

% Script font for collections/families of sets (amssymb needed, already included)
\newcommand{\collection}[1]{\mathcal{#1}} % Usage: \collection{F}

% --- End of suggested additions ---

% Add to macros.tex
% \DeclareMathOperator*{\limsup}{lim\,sup} % Creates \limsup command
% \DeclareMathOperator*{\liminf}{lim\,inf} % Creates \liminf command

% Add to macros.tex
\newcommand{\sequence}[1]{\left\{ #1_n \right\}_{n=1}^\infty} % For sequence {x_n} from n=1 to infinity
\newcommand{\seqterm}[1]{#1_n} % For a generic term like s_n

% Add to macros.tex (Optional)
\newcommand{\nhood}[2]{N_{#1}(#2)} % Usage: \nhood{r}{p}

\newcommand{\Usum}[3]{U(#1, #2, #3)} % Upper Sum
\newcommand{\Lsum}[3]{L(#1, #2, #3)} % Lower Sum

\newcommand{\RSintegrable}[1]{\mathcal{R}(#1)} % Set of RS-integrable functions wrt alpha

\newcommand{\arclength}[1]{\Lambda(#1)} % Arc length/Rectifiable curve length

% Add to macros.tex
\newcommand{\contfunc}[1]{\mathcal{C}(#1)} % Space of continuous (bounded) functions

% Add to macros.tex
\newcommand{\avgint}[1]{\frac{1}{2\pi} \int_{-\pi}^{\pi} #1 \, dx} % Average integral over [-pi, pi]
% Add to macros.tex (Optional)
\newcommand{\Lsqinner}[2]{\avgint{#1 \overline{#2}}} % L^2 inner product on [-pi, pi]
% Or using angle brackets if preferred:
% \newcommand{\Lsqinner}[2]{\left\langle #1, #2 \right\rangle_{L^2}}

\DeclareMathOperator{\rank}{rank} % Usage: \rank A

\newcommand{\id}{\mathbf{I}} % Or just \newcommand{\id}{I}

\DeclareMathOperator{\supp}{supp} % Usage: \supp f

\newcommand{\dV}{dV} % For volume element dx_1 \wedge ... \wedge dx_k or dxdydz
\newcommand{\dA}{dA} % For area element
\newcommand{\ds}{ds} % For arc length element

\newcommand{\Measurable}[1]{\mathfrak{M}(#1)} % Usage: \Measurable{\mu}
    \newcommand{\FMeasurable}[1]{\mathfrak{M}_F(#1)} % Usage: \FMeasurable{\mu}
    \newcommand{\mustar}{\mu^*} % Usage: \mustar(E)
    \newcommand{\indicator}[1]{K_{#1}} % Usage: \indicator{E}(x)
    \newcommand{\Lpspace}[1]{\mathcal{L}^{#1}(\mu)} % Usage \Lpspace{2}

% --- End of macros.tex ---