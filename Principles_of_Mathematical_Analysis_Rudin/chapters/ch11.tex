% --- chapters/chapter11.tex ---
% Generated by Gemini (Google AI) on 2025-04-07.
% Contains extracted items from W. Rudin, PMA, Chapter 11.
% Assumes \Measurable, \FMeasurable, \mustar, \indicator, \Lpspace
% macros are defined.

\chapter{The Lebesgue Theory}
\label{chap:rudin11}

% Intro text condensed
This chapter presents the fundamental concepts of Lebesgue measure
and integration theory. Proofs may be sketched or omitted in some cases.

\section{Set Functions}

% Notation: A-B = {x | x in A, x not in B}. 0 = empty set.
% Disjoint sets: A cap B = 0.

\begin{definition}[Ring, $\sigma$-ring]
  \label{def:chap11:ring_sigma_ring}
  A family $\mathcal{R}$ of sets is called a \textbf{ring} if $A \in
  \mathcal{R}$ and $B \in \mathcal{R}$ implies $A \cup B \in
  \mathcal{R}$ and $A - B \in \mathcal{R}$. (This implies $A \cap B
  \in \mathcal{R}$).
  A ring $\mathcal{R}$ is called a \textbf{$\sigma$-ring} if
  $\bigcup_{n=1}^\infty A_n \in \mathcal{R}$ whenever $A_n \in
  \mathcal{R}$ for $n=1, 2, 3, \dots$. (This implies
  $\bigcap_{n=1}^\infty A_n \in \mathcal{R}$).
\end{definition}

\begin{definition}[Set Function, Additivity]
  \label{def:chap11:set_function_additive}
  A \textbf{set function} $\phi$ defined on a ring $\mathcal{R}$
  assigns a number $\phi(A)$ (from the extended real number system)
  to every $A \in \mathcal{R}$.
  \begin{itemize}
    \item $\phi$ is \textbf{additive} if $A \cap B = \emptyset$
      implies $\phi(A \cup B) = \phi(A) + \phi(B)$.
    \item $\phi$ is \textbf{countably additive} if $A_i \cap A_j =
      \emptyset$ for $i \ne j$ implies $\phi(\bigcup_{n=1}^\infty
      A_n) = \sum_{n=1}^\infty \phi(A_n)$.
  \end{itemize}
  (We assume the range of $\phi$ does not contain both $+\infty$ and
  $-\infty$, and $\phi$ is not identically $+\infty$ or $-\infty$.)
\end{definition}

\begin{remark}[Properties of Additive Set Functions]
  \label{rem:chap11:additive_properties}
  If $\phi$ is additive on a ring $\mathcal{R}$:
  \begin{itemize}
    \item $\phi(\emptyset) = 0$.
    \item If $A_i \cap A_j = \emptyset$ for $i \ne j$, $\phi(A_1 \cup
      \dots \cup A_n) = \phi(A_1) + \dots + \phi(A_n)$.
    \item $\phi(A_1 \cup A_2) + \phi(A_1 \cap A_2) = \phi(A_1) + \phi(A_2)$.
    \item If $\phi(A) \ge 0$ for all A ( $\phi$ is non-negative),
      then $A_1 \subset A_2 \implies \phi(A_1) \le \phi(A_2)$ ($\phi$
      is monotonic).
    \item If $B \subset A$ and $|\phi(B)| < +\infty$, then $\phi(A-B)
      = \phi(A) - \phi(B)$.
  \end{itemize}
\end{remark}

\begin{theorem}[Continuity of Measure for Increasing Sets]
  \label{thm:chap11:measure_continuity_increasing}
  Suppose $\phi$ is countably additive on a ring $\mathcal{R}$.
  Suppose $A_n \in \mathcal{R}$ ($n=1, 2, 3, \dots$), $A_1 \subset
  A_2 \subset A_3 \subset \dots$, $A = \bigcup_{n=1}^\infty A_n$, and
  $A \in \mathcal{R}$. Then $\phi(A_n) \to \phi(A)$ as $n \to \infty$.
\end{theorem}
% Proof omitted. (Proof uses $B_1=A_1, B_n = A_n - A_{n-1}$ disjoint sets).

% --- End of transcription chunk ---

% --- Previous content (up to Thm 11.3) from chapters/chapter11.tex above ---

\section{Construction of the Lebesgue Measure}

\begin{definition}[Intervals, Elementary Sets, Pre-measure]
  \label{def:chap11:elementary_sets_premeasure}
  An \textbf{interval} in $\R^p$ is a set of points $\vect{x}=(x_1,
  \dots, x_p)$ defined by $a_i \le x_i \le b_i$ ($i=1, \dots, p$), or
  with any $\le$ replaced by $<$. (Empty set is included).
  If I is such an interval, its \textbf{measure} is $m(I) =
  \prod_{i=1}^p (b_i - a_i)$.

  A set $A \subset \R^p$ is an \textbf{elementary set} if it is the
  union of a finite number of intervals. Let $\mathcal{E}$ denote the
  family of all elementary subsets of $\R^p$.
  If $A = I_1 \cup \dots \cup I_n$ where $I_j$ are disjoint
  intervals, we define $m(A) = \sum_{j=1}^n m(I_j)$.

  Properties:
  \begin{itemize}
    \item $\mathcal{E}$ is a ring (but not a $\sigma$-ring).
    \item Every $A \in \mathcal{E}$ is a finite disjoint union of intervals.
    \item $m(A)$ is well-defined (independent of the disjoint decomposition).
    \item The set function m is additive on $\mathcal{E}$.
  \end{itemize}
  (For $p=1, 2, 3$, m is length, area, volume, respectively).
\end{definition}

\begin{definition}[Regular Set Function]
  \label{def:chap11:regular_set_function}
  A non-negative additive set function $\phi$ on $\mathcal{E}$ is
  \textbf{regular} if for every $A \in \mathcal{E}$ and $\epsilon >
  0$, there exist sets $F, G \in \mathcal{E}$ such that F is closed,
  G is open, $F \subset A \subset G$, and $\phi(G) - \epsilon \le
  \phi(A) \le \phi(F) + \epsilon$.
\end{definition}

\begin{example}[Regular Set Functions]
  \label{ex:chap11:regular_examples}
  ~ % Using itemize
  \begin{itemize}
    \item[(a)] The pre-measure m defined in Def
      \ref{def:chap11:elementary_sets_premeasure} is regular on $\mathcal{E}$.
    \item[(b)] Let $\alpha: \R \to \R$ be monotonically increasing.
      Define $\mu([a, b)) = \alpha(b-) - \alpha(a-)$, etc., for
      intervals, and extend additively to $\mathcal{E}$. This
      Stieltjes pre-measure $\mu$ is regular on $\mathcal{E}$.
  \end{itemize}
\end{example}

% Goal: Extend regular $\mu$ on $\mathcal{E}$ to a countably additive
% measure on a $\sigma$-ring $\mathfrak{M}(\mu) \supset \mathcal{E}$.

\begin{definition}[Outer Measure]
  \label{def:chap11:outer_measure}
  Let $\mu$ be an additive, regular, non-negative, finite set
  function on $\mathcal{E}$. For any set $E \subset \R^p$, define the
  \textbf{outer measure} of E corresponding to $\mu$ as
  \[
    \mustar(E) = \inf \sum_{n=1}^\infty \mu(A_n),
  \]
  where the infimum is taken over all countable coverings $E \subset
  \bigcup_{n=1}^\infty A_n$ by open elementary sets $A_n$.
  Properties: $\mustar(E) \ge 0$; $E_1 \subset E_2 \implies
  \mustar(E_1) \le \mustar(E_2)$.
\end{definition}

\begin{theorem}[Properties of Outer Measure]
  \label{thm:chap11:outer_measure_properties}
  Let $\mustar$ be the outer measure corresponding to $\mu$ as in Def
  \ref{def:chap11:outer_measure}.
  \begin{itemize}
    \item[(a)] $\mustar$ is an extension of $\mu$: For every $A \in
      \mathcal{E}$, $\mustar(A) = \mu(A)$.
    \item[(b)] $\mustar$ is countably subadditive: If $E =
      \bigcup_{n=1}^\infty E_n$, then $\mustar(E) \le
      \sum_{n=1}^\infty \mustar(E_n)$.
  \end{itemize}
\end{theorem}
% Proof omitted. (Proof uses regularity of $\mu$).

% --- End of transcription chunk ---

% --- Previous content (up to Thm 11.8) from chapters/chapter11.tex above ---

\begin{definition}[Measurable Sets]
  \label{def:chap11:measurable_sets}
  For any $A, B \subset \R^p$, define their \textbf{symmetric
  difference} by $S(A, B) = (A - B) \cup (B - A)$.
  Define the "distance" between A and B by $d(A, B) = \mustar(S(A, B))$.
  We write $A_n \to A$ if $\lim_{n\to\infty} d(A, A_n) = 0$.

  A set A is \textbf{finitely $\mu$-measurable}, written $A \in
  \FMeasurable{\mu}$, if there exists a sequence $\{ A_n \}$ of
  elementary sets ($A_n \in \mathcal{E}$) such that $A_n \to A$.
  A set A is \textbf{$\mu$-measurable}, written $A \in
  \Measurable{\mu}$, if A is the union of a countable collection of
  finitely $\mu$-measurable sets.
\end{definition}

\begin{remark}[Distance Properties]
  \label{rem:chap11:distance_properties}
  $S(A, B)$ satisfies $S(A,B)=S(B,A)$, $S(A,A)=\emptyset$, and
  $S(A,B) \subset S(A,C) \cup S(C,B)$. This implies $d(A,B)=d(B,A)$,
  $d(A,A)=0$, and the triangle inequality $d(A,B) \le d(A,C) +
  d(C,B)$. Also $d(A, B)=0$ does not imply $A=B$ (e.g., if A is
  countable and B is empty for Lebesgue measure). Identifying sets
  with $d(A,B)=0$ makes the set of equivalence classes a metric
  space, and $\FMeasurable{\mu}$ is the closure of $\mathcal{E}$ in
  this space. Also, $|\mustar(A) - \mustar(B)| \le d(A,B)$ if
  $\mustar(A)$ or $\mustar(B)$ is finite.
\end{remark}

\begin{theorem}[Properties of Measurable Sets and Measure Extension]
  \label{thm:chap11:measurable_sets_extension}
  The collection $\Measurable{\mu}$ is a $\sigma$-ring. The set
  function $\mustar$ is countably additive on $\Measurable{\mu}$.
  We now define the \textbf{measure} $\mu(A)$ for $A \in
  \Measurable{\mu}$ by setting $\mu(A) = \mustar(A)$. This extends
  the original set function $\mu$ from $\mathcal{E}$ to a countably
  additive measure on the $\sigma$-ring $\Measurable{\mu}$.
  (The special case $\mu=m$ yields the \textbf{Lebesgue measure} on $\R^p$).
\end{theorem}
% Proof omitted. (Proof involves showing $\FMeasurable{\mu}$ is a
% ring, $\mustar$ is additive on it, extending to $\Measurable{\mu}$,
% and proving countable additivity).

\begin{remark}[Properties of $\Measurable{\mu}$ and $\mu$]
  \label{rem:chap11:measurable_properties}
  ~ % itemize
  \begin{itemize}
    \item[(a)] Every open set and every closed set in $\R^p$ belongs
      to $\Measurable{\mu}$.
    \item[(b)] If $A \in \Measurable{\mu}$ and $\epsilon > 0$, there
      exist $F \subset A \subset G$ where F is closed, G is open, and
      $\mu(G-A) < \epsilon$, $\mu(A-F) < \epsilon$.
    \item[(c)] A set is a \textbf{Borel set} if it can be obtained
      from open sets through countably many unions, intersections,
      and complementations. The collection $\mathcal{B}$ of Borel
      sets is the smallest $\sigma$-ring containing all open sets.
      Every Borel set is in $\Measurable{\mu}$ for any $\mu$.
    \item[(d)] If $A \in \Measurable{\mu}$, there exist Borel sets
      $F, G$ such that $F \subset A \subset G$ and $\mu(G-A) =
      \mu(A-F) = 0$. (Every measurable set is the union of a Borel
      set and a set of measure zero).
    \item[(e)] For any measure $\mu$, the collection of sets E with
      $\mu(E)=0$ (sets of measure zero) forms a $\sigma$-ring.
    \item[(f)] For Lebesgue measure m on $\R^1$, every countable set
      has measure 0. The Cantor set (Sec 2.44) is an example of an
      uncountable set with Lebesgue measure 0.
  \end{itemize}
\end{remark}

% --- End of transcription chunk ---

% --- Previous content (up to Rem 11.11) from chapters/chapter11.tex above ---

\section{Measure Spaces}

\begin{definition}[Measure Space]
  \label{def:chap11:measure_space}
  A set X is a \textbf{measure space} if there exists a $\sigma$-ring
  $\mathfrak{M}$ of subsets of X (called \textbf{measurable sets})
  and a non-negative countably additive set function $\mu$ (called a
  \textbf{measure}) defined on $\mathfrak{M}$.
  If, in addition, $X \in \mathfrak{M}$, then X is called a
  \textbf{measurable space}.
\end{definition}

\begin{remark}[Examples of Measure Spaces]
  \label{rem:chap11:measure_space_examples}
  Examples include $(\R^p, \Measurable{m}, m)$ where m is Lebesgue
  measure; or $(J, \mathcal{P}(J), \mu)$ where J is the set of
  positive integers, $\mathcal{P}(J)$ is the power set (all subsets),
  and $\mu(E)$ is the number of elements in E. Probability spaces are
  also measure spaces. Subsequent discussion assumes X is a measurable space.
\end{remark}

\section{Measurable Functions}

\begin{definition}[Measurable Function]
  \label{def:chap11:measurable_function}
  Let f be a function defined on the measurable space X, with values
  in the extended real number system $[-\infty, +\infty]$. The
  function f is said to be \textbf{measurable} if the set $\{
  \vect{x} \mid f(\vect{x}) > a \}$ is measurable (i.e., belongs to
  $\mathfrak{M}$) for every real $a$.
\end{definition}

\begin{example}[Continuous Functions are Measurable]
  \label{ex:chap11:continuous_implies_measurable}
  If $X = \R^p$ and $\mathfrak{M}$ is the $\sigma$-ring of Lebesgue
  measurable sets, every continuous real-valued function f on $X$ is
  measurable, since $\{ \vect{x} \mid f(\vect{x}) > a \}$ is an open
  set, hence measurable.
\end{example}

\begin{theorem}[Equivalent Conditions for Measurability]
  \label{thm:chap11:measurable_equivalences}
  Each of the following four conditions on a function $f: X \to
  [-\infty, +\infty]$ implies the other three:
  \begin{itemize}
    \item[(a)] $\{ \vect{x} \mid f(\vect{x}) > a \}$ is measurable
      for every real $a$.
    \item[(b)] $\{ \vect{x} \mid f(\vect{x}) \ge a \}$ is measurable
      for every real $a$.
    \item[(c)] $\{ \vect{x} \mid f(\vect{x}) < a \}$ is measurable
      for every real $a$.
    \item[(d)] $\{ \vect{x} \mid f(\vect{x}) \le a \}$ is measurable
      for every real $a$.
  \end{itemize}
\end{theorem}
% Proof omitted.

\begin{theorem}[Absolute Value]
  \label{thm:chap11:abs_f_measurable}
  If f is measurable, then $|f|$ is measurable.
\end{theorem}
% Proof omitted.

\begin{theorem}[Sup, Inf, Lim sup, Lim inf]
  \label{thm:chap11:sup_inf_limsup_liminf_measurable}
  Let $\{ f_n \}$ be a sequence of measurable functions on X. Put
  \begin{align*}
    g(\vect{x}) &= \sup_{n \ge 1} f_n(\vect{x}) \\
    k(\vect{x}) &= \inf_{n \ge 1} f_n(\vect{x}) \\
    h(\vect{x}) &= \limsup_{n \to \infty} f_n(\vect{x}) = \inf_{m \ge
    1} \left( \sup_{n \ge m} f_n(\vect{x}) \right) \\
    l(\vect{x}) &= \liminf_{n \to \infty} f_n(\vect{x}) = \sup_{m \ge
    1} \left( \inf_{n \ge m} f_n(\vect{x}) \right)
  \end{align*}
  Then g, k, h, and l are measurable.
\end{theorem}
% Proof omitted.

\begin{corollary}
  \label{cor:chap11:measurable_corollaries}
  ~ % itemize
  \begin{itemize}
    \item[(a)] If f and g are measurable, then $\max(f, g)$ and
      $\min(f, g)$ are measurable. In particular, $f^+ = \max(f, 0)$
      and $f^- = -\min(f, 0)$ are measurable.
    \item[(b)] The limit of a pointwise convergent sequence of
      measurable functions is measurable.
  \end{itemize}
\end{corollary}

% --- End of corrected transcription chunk ---

% --- Previous content (up to Cor 11.17) from chapters/chapter11.tex above ---

\begin{theorem}[Functions of Measurable Functions]
  \label{thm:chap11:func_of_measurable}
  Let f and g be measurable real-valued functions defined on X, let F
  be real and continuous on $\R^2$, and put $h(\vect{x}) =
  F(f(\vect{x}), g(\vect{x}))$ for $\vect{x} \in X$. Then h is measurable.
  In particular, $f+g$ and $fg$ are measurable.
\end{theorem}
% Proof omitted. (Proof uses that the preimage of an open set under F
% is open, hence a countable union of open intervals/rectangles, and
% considers the preimage under (f,g)).

% Remark about composition f(g(x)) potentially not being measurable
% omitted for conciseness.
% Remark about measurability depending only on the sigma-ring, not
% the measure, omitted.

\section{Simple Functions}

\begin{definition}[Simple Function, Characteristic Function]
  \label{def:chap11:simple_function}
  Let s be a real-valued function defined on X. If the range of s is
  finite, s is called a \textbf{simple function}.
  Let $E \subset X$. The \textbf{characteristic function}
  $\indicator{E}$ of E is defined by
  \[
    \indicator{E}(\vect{x}) =
    \begin{cases} 1 & (\vect{x} \in E), \\ 0 & (\vect{x} \notin E).
    \end{cases}
  \]
  If the range of a simple function s consists of distinct numbers
  $c_1, \dots, c_n$, and $E_i = \{ \vect{x} \mid s(\vect{x}) = c_i
  \}$, then $s = \sum_{i=1}^n c_i \indicator{E_i}$.
  The simple function s is measurable if and only if the sets $E_1,
  \dots, E_n$ are measurable.
\end{definition}

\begin{theorem}[Approximation by Simple Functions]
  \label{thm:chap11:approx_by_simple}
  Let f be a real function on X. There exists a sequence $\{ s_n \}$
  of simple functions such that $s_n(\vect{x}) \to f(\vect{x})$ as $n
  \to \infty$, for every $\vect{x} \in X$. If f is measurable, $\{
  s_n \}$ may be chosen to be a sequence of measurable simple
  functions. If $f \ge 0$, $\{ s_n \}$ may be chosen to be
  monotonically increasing ($0 \le s_1 \le s_2 \le \dots \le f$).
\end{theorem}
% Proof omitted. (Proof involves partitioning the range of f).
% Remark about uniform convergence if f is bounded omitted.

\section{Integration}

\begin{definition}[Lebesgue Integral: Non-negative Functions]
  \label{def:chap11:integral_nonnegative}
  Let $(X, \mathfrak{M}, \mu)$ be a measure space.
  If $s = \sum_{i=1}^n c_i \indicator{E_i}$ ($c_i > 0$) is a
  non-negative measurable simple function, and $E \in \mathfrak{M}$,
  define the integral of s over E as
  \[
    I_E(s) = \sum_{i=1}^n c_i \mu(E \cap E_i).
  \]
  If f is a non-negative measurable function on X, define the
  \textbf{Lebesgue integral} of f over $E \in \mathfrak{M}$ as
  \[
    \int_E f d\mu = \sup I_E(s),
  \]
  where the supremum is taken over all measurable simple functions s
  such that $0 \le s \le f$. (The value may be $+\infty$).
\end{definition}

\begin{remark}
  \label{rem:chap11:integral_simple_consistency}
  For a non-negative simple measurable function s, $\int_E s d\mu = I_E(s)$.
\end{remark}

% --- End of transcription chunk ---

% --- Previous content (up to Rem 11.21) from chapters/chapter11.tex above ---

\begin{definition}[Lebesgue Integral: General Functions]
  \label{def:chap11:integral_general}
  Let f be a measurable function on X, and $E \in \mathfrak{M}$.
  Consider the integrals of the non-negative functions $f^+ = \max(f,
  0)$ and $f^- = -\min(f, 0)$ (which are measurable by Cor
  \ref{cor:chap11:measurable_corollaries}(a)):
  \[
    \int_E f^+ d\mu, \quad \int_E f^- d\mu.
  \]
  If at least one of these integrals is finite, we define the
  \textbf{Lebesgue integral} of f over E as
  \[
    \int_E f d\mu = \int_E f^+ d\mu - \int_E f^- d\mu.
  \]
  If both integrals are finite, we say that f is \textbf{Lebesgue
  integrable} (or \textbf{summable}) on E with respect to $\mu$, and
  we write $f \in \mathcal{L}(\mu)$ on E. (If $\mu=m$ is Lebesgue
  measure, we write $f \in L$ on E).
  (Note: $f \in \mathcal{L}(\mu)$ on E if and only if $\int_E f d\mu$
  is finite).
\end{definition}

\begin{remark}[Properties of the Integral]
  \label{rem:chap11:integral_properties}
  Let E be a measurable set.
  \begin{itemize}
    \item[(a)] If f is measurable and bounded on E, and if $\mu(E) <
      +\infty$, then $f \in \mathcal{L}(\mu)$ on E.
    \item[(b)] If $a \le f(\vect{x}) \le b$ for $\vect{x} \in E$, and
      $\mu(E) < +\infty$, then $a\mu(E) \le \int_E f d\mu \le b\mu(E)$.
    \item[(c)] If $f, g \in \mathcal{L}(\mu)$ on E and $f(\vect{x})
      \le g(\vect{x})$ for $\vect{x} \in E$, then $\int_E f d\mu \le
      \int_E g d\mu$.
    \item[(d)] If $f \in \mathcal{L}(\mu)$ on E, then $cf \in
      \mathcal{L}(\mu)$ on E for finite constant c, and $\int_E cf
      d\mu = c \int_E f d\mu$.
    \item[(e)] If $\mu(E) = 0$ and f is measurable, then $\int_E f d\mu = 0$.
    \item[(f)] If $f \in \mathcal{L}(\mu)$ on E, $A \in
      \mathfrak{M}$, and $A \subset E$, then $f \in \mathcal{L}(\mu)$ on A.
  \end{itemize}
\end{remark}

\begin{theorem}[Integral as a Measure]
  \label{thm:chap11:integral_as_measure}
  ~ % itemize
  \begin{itemize}
    \item[(a)] Suppose f is measurable and non-negative on X. For $A
      \in \mathfrak{M}$, define $\phi(A) = \int_A f d\mu$. Then
      $\phi$ is countably additive on $\mathfrak{M}$.
    \item[(b)] The same conclusion holds if $f \in \mathcal{L}(\mu)$ on X.
  \end{itemize}
\end{theorem}
% Proof omitted. (Proof uses definition for simple functions, then
% monotone convergence for non-negative f, then decomposition $f=f^+ - f^-$).

\begin{corollary}
  \label{cor:chap11:integral_measure_zero_set}
  If $A, B \in \mathfrak{M}$, $B \subset A$, and $\mu(A - B) = 0$,
  then $\int_A f d\mu = \int_B f d\mu$ (if the integrals exist).
\end{corollary}

\begin{remark}[Almost Everywhere]
  \label{rem:chap11:almost_everywhere}
  The corollary shows sets of measure zero are negligible in
  integration. We write $f \sim g$ on E if $\{ \vect{x} \in E \mid
  f(\vect{x}) \ne g(\vect{x}) \}$ has measure zero. This is an
  equivalence relation. If $f \sim g$ on E, then $\int_A f d\mu =
  \int_A g d\mu$ for all measurable $A \subset E$.
  A property P holds \textbf{almost everywhere (a.e.)} on E if it
  holds for all $\vect{x} \in E - A$ where $\mu(A)=0$.
  If $f \in \mathcal{L}(\mu)$ on E, then $f(\vect{x})$ must be finite
  almost everywhere on E.
\end{remark}

\begin{theorem}[Absolute Value Inequality]
  \label{thm:chap11:integral_abs_value_ineq}
  If $f \in \mathcal{L}(\mu)$ on E, then $|f| \in \mathcal{L}(\mu)$ on E, and
  \[
    \left| \int_E f d\mu \right| \le \int_E |f| d\mu.
  \]
  (Integrability of f implies integrability of $|f|$.)
\end{theorem}
% Proof omitted. (Proof uses $f \le |f|$, $-f \le |f|$, or considers
% $c \int f d\mu$ where $|c|=1$).

% --- End of transcription chunk ---

% --- Previous content (up to Thm 11.26) from chapters/chapter11.tex above ---

\begin{theorem}[Domination Test for Integrability]
  \label{thm:chap11:integrability_domination}
  Suppose f is measurable on E, $|f| \le g$, and $g \in
  \mathcal{L}(\mu)$ on E. Then $f \in \mathcal{L}(\mu)$ on E.
\end{theorem}
% Proof omitted. (Uses $f^+ \le g, f^- \le g$).

\begin{theorem}[Lebesgue's Monotone Convergence Theorem (MCT)]
  \label{thm:chap11:monotone_convergence}
  Suppose $E \in \mathfrak{M}$. Let $\{ f_n \}$ be a sequence of
  measurable functions such that $0 \le f_1(\vect{x}) \le
  f_2(\vect{x}) \le \dots$ for $\vect{x} \in E$. Let f be defined by
  $f_n(\vect{x}) \to f(\vect{x})$ as $n \to \infty$ for $\vect{x} \in E$. Then
  \[
    \lim_{n \to \infty} \int_E f_n d\mu = \int_E f d\mu.
  \]
  (The limit and integral may be $+\infty$).
\end{theorem}
% Proof omitted. (Proof involves integrals of simple functions approximating f).

\begin{theorem}[Linearity of Integral]
  \label{thm:chap11:linearity_integral}
  Suppose $f = f_1 + f_2$, where $f_1, f_2 \in \mathcal{L}(\mu)$ on
  E. Then $f \in \mathcal{L}(\mu)$ on E, and
  \[
    \int_E f d\mu = \int_E f_1 d\mu + \int_E f_2 d\mu.
  \]
\end{theorem}
% Proof omitted. (Proof considers cases based on sign of f1, f2, uses MCT).

\begin{theorem}[Integration of Series]
  \label{thm:chap11:integration_series}
  Suppose $E \in \mathfrak{M}$. If $\{ f_n \}$ is a sequence of
  non-negative measurable functions and $f(\vect{x}) =
  \sum_{n=1}^\infty f_n(\vect{x})$ for $\vect{x} \in E$, then
  \[
    \int_E f d\mu = \sum_{n=1}^\infty \int_E f_n d\mu.
  \]
\end{theorem}
% Proof omitted. (Follows from MCT applied to partial sums).

\begin{theorem}[Fatou's Lemma]
  \label{thm:chap11:fatou_lemma}
  Suppose $E \in \mathfrak{M}$. If $\{ f_n \}$ is a sequence of
  non-negative measurable functions and $f(\vect{x}) = \liminf_{n \to
  \infty} f_n(\vect{x})$ for $\vect{x} \in E$, then
  \[
    \int_E f d\mu \le \liminf_{n \to \infty} \int_E f_n d\mu.
  \]
  (Strict inequality can occur).
\end{theorem}
% Proof omitted. (Proof applies MCT to $g_n = \inf_{i \ge n} f_i$).

\begin{theorem}[Lebesgue's Dominated Convergence Theorem (LDCT)]
  \label{thm:chap11:dominated_convergence}
  Suppose $E \in \mathfrak{M}$. Let $\{ f_n \}$ be a sequence of
  measurable functions such that $f_n(\vect{x}) \to f(\vect{x})$ as
  $n \to \infty$ for $\vect{x} \in E$. If there exists a function $g
  \in \mathcal{L}(\mu)$ on E such that $|f_n(\vect{x})| \le
  g(\vect{x})$ for $n=1, 2, 3, \dots$ and $\vect{x} \in E$, then
  \[
    \lim_{n \to \infty} \int_E f_n d\mu = \int_E f d\mu.
  \]
  (The conclusion also holds if $f_n \to f$ almost everywhere on E).
\end{theorem}
% Proof omitted. (Proof uses Fatou's Lemma on $g+f_n$ and $g-f_n$).

\begin{corollary}[Bounded Convergence Theorem]
  \label{cor:chap11:bounded_convergence}
  If $\mu(E) < +\infty$, $\{ f_n \}$ is uniformly bounded on E (i.e.,
  $|f_n(\vect{x})| \le M < \infty$ for all n, $\vect{x}$), and
  $f_n(\vect{x}) \to f(\vect{x})$ on E, then $\lim_{n \to \infty}
  \int_E f_n d\mu = \int_E f d\mu$.
\end{corollary}
% Proof omitted. (Follows from LDCT with g=M).

% --- End of transcription chunk ---

% --- Previous content (up to Thm 11.32) from chapters/chapter11.tex above ---

\section{Comparison with the Riemann Integral}

% Intro text condensed
The Lebesgue integral extends the Riemann integral, includes a larger
class of functions, and handles limit operations more easily. We
compare them on an interval $[a, b]$ with Lebesgue measure $\mu=m$.
We denote the Lebesgue integral by $\int_a^b f dx$ and the Riemann
integral by $\mathcal{R}\int_a^b f dx$.

\begin{theorem}[Riemann vs. Lebesgue Integral]
  \label{thm:chap11:riemann_vs_lebesgue}
  Let f be a real-valued function on $[a, b]$.
  \begin{itemize}
    \item[(a)] If f is Riemann-integrable on $[a, b]$ (written $f \in
      \mathcal{R}$ on $[a, b]$), then f is Lebesgue integrable on
      $[a, b]$ ($f \in L$ on $[a, b]$), f is measurable, and
      \[
        \int_a^b f dx = \mathcal{R}\int_a^b f dx.
      \]
    \item[(b)] Assume f is bounded on $[a, b]$. Then $f \in
      \mathcal{R}$ on $[a, b]$ if and only if f is continuous almost
      everywhere on $[a, b]$.
  \end{itemize}
\end{theorem}
% Proof omitted. (Proof relates upper/lower Riemann sums/integrals to
% limits of simple functions and uses LDCT).

\begin{remark}[Fundamental Theorem of Calculus for Lebesgue Integral]
  \label{rem:chap11:ftc_lebesgue}
  The connection between differentiation and integration largely
  holds for the Lebesgue integral:
  \begin{itemize}
    \item If $f \in L$ on $[a, b]$ and $F(x) = \int_a^x f(t) dt$ for
      $a \le x \le b$, then $F'(x) = f(x)$ almost everywhere on $[a, b]$.
    \item Conversely, if F is differentiable at \textit{every} point
      of $[a, b]$ and if $F' \in L$ on $[a, b]$, then $F(x) - F(a) =
      \int_a^x F'(t) dt$ for $a \le x \le b$.
  \end{itemize}
  (Proofs omitted, refer to specialized texts).
\end{remark}

\section{Integration of Complex Functions}

\begin{definition}[Measurable and Integrable Complex Functions]
  \label{def:chap11:complex_integration}
  Let $(X, \mathfrak{M}, \mu)$ be a measure space. A complex function
  $f = u + iv$ (u, v real) on X is \textbf{measurable} if u and v are
  measurable.
  If $f$ is measurable and $\int_X |f| d\mu < +\infty$, then f is
  \textbf{Lebesgue integrable} on $E \in \mathfrak{M}$, written $f
  \in \mathcal{L}(\mu)$ on E. The integral is defined as
  \[
    \int_E f d\mu = \int_E u d\mu + i \int_E v d\mu.
  \]
  (Note: $f \in \mathcal{L}(\mu)$ on E iff $u, v \in
    \mathcal{L}(\mu)$ on E, which holds iff $|f| \in \mathcal{L}(\mu)$
  on E since $|u|, |v| \le |f| \le |u| + |v|$).
\end{definition}

% --- Regenerated Remark Environment ---

\begin{remark}[Properties Extended to Complex Functions]
  \label{rem:chap11:complex_integral_properties}
  Properties like linearity (item (d) in
    Remark~\ref{rem:chap11:integral_properties} and
  \autoref{thm:chap11:linearity_integral}), integrability on subsets
  (item (f) in Remark~\ref{rem:chap11:integral_properties}), the
  integral over sets of measure zero being zero (item (e) in
  Remark~\ref{rem:chap11:integral_properties}), the domination test
  (\autoref{thm:chap11:integrability_domination}), the absolute value
  inequality (\autoref{thm:chap11:integral_abs_value_ineq}: $|\int_E
  f d\mu| \le \int_E |f| d\mu$), and the Dominated Convergence
  Theorem (\autoref{thm:chap11:dominated_convergence}) extend
  directly to complex integrable functions. Integrability of f still
  implies integrability of $|f|$.
  (\autoref{thm:chap11:integral_as_measure}(b) also extends).
\end{remark}

% --- End of Regenerated Remark ---

% --- End of transcription chunk ---

% --- Previous content (up to Thm 11.32) from chapters/chapter11.tex above ---

% Using \texorpdfstring for bookmark compatibility with hyperref package
\section{Functions of Class \texorpdfstring{$\mathcal{L}^2$}{L2}}

\begin{definition}[L2 Space and Norm]
  \label{def:chap11:L2_space}
  Let X be a measure space with measure $\mu$. We say a complex
  function f belongs to $\mathcal{L}^2(\mu)$ on X (or $f \in
  \Lpspace{2}$) if f is measurable and $\int_X |f|^2 d\mu < +\infty$.
  For $f \in \Lpspace{2}$, we define the $\mathcal{L}^2(\mu)$-norm of f as
  \[
    \norm{f} = \left\{ \int_X |f|^2 d\mu \right\}^{1/2}.
  \]
  (If $\mu=m$ is Lebesgue measure, we write $f \in L^2$).
\end{definition}

\begin{theorem}[Schwarz Inequality for Integrals]
  \label{thm:chap11:schwarz_inequality_integral}
  Suppose $f \in \Lpspace{2}$ and $g \in \Lpspace{2}$. Then $fg \in
  \mathcal{L}(\mu)$, and
  \[
    \int_X |fg| d\mu \le \norm{f} \norm{g}.
  \]
\end{theorem}
% Proof omitted. (Proof uses $\int (|f|+\lambda|g|)^2 d\mu \ge 0$).

\begin{theorem}[Triangle Inequality for L2 Norm]
  \label{thm:chap11:triangle_inequality_L2}
  If $f \in \Lpspace{2}$ and $g \in \Lpspace{2}$, then $f+g \in
  \Lpspace{2}$, and
  \[
    \norm{f+g} \le \norm{f} + \norm{g}.
  \]
\end{theorem}
% Proof omitted. (Proof uses Schwarz inequality on $\int
% f\overline{g} + \int \overline{f}g$).

\begin{remark}[L2 as a Metric Space]
  \label{rem:chap11:L2_metric_space}
  The L2-norm satisfies the properties of a norm, and $d(f, g) =
  \norm{f-g}$ defines a metric on $\Lpspace{2}$, provided we identify
  functions f and g if $\norm{f-g}=0$ (i.e., if $f = g$ almost
  everywhere). With this identification, $\Lpspace{2}$ is a metric space.
\end{remark}

\begin{theorem}[Density of Continuous Functions]
  \label{thm:chap11:density_continuous_L2}
  The continuous functions form a dense subset of $\Lpspace{2}$ on
  $[a, b]$ (with Lebesgue measure). That is, for any $f \in
  \Lpspace{2}$ on $[a, b]$ and any $\epsilon > 0$, there exists a
  function g, continuous on $[a, b]$, such that $\norm{f-g} < \epsilon$.
\end{theorem}
% Proof omitted. (Proof approximates characteristic functions of
% closed sets, then measurable sets, then simple functions, then f).

\begin{definition}[Orthonormal Sets and Fourier Series in L2]
  \label{def:chap11:orthonormal_L2}
  A sequence of complex functions $\{ \phi_n \}$ is an
  \textbf{orthonormal set} on a measure space X if $\phi_n \in
  \Lpspace{2}$ for all n, and
  \[
    \int_X \phi_n \overline{\phi}_m d\mu =
    \begin{cases} 0 & (n \ne m), \\ 1 & (n = m).
    \end{cases}
  \]
  If $f \in \Lpspace{2}$ and $c_n = \int_X f \overline{\phi}_n d\mu$
  ($n=1, 2, 3, \dots$), the series $\sum_{n=1}^\infty c_n \phi_n$ is
  called the \textbf{Fourier series} of f (relative to $\{ \phi_n
  \}$) and we write $f \sim \sum_{n=1}^\infty c_n \phi_n$.
  (The Bessel inequality $\sum |c_n|^2 \le \norm{f}^2$ holds for $f
  \in \Lpspace{2}$).
\end{definition}

% --- End of transcription chunk ---

% --- Previous content (up to Def 11.39) from chapters/chapter11.tex above ---

\begin{theorem}[Parseval's Theorem for $\mathcal{L}^2$]
  \label{thm:chap11:parseval_L2}
  Suppose $f(x) \sim \sum_{n=-\infty}^\infty c_n e^{inx}$, where $f
  \in \Lpspace{2}$ on $[-\pi, \pi]$ (with Lebesgue measure). Let
  $s_N(x) = \sum_{n=-N}^N c_n e^{inx}$ be the N-th partial sum. Then
  \begin{gather*}
    \lim_{N \to \infty} \norm{f - s_N} = 0, \\
    \sum_{n=-\infty}^\infty |c_n|^2 = \frac{1}{2\pi} \int_{-\pi}^\pi
    |f(x)|^2 dx.
  \end{gather*}
  (The first equation means convergence in the $\mathcal{L}^2(\mu)$ sense).
\end{theorem}
% Proof omitted. (Uses density Thm 11.38 and Parseval for continuous
% functions Thm 8.16).

\begin{corollary}
  \label{cor:chap11:fourier_uniqueness_L2}
  If $f \in \Lpspace{2}$ on $[-\pi, \pi]$ and all its Fourier
  coefficients are 0, i.e., $\int_{-\pi}^\pi f(x) e^{-inx} dx = 0$
  for all integers n, then $\norm{f} = 0$ (so $f=0$ almost everywhere).
  (Thus, two functions in $\Lpspace{2}$ with the same Fourier series
  differ at most on a set of measure zero).
\end{corollary}

\begin{definition}[Convergence and Cauchy Sequence in L2]
  \label{def:chap11:convergence_cauchy_L2}
  Let f and $f_n \in \Lpspace{2}$ ($n=1, 2, 3, \dots$).
  We say $\{ f_n \}$ \textbf{converges to f in $\mathcal{L}^2(\mu)$}
  if $\norm{f_n - f} \to 0$ as $n \to \infty$.
  We say $\{ f_n \}$ is a \textbf{Cauchy sequence in
  $\mathcal{L}^2(\mu)$} if for every $\epsilon > 0$, there exists N
  such that $n, m \ge N$ implies $\norm{f_n - f_m} \le \epsilon$.
\end{definition}

\begin{theorem}[Completeness of L2]
  \label{thm:chap11:completeness_L2}
  If $\{ f_n \}$ is a Cauchy sequence in $\Lpspace{2}$, then there
  exists a function $f \in \Lpspace{2}$ such that $\{ f_n \}$
  converges to f in $\Lpspace{2}$.
  (In other words, $\Lpspace{2}$ is a complete metric space).
\end{theorem}
% Proof omitted. (Proof involves finding a subsequence that converges
% pointwise a.e., showing its limit f is in L2, and that the original
% sequence converges to f in L2 norm).

\begin{theorem}[Riesz-Fischer Theorem]
  \label{thm:chap11:riesz_fischer}
  Let $\{ \phi_n \}$ be orthonormal on X. Suppose $\sum_{n=1}^\infty
  |c_n|^2$ converges. Put $s_N = \sum_{n=1}^N c_n \phi_n$. Then there
  exists a function $f \in \Lpspace{2}$ such that $\{ s_N \}$
  converges to f in $\Lpspace{2}$, and such that $f \sim
  \sum_{n=1}^\infty c_n \phi_n$.
\end{theorem}
% Proof omitted. (Uses completeness Thm 11.42, shows $\{s_N\}$ is Cauchy).

\begin{definition}[Complete Orthonormal Set]
  \label{def:chap11:complete_orthonormal_set}
  An orthonormal set $\{ \phi_n \}$ on X is said to be
  \textbf{complete} if, for $f \in \Lpspace{2}$, the condition
  $\int_X f \overline{\phi}_n d\mu = 0$ for all $n=1, 2, 3, \dots$
  implies that $\norm{f} = 0$.
\end{definition}

\begin{theorem}[Parseval for Complete Orthonormal Sets]
  \label{thm:chap11:parseval_complete_ons}
  Let $\{ \phi_n \}$ be a complete orthonormal set. If $f \in
  \Lpspace{2}$ and $f \sim \sum_{n=1}^\infty c_n \phi_n$, then
  \[
    \int_X |f|^2 d\mu = \sum_{n=1}^\infty |c_n|^2.
  \]
\end{theorem}
% Proof omitted. (Uses Riesz-Fischer Thm 11.43 and completeness Def 11.44).

\begin{remark}[Hilbert Space Interpretation]
  \label{rem:chap11:hilbert_space}
  The Riesz-Fischer theorem (\ref{thm:chap11:riesz_fischer}) and
  Parseval's theorem (\ref{thm:chap11:parseval_complete_ons})
  together establish a 1-1 correspondence between functions $f \in
  \Lpspace{2}$ (identifying functions equal a.e.) and sequences $\{
  c_n \}$ with $\sum |c_n|^2 < \infty$. The space $\Lpspace{2}$ can
  be viewed as an infinite-dimensional Euclidean space (a Hilbert
  space) where f corresponds to the vector with coordinates $\{ c_n
  \}$ relative to the basis $\{ \phi_n \}$.
\end{remark}

% --- End of transcription chunk ---
