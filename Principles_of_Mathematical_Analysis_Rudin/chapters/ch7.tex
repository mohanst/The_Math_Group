% --- chapters/chapter7.tex ---
% Generated by Gemini (Google AI) on 2025-04-05.
% Based on principles_of_mathematical_analysis_walter_rudin_ch_7.pdf
% Assumes macros from user-provided macros.tex (incl. \contfunc{}) are defined.
% Proofs are omitted as requested. Using \norm{} for vector norms if applicable.

\chapter{Sequences and Series of Functions}
\label{chap:rudin7}

% In the present chapter we confine our attention to complex-valued functions
% (including the real-valued ones, of course), although many of the theorems and
% proofs which follow extend without difficulty to vector-valued functions, and
% even to mappings into general metric spaces. We choose to stay within this
% simple framework in order to focus attention on the most important aspects of
% the problems that arise when limit processes are interchanged.

\section{Discussion of Main Problem}
\label{sec:chap7:main_problem}

\begin{definition}[Pointwise Convergence] % Definition 7.1 from Ch 7 PDF
  \label{def:chap7:pointwise_convergence}
  Suppose $\sequence{f}$, $n=1, 2, 3, \dots$, is a sequence of
  functions defined on a set E, and suppose that the sequence of
  numbers $\sequence{f(x)}$ converges for every $x \in E$. We can
  then define a function f by
  \begin{equation} \label{eq:chap7:pointwise_limit_func}
    f(x) = \lim_{n \to \infty} f_n(x) \quad (x \in E).
  \end{equation}
  Under these circumstances we say that $\sequence{f}$ converges on E
  and that f is the limit, or the limit function, of $\sequence{f}$.
  Sometimes we shall use a more descriptive terminology and shall say
  that "$\sequence{f}$ converges to f pointwise on E" if
  \eqref{eq:chap7:pointwise_limit_func} holds.
  Similarly, if $\sum f_n(x)$ converges for every $x \in E$, and if we define
  \begin{equation} \label{eq:chap7:pointwise_limit_series}
    f(x) = \sum_{n=1}^\infty f_n(x) \quad (x \in E),
  \end{equation}
  the function f is called the sum of the series $\sum f_n$.
\end{definition}

% The main problem which arises is to determine whether important
% properties of functions are preserved under the limit operations (1) and (2).
% For instance, if the functions $f_n$ are continuous, or
% differentiable, or integrable,
% is the same true of the limit function? What are the relations
% between $f_n'$ and $f'$,
% say, or between the integrals of $f_n$ and that of f?
%
% To say that f is continuous at a limit point x means
% \[ \lim_{t \to x} f(t) = f(x). \]
% Hence, to ask whether the limit of a sequence of continuous functions is con-
% tinuous is the same as to ask whether
% \begin{equation} \label{eq:chap7:limit_interchange}
% \lim_{t \to x} \lim_{n \to \infty} f_n(t) = \lim_{n \to \infty}
% \lim_{t \to x} f_n(t),
% \end{equation}
% i.e., whether the order in which limit processes are carried out is
% immaterial.
% On the left side of (3), we first let $n \to \infty$, then $t \to
% x$; on the right side, $t \to x$
% first, then $n \to \infty$.
%
% We shall now show by means of several examples that limit processes
% cannot in general be interchanged without affecting the result. Afterward, we
% shall prove that under certain conditions the order in which limit operations
% are carried out is immaterial.

\begin{example}[Double Sequence Limit Interchange] % Example 7.2 from Ch 7 PDF
  \label{ex:chap7:double_sequence}
  For $m=1, 2, 3, \dots, n=1, 2, 3, \dots$, let
  \[ s_{m,n} = \frac{m}{m+n}. \]
  Then, for every fixed n,
  \[ \lim_{m \to \infty} s_{m,n} = 1, \]
  so that
  \begin{equation} \label{eq:chap7:double_seq_lim1}
    \lim_{n \to \infty} \lim_{m \to \infty} s_{m,n} = 1.
  \end{equation}
  On the other hand, for every fixed m,
  \[ \lim_{n \to \infty} s_{m,n} = 0, \]
  so that
  \begin{equation} \label{eq:chap7:double_seq_lim2}
    \lim_{m \to \infty} \lim_{n \to \infty} s_{m,n} = 0.
  \end{equation}
\end{example}

\begin{example}[Series Limit vs Function Limit] % Example 7.3 from Ch 7 PDF
  \label{ex:chap7:series_limit}
  Let
  \[ f_n(x) = \frac{x^2}{(1+x^2)^n} \quad (x \text{ real}; n=0, 1, 2, \dots), \]
  and consider
  \begin{equation} \label{eq:chap7:series_limit_func}
    f(x) = \sum_{n=0}^\infty f_n(x) = \sum_{n=0}^\infty \frac{x^2}{(1+x^2)^n}.
  \end{equation}
  Since $f_n(0) = 0$, we have $f(0)=0$. For $x \ne 0$, the last
  series in \eqref{eq:chap7:series_limit_func} is a convergent
  geometric series with sum $x^2 / (1 - 1/(1+x^2)) = x^2 /
  (x^2/(1+x^2)) = 1+x^2$ (compare \autoref{thm:chap3:geometric_series}). Hence
  \begin{equation} \label{eq:chap7:series_limit_result}
    f(x) =
    \begin{cases} 0 & (x=0), \\ 1+x^2 & (x \ne 0),
    \end{cases}
  \end{equation}
  so that a convergent series of continuous functions may have a
  discontinuous sum.
\end{example}

\begin{example}[Dirichlet Function as Limit] % Example 7.4 from Ch 7 PDF
  \label{ex:chap7:dirichlet_limit}
  For $m=1, 2, 3, \dots$, put
  \[ f_m(x) = \lim_{n \to \infty} (\cos(m! \pi x))^{2n}. \]
  When $m!x$ is an integer, $f_m(x)=1$. For all other values of x,
  $f_m(x)=0$. Now let
  \[ f(x) = \lim_{m \to \infty} f_m(x). \]
  For irrational x, $f_m(x)=0$ for every m; hence $f(x)=0$. For
  rational x, say $x=p/q$, where p and q are integers, we see that
  $m!x$ is an integer if $m \ge q$, so that $f(x)=1$. Hence
  \begin{equation} \label{eq:chap7:dirichlet_result}
    \lim_{m \to \infty} \lim_{n \to \infty} (\cos(m! \pi x))^{2n} =
    \begin{cases} 0 & (x \text{ irrational}), \\ 1 & (x \text{ rational}).
    \end{cases}
  \end{equation}
  We have thus obtained an everywhere discontinuous limit function,
  which is not Riemann-integrable. %(Exercise 4, Chap. 6).
\end{example}

\begin{example}[Limit of Derivatives vs Derivative of Limit] %
  % Example 7.5 from Ch 7 PDF
  \label{ex:chap7:deriv_limit_swap}
  Let
  \begin{equation} \label{eq:chap7:deriv_limit_fn}
    f_n(x) = \frac{\sin(nx)}{\sqrt{n}} \quad (x \text{ real}, n=1, 2, 3, \dots),
  \end{equation}
  and
  \[ f(x) = \lim_{n \to \infty} f_n(x) = 0. \]
  Then $f'(x)=0$, and
  \[ f_n'(x) = \sqrt{n} \cos(nx), \]
  so that $\sequence{f'}$ does not converge to $f'$. For instance,
  \[ f_n'(0) = \sqrt{n} \to +\infty \]
  as $n \to \infty$, whereas $f'(0)=0$.
\end{example}

\begin{example}[Limit of Integral vs Integral of Limit] % Example 7.6
  % from Ch 7 PDF
  \label{ex:chap7:integral_limit_swap}
  Let
  \begin{equation} \label{eq:chap7:integral_limit_fn}
    f_n(x) = n^2 x (1-x^2)^n \quad (0 \le x \le 1, n=1, 2, 3, \dots).
  \end{equation}
  For $0 < x \le 1$, we have
  \[ \lim_{n \to \infty} f_n(x) = 0, \]
  by \autoref{thm:chap3:special_limits}(d) (since $0 \le 1-x^2 < 1$).
  Since $f_n(0)=0$, we see that
  \begin{equation} \label{eq:chap7:integral_limit_pointwise}
    \lim_{n \to \infty} f_n(x) = 0 \quad (0 \le x \le 1).
  \end{equation}
  A simple calculation shows that
  \[ \int_0^1 x(1-x^2)^n \, dx = \left[ -\frac{(1-x^2)^{n+1}}{2(n+1)}
  \right]_0^1 = \frac{1}{2n+2}. \]
  Thus, in spite of \eqref{eq:chap7:integral_limit_pointwise},
  \[ \int_0^1 f_n(x) \, dx = n^2 \int_0^1 x(1-x^2)^n \, dx =
  \frac{n^2}{2n+2} \to +\infty \]
  as $n \to \infty$.
  If, in \eqref{eq:chap7:integral_limit_fn}, we replace $n^2$ by $n$,
  \eqref{eq:chap7:integral_limit_pointwise} still holds, but we now have
  \[ \lim_{n \to \infty} \int_0^1 f_n(x) \, dx = \lim_{n \to \infty}
  \frac{n}{2n+2} = \frac{1}{2}, \]
  whereas
  \[ \int_0^1 \left[ \lim_{n \to \infty} f_n(x) \right] \, dx =
  \int_0^1 0 \, dx = 0. \]
  Thus the limit of the integral need not be equal to the integral of
  the limit, even if both are finite.
\end{example}

% --- End of content chunk ---

% --- Content from Def 7.7 to Thm 7.12 (Proofs Omitted) to append ---

\section{Uniform Convergence}
\label{sec:chap7:uniform_convergence}

\begin{definition}[Uniform Convergence] % Definition 7.7 from Ch 7 PDF
  \label{def:chap7:uniform_convergence}
  We say that a sequence of functions $\sequence{f}$, $n=1, 2, 3,
  \dots$, converges uniformly on E to a function f if for every
  $\epsilon > 0$ there is an integer N such that $n \ge N$ implies
  \begin{equation} \label{eq:chap7:uniform_conv_cond}
    \abs{f_n(x) - f(x)} \le \epsilon
  \end{equation}
  for all $x \in E$.
  It is clear that every uniformly convergent sequence is pointwise
  convergent (\autoref{def:chap7:pointwise_convergence}). Quite
  explicitly, the difference between the two concepts is this: If
  $\sequence{f}$ converges pointwise on E, then there exists a
  function f such that, for every $\epsilon > 0$, and for every $x
  \in E$, there is an integer N, depending on $\epsilon$ and on x,
  such that \eqref{eq:chap7:uniform_conv_cond} holds if $n \ge N$; if
  $\sequence{f}$ converges uniformly on E, it is possible, for each
  $\epsilon > 0$, to find one integer N which will do for all $x \in E$.

  We say that the series $\sum f_n(x)$ converges uniformly on E if
  the sequence $\sequence{s}$ of partial sums defined by
  \[ s_n(x) = \sum_{i=1}^n f_i(x) \]
  converges uniformly on E.
\end{definition}

\begin{theorem}[Cauchy Criterion for Uniform Convergence] % Theorem
  % 7.8 from Ch 7 PDF
  \label{thm:chap7:cauchy_uniform_conv}
  The sequence of functions $\sequence{f}$, defined on E, converges
  uniformly on E if and only if for every $\epsilon > 0$ there exists
  an integer N such that $m \ge N$, $n \ge N$, $x \in E$ implies
  \begin{equation} \label{eq:chap7:cauchy_uniform_cond}
    \abs{f_n(x) - f_m(x)} \le \epsilon.
  \end{equation}
  % Proof Omitted
\end{theorem}

\begin{theorem}[Uniform Convergence via Supremum Norm] % Theorem 7.9
  % from Ch 7 PDF
  \label{thm:chap7:uniform_conv_sup_norm}
  Suppose
  \[ \lim_{n \to \infty} f_n(x) = f(x) \quad (x \in E). \]
  Put
  \[ M_n = \sup_{x \in E} \abs{f_n(x) - f(x)}. \]
  Then $f_n \to f$ uniformly on E if and only if $M_n \to 0$ as $n \to \infty$.
  % Proof Omitted (Immediate consequence of Def 7.7)
\end{theorem}

\begin{theorem}[Weierstrass M-test] % Theorem 7.10 from Ch 7 PDF
  \label{thm:chap7:weierstrass_m_test}
  Suppose $\sequence{f}$ is a sequence of functions defined on E, and suppose
  \[ \abs{f_n(x)} \le M_n \quad (x \in E, n=1, 2, 3, \dots). \]
  Then $\sum f_n$ converges uniformly on E if $\sum M_n$ converges.
  (Note that the converse is not asserted (and is, in fact, not true).)
  % Proof Omitted (Uses Thm 7.8)
\end{theorem}

\section{Uniform Convergence and Continuity}
\label{sec:chap7:uniform_conv_continuity}

\begin{theorem}[Interchange of Limits for Functions] % Theorem 7.11
  % from Ch 7 PDF
  \label{thm:chap7:limit_interchange_functions}
  Suppose $f_n \to f$ uniformly on a set E in a metric space. Let x
  be a limit point of E, and suppose that
  \begin{equation} \label{eq:chap7:limit_fn_t}
    \lim_{t \to x} f_n(t) = A_n \quad (n=1, 2, 3, \dots).
  \end{equation}
  Then $\sequence{A}$ converges, and
  \begin{equation} \label{eq:chap7:limit_f_t_result}
    \lim_{t \to x} f(t) = \lim_{n \to \infty} A_n.
  \end{equation}
  In other words, the conclusion is that
  \begin{equation} \label{eq:chap7:limit_interchange_explicit}
    \lim_{t \to x} \lim_{n \to \infty} f_n(t) = \lim_{n \to \infty}
    \lim_{t \to x} f_n(t).
  \end{equation}
  % Proof Omitted
\end{theorem}

\begin{theorem}[Uniform Limit of Continuous Functions is Continuous]
  % Theorem 7.12 from Ch 7 PDF
  \label{thm:chap7:uniform_limit_continuous}
  If $\sequence{f}$ is a sequence of continuous functions on E, and
  if $f_n \to f$ uniformly on E, then f is continuous on E.
  (This very important result is an immediate corollary of
  \autoref{thm:chap7:limit_interchange_functions}.)
  % Proof Omitted
\end{theorem}

% --- End of content chunk ---

% --- Content from Thm 7.13 to Cor 7.16 (Proofs Omitted) to append ---

% Note: Rudin mentions that the converse to Thm 7.12 is not true, citing Ex 7.6,
% but that the following theorem gives a case where it holds.

\begin{theorem}[Dini's Theorem] % Theorem 7.13 from Ch 7 PDF
  \label{thm:chap7:dinis_theorem}
  Suppose K is compact, and
  (a) $\sequence{f}$ is a sequence of continuous functions on K,
  (b) $\sequence{f}$ converges pointwise to a continuous function f on K,
  (c) $f_n(x) \ge f_{n+1}(x)$ for all $x \in K$, $n=1, 2, 3, \dots$.
  Then $f_n \to f$ uniformly on K.
  % Proof Omitted (Uses properties of compact sets K_n = {x |
  % f_n(x)-f(x) >= epsilon})
\end{theorem}

% Note: Rudin provides a counterexample for non-compact domain:
% f_n(x)=1/(nx+1) on (0,1).

\begin{definition}[Space C(X) and Supremum Norm] % Definition 7.14 from Ch 7 PDF
  \label{def:chap7:space_C_X}
  If X is a metric space, $\contfunc{X}$ will denote the set of all
  complex-valued, continuous, bounded functions with domain X.
  [Note that boundedness is redundant if X is compact
  (\autoref{thm:chap4:image_compact_is_compact} Corollary).] Thus
  $\contfunc{X}$ consists of all complex continuous functions on X if
  X is compact.
  We associate with each $f \in \contfunc{X}$ its supremum norm
  \[ \norm{f} = \sup_{x \in X} \abs{f(x)}. \]
  Since f is assumed to be bounded, $\norm{f} < \infty$. It is
  obvious that $\norm{f}=0$ only if $f(x)=0$ for every $x \in X$,
  that is, only if $f=0$. If $h=f+g$, then
  \[ \abs{h(x)} \le \abs{f(x)} + \abs{g(x)} \le \norm{f} + \norm{g} \]
  for all $x \in X$; hence
  \[ \norm{f+g} \le \norm{f} + \norm{g}. \]
  If we define the distance between $f \in \contfunc{X}$ and $g \in
  \contfunc{X}$ to be $\norm{f-g}$, it follows that Axioms 2.15 for a
  metric are satisfied. We have thus made $\contfunc{X}$ into a metric space.
  \autoref{thm:chap7:uniform_conv_sup_norm} can be rephrased as follows:
  A sequence $\sequence{f}$ converges to f with respect to the metric
  of $\contfunc{X}$ if and only if $f_n \to f$ uniformly on X.
  Accordingly, closed subsets of $\contfunc{X}$ are sometimes called
  uniformly closed, the closure of a set $A \subset \contfunc{X}$ is
  called its uniform closure, and so on.
\end{definition}

\begin{theorem}[Completeness of C(X)] % Theorem 7.15 from Ch 7 PDF
  \label{thm:chap7:completeness_C_X}
  The metric space $\contfunc{X}$ defined in
  \autoref{def:chap7:space_C_X} is a complete metric space.
  % Proof Omitted (Uses Thm 7.8 and Thm 7.12)
\end{theorem}

\section{Uniform Convergence and Integration}
\label{sec:chap7:uniform_conv_integration}

\begin{theorem}[Uniform Convergence and Integration Interchange] %
  % Theorem 7.16 from Ch 7 PDF
  \label{thm:chap7:uniform_conv_integral}
  Let $\alpha$ be monotonically increasing on $[a, b]$. Suppose $f_n
  \in \RSintegrable{\alpha}$ on $[a, b]$, for $n=1, 2, 3, \dots$, and
  suppose $f_n \to f$ uniformly on $[a, b]$. Then $f \in
  \RSintegrable{\alpha}$ on $[a, b]$, and
  \begin{equation} \label{eq:chap7:integral_limit_result}
    \int_a^b f \, d\alpha = \lim_{n \to \infty} \int_a^b f_n \, d\alpha.
  \end{equation}
  (The existence of the limit is part of the conclusion.)
  % Proof Omitted (Uses sup norm epsilon_n and Def 6.2 integral bounds)
\end{theorem}

\begin{corollary}[Term-by-Term Integration] % Corollary to 7.16 from Ch 7 PDF
  \label{cor:chap7:term_by_term_integration}
  If $f_n \in \RSintegrable{\alpha}$ on $[a, b]$ and if
  \[ f(x) = \sum_{n=1}^\infty f_n(x) \quad (a \le x \le b), \]
  the series converging uniformly on $[a, b]$, then
  \[ \int_a^b f \, d\alpha = \sum_{n=1}^\infty \int_a^b f_n \, d\alpha. \]
  In other words, the series may be integrated term by term.
\end{corollary}

% --- End of content chunk ---

% --- Content from Thm 7.17 to Ex 7.21 (Proofs Omitted) to append ---

\section{Uniform Convergence and Differentiation}
\label{sec:chap7:uniform_conv_differentiation}

% We have already seen, in Example 7.5, that uniform convergence of
% {f_n} implies
% nothing about the sequence {f'_n}. Thus stronger hypotheses are
% required for the
% assertion that f'_n -> f' if f_n -> f.

\begin{theorem}[Interchange of Limit and Differentiation] % Theorem
  % 7.17 from Ch 7 PDF
  \label{thm:chap7:limit_differentiation}
  Suppose $\sequence{f}$ is a sequence of functions, differentiable
  on $[a, b]$ and such that $\sequence{f(x_0)}$ converges for some
  point $x_0$ on $[a, b]$. If $\sequence{f'}$ converges uniformly on
  $[a, b]$, then $\sequence{f}$ converges uniformly on $[a, b]$, to a
  function f, and
  \begin{equation} \label{eq:chap7:deriv_limit_result}
    f'(x) = \lim_{n \to \infty} f'_n(x) \quad (a \le x \le b).
  \end{equation}
  % Proof Omitted (Uses Mean Value Theorem and Thm 7.11)
\end{theorem}

\begin{theorem}[Weierstrass Non-Differentiable Function] % Theorem
  % 7.18 from Ch 7 PDF
  \label{thm:chap7:weierstrass_no_diff_func}
  There exists a real continuous function on the real line which is
  nowhere differentiable.
  % The text provides a specific construction:
  % Define phi(x) = |x| on [-1, 1] and phi(x+2) = phi(x).
  % Define f(x) = sum_{n=0}^infty (3/4)^n * phi(4^n x).
  % f is continuous by uniform convergence (Thm 7.10, 7.12).
  % Proof of non-differentiability omitted.
\end{theorem}

\section{Equicontinuous Families of Functions}
\label{sec:chap7:equicontinuous}

% In Theorem 3.6 we saw that every bounded sequence of complex numbers
% contains a convergent subsequence, and the question arises whether something
% similar is true for sequences of functions. To make the question more precise,
% we shall define two kinds of boundedness.

\begin{definition}[Pointwise and Uniformly Bounded] % Definition 7.19
  % from Ch 7 PDF
  \label{def:chap7:boundedness_types}
  Let $\sequence{f}$ be a sequence of functions defined on a set E.
  We say that $\sequence{f}$ is pointwise bounded on E if the
  sequence $\sequence{f(x)}$ is bounded for every $x \in E$, that is,
  if there exists a finite-valued function $\phi$ defined on E such that
  \[ \abs{f_n(x)} < \phi(x) \quad (x \in E, n=1, 2, 3, \dots). \]
  We say that $\sequence{f}$ is uniformly bounded on E if there
  exists a number M such that
  \[ \abs{f_n(x)} < M \quad (x \in E, n=1, 2, 3, \dots). \]
\end{definition}

% Now if {f_n} is pointwise bounded on E and E_1 is a countable subset of E,
% it is always possible to find a subsequence {f_{n_k}} such that
% {f_{n_k}(x)} converges for
% every x in E_1. This can be done by the diagonal process which is used in the
% proof of Theorem 7.23.
%
% However, even if {f_n} is a uniformly bounded sequence of continuous
% functions on a compact set E, there need not exist a subsequence which con-
% verges pointwise on E.

\begin{example}[Bounded Sequence with No Pointwise Convergent
  Subsequence] % Example 7.20 from Ch 7 PDF
  \label{ex:chap7:sin_nx_no_conv_subseq}
  Let
  \[ f_n(x) = \sin(nx) \quad (0 \le x \le 2\pi, n=1, 2, 3, \dots). \]
  Suppose there exists a sequence $\sequence{n}$ such that
  $\{\sin(n_k x)\}$ converges, for every $x \in [0, 2\pi]$. In that
  case we must have
  \[ \lim_{k \to \infty} (\sin(n_k x) - \sin(n_{k+1} x)) = 0 \quad (0
  \le x \le 2\pi); \]
  hence
  \begin{equation} \label{eq:chap7:sin_diff_sq_limit}
    \lim_{k \to \infty} (\sin(n_k x) - \sin(n_{k+1} x))^2 = 0 \quad
    (0 \le x \le 2\pi).
  \end{equation}
  By Lebesgue's theorem concerning integration of boundedly
  convergent sequences (\autoref{thm:11.32} - Note: Reference to Ch
  11), \eqref{eq:chap7:sin_diff_sq_limit} implies
  \begin{equation} \label{eq:chap7:sin_int_limit}
    \lim_{k \to \infty} \int_0^{2\pi} (\sin(n_k x) - \sin(n_{k+1}
    x))^2 \, dx = 0.
  \end{equation}
  But a simple calculation shows that
  \[ \int_0^{2\pi} (\sin(n_k x) - \sin(n_{k+1} x))^2 \, dx =
    \int_0^{2\pi} (\sin^2(n_k x) - 2\sin(n_k x)\sin(n_{k+1} x) +
  \sin^2(n_{k+1} x)) \, dx \]
  Assuming $n_k \ne n_{k+1}$, the middle term integrates to 0, and
  $\int_0^{2\pi} \sin^2(mx) dx = \pi$. So the integral is $2\pi$,
  which contradicts \eqref{eq:chap7:sin_int_limit}.
\end{example}

% Another question is whether every convergent sequence contains a
% uniformly convergent subsequence. Our next example will show that this
% need not be so, even if the sequence is uniformly bounded on a compact set.

\begin{example}[Uniformly Bounded, Pointwise Conv, Not Uniformly Conv
  Subseq] % Example 7.21 from Ch 7 PDF
  \label{ex:chap7:pointwise_not_uniform_conv}
  Let
  \[ f_n(x) = \frac{x^2}{x^2 + (1 - nx)^2} \quad (0 \le x \le 1, n=1,
  2, 3, \dots). \]
  Then $\abs{f_n(x)} \le 1$, so that $\sequence{f}$ is uniformly
  bounded on $[0, 1]$. Also
  \[ \lim_{n \to \infty} f_n(x) = 0 \quad (0 \le x \le 1), \]
  but
  \[ f_n(1/n) = \frac{(1/n)^2}{(1/n)^2 + (1 - n(1/n))^2} =
  \frac{1/n^2}{1/n^2 + 0} = 1 \quad (n=1, 2, 3, \dots), \]
  so that $\sup \abs{f_n(x) - 0} \ge 1$ for all n. By
  \autoref{thm:chap7:uniform_conv_sup_norm}, the convergence is not
  uniform, and no subsequence can converge uniformly to 0 on $[0, 1]$.
\end{example}

% --- End of content chunk ---

% --- Content from Def 7.22 to Thm 7.25 (Proofs Omitted) to append ---

% The concept which is needed in this connection is that of equicontinuity;
% it is given in the following definition.

\begin{definition}[Equicontinuity] % Definition 7.22 from Ch 7 PDF
  \label{def:chap7:equicontinuity}
  A family $\collection{F}$ of complex functions defined on a set E
  in a metric space X is said to be equicontinuous on E if for every
  $\epsilon > 0$ there exists a $\delta > 0$ such that
  \[ \abs{f(x) - f(y)} < \epsilon \]
  whenever $d(x, y) < \delta$, $x \in E$, $y \in E$, and $f \in
  \collection{F}$. Here d denotes the metric of X.
  It is clear that every member of an equicontinuous family is
  uniformly continuous.
  (The sequence of Example 7.21 is not equicontinuous.)
\end{definition}

% Theorems 7.24 and 7.25 will show that there is a very close relation
% between equicontinuity, on the one hand, and uniform convergence of sequences
% of continuous functions, on the other. But first we describe a
% selection process
% which has nothing to do with continuity.

\begin{theorem}[Helly's Selection Theorem - Diagonal Process] %
  % Theorem 7.23 from Ch 7 PDF
  \label{thm:chap7:hellys_selection}
  If $\sequence{f}$ is a pointwise bounded sequence of complex
  functions on a countable set E, then $\sequence{f}$ has a
  subsequence $\sequence{f \circ n}$ such that $\sequence{f \circ
  n(x)}$ converges for every $x \in E$.
  % Proof Omitted (Uses diagonal process on sequences converging at
  % each point x_i)
\end{theorem}

\begin{theorem}[Uniform Convergence implies Equicontinuity] % Theorem
  % 7.24 from Ch 7 PDF
  \label{thm:chap7:unif_conv_implies_equicont}
  If K is a compact metric space, if $f_n \in \contfunc{K}$ for $n=1,
  2, 3, \dots$, and if $\sequence{f}$ converges uniformly on K, then
  $\sequence{f}$ is equicontinuous on K.
  % Proof Omitted
\end{theorem}

\begin{theorem}[Arzelà-Ascoli Variant] % Theorem 7.25 from Ch 7 PDF
  \label{thm:chap7:arzela_ascoli}
  If K is compact, if $f_n \in \contfunc{K}$ for $n=1, 2, 3, \dots$,
  and if $\sequence{f}$ is pointwise bounded and equicontinuous on K, then
  (a) $\sequence{f}$ is uniformly bounded on K,
  (b) $\sequence{f}$ contains a uniformly convergent subsequence.
  % Proof Omitted (Part (a) uses compactness and equicontinuity. Part
  % (b) uses density of countable subset, Thm 7.23, and equicontinuity.)
\end{theorem}

% --- End of content chunk ---

% --- Content from Thm 7.26 to Thm 7.31 (Proofs Omitted) to append ---

\section{The Stone-Weierstrass Theorem}
\label{sec:chap7:stone_weierstrass}

\begin{theorem}[Weierstrass Approximation Theorem] % Theorem 7.26 from Ch 7 PDF
  \label{thm:chap7:weierstrass_approx}
  If f is a continuous complex function on $[a, b]$, there exists a
  sequence of polynomials $P_n$ such that
  \[ \lim_{n \to \infty} P_n(x) = f(x) \]
  uniformly on $[a, b]$. If f is real, the $P_n$ may be taken real.
  % Proof Omitted (Uses convolution with an approximate identity Q_n(x))
\end{theorem}

\begin{corollary} % Corollary 7.27 from Ch 7 PDF
  \label{cor:chap7:abs_x_poly_approx}
  For every interval $[-a, a]$ there is a sequence of real
  polynomials $P_n$ such that $P_n(0) = 0$ and such that
  \[ \lim_{n \to \infty} P_n(x) = \abs{x} \]
  uniformly on $[-a, a]$.
  % Proof Omitted (Uses Thm 7.26)
\end{corollary}

% We shall now isolate those properties of the polynomials which make
% the Weierstrass theorem possible.

\begin{definition}[Algebra of Functions, Uniform Closure] %
  % Definition 7.28 from Ch 7 PDF
  \label{def:chap7:algebra_uniform_closure}
  A family $\collection{A}$ of complex functions defined on a set E
  is said to be an algebra if (i) $f+g \in \collection{A}$, (ii) $fg
  \in \collection{A}$, and (iii) $cf \in \collection{A}$ for all $f
  \in \collection{A}$, $g \in \collection{A}$ and for all complex
  constants c, that is, if $\collection{A}$ is closed under addition,
  multiplication, and scalar multiplication.
  (We shall also have to consider algebras of real functions; in this
  case, (iii) is of course only required to hold for all real c.)
  If $\collection{A}$ has the property that $f \in \collection{A}$
  whenever $f_n \in \collection{A}$ ($n=1, 2, 3, \dots$) and $f_n \to
  f$ uniformly on E, then $\collection{A}$ is said to be uniformly closed.
  Let $\collection{B}$ be the set of all functions which are limits
  of uniformly convergent sequences of members of $\collection{A}$.
  Then $\collection{B}$ is called the uniform closure of
  $\collection{A}$. (See \autoref{def:chap7:space_C_X}.)
\end{definition}

% For example, the set of all polynomials is an algebra, and the Weierstrass
% theorem may be stated by saying that the set of continuous functions on [a, b]
% is the uniform closure of the set of polynomials on [a, b].

\begin{theorem}[Uniform Closure of Algebra is Algebra] % Theorem 7.29
  % from Ch 7 PDF
  \label{thm:chap7:closure_of_algebra}
  Let $\collection{B}$ be the uniform closure of an algebra
  $\collection{A}$ of bounded functions. Then $\collection{B}$ is a
  uniformly closed algebra.
  % Proof Omitted (Uses limit properties for uniform convergence)
\end{theorem}

\begin{definition}[Separates Points, Vanishes at No Point] %
  % Definition 7.30 from Ch 7 PDF
  \label{def:chap7:separates_points_vanishes_no_point}
  Let $\collection{F}$ be a family of functions on a set E. Then
  $\collection{F}$ is said to separate points on E if to every pair
  of distinct points $x_1, x_2 \in E$ there corresponds a function $f
  \in \collection{F}$ such that $f(x_1) \ne f(x_2)$.
  If to each $x \in E$ there corresponds a function $g \in
  \collection{F}$ such that $g(x) \ne 0$, we say that
  $\collection{F}$ vanishes at no point of E.
\end{definition}

% The algebra of all polynomials in one variable clearly has these properties
% on R^1. An example of an algebra which does not separate points is the set of
% all even polynomials, say on [-1, 1], since f(-x) = f(x) for every
% even function f.
% The following theorem will illustrate these concepts further.

\begin{theorem}[Existence of Function with Prescribed Values] %
  % Theorem 7.31 from Ch 7 PDF
  \label{thm:chap7:algebra_prescribed_values}
  Suppose $\collection{A}$ is an algebra of functions on a set E,
  $\collection{A}$ separates points on E, and $\collection{A}$
  vanishes at no point of E. Suppose $x_1, x_2$ are distinct points
  of E, and $c_1, c_2$ are constants (real if $\collection{A}$ is a
  real algebra). Then $\collection{A}$ contains a function f such
  that $f(x_1) = c_1$ and $f(x_2) = c_2$.
  % Proof Omitted (Constructs f using combinations of functions from A)
\end{theorem}

% --- End of content chunk ---

% --- Content from Thm 7.32 to Thm 7.33 (Proofs Omitted) to append ---

% We now have all the material needed for Stone's generalization of the
% Weierstrass theorem.

\begin{theorem}[Stone-Weierstrass Theorem - Real Case] % Theorem 7.32
  % from Ch 7 PDF
  \label{thm:chap7:stone_weierstrass_real}
  Let $\collection{A}$ be an algebra of real continuous functions on
  a compact set K. If $\collection{A}$ separates points on K and if
  $\collection{A}$ vanishes at no point of K, then the uniform
  closure $\collection{B}$ of $\collection{A}$ consists of all real
  continuous functions on K.
  % Proof Omitted (Involves 4 steps: |f| in closure, max/min in
  % closure, constructing local approximations, constructing global
  % approximation)
\end{theorem}

% Note: Rudin mentions Thm 7.32 does not hold for complex algebras
% without an extra condition.

% This means that for every f in A its complex conjugate f-bar must
% also belong to A; f-bar is defined by f-bar(x) = conjugate(f(x)).

\begin{theorem}[Stone-Weierstrass Theorem - Complex Case] % Theorem
  % 7.33 from Ch 7 PDF
  \label{thm:chap7:stone_weierstrass_complex}
  Suppose $\collection{A}$ is a self-adjoint algebra of complex
  continuous functions on a compact set K, $\collection{A}$ separates
  points on K, and $\collection{A}$ vanishes at no point of K. Then
  the uniform closure $\collection{B}$ of $\collection{A}$ consists
  of all complex continuous functions on K. (In other words,
  $\collection{A}$ is dense in $\contfunc{K}$.)
  % Proof Omitted (Reduces to the real case by considering the
  % algebra of real parts)
\end{theorem}

% --- End of Chapter 7 main content chunk (Proofs Omitted) ---
% --- End of chapters/chapter7.tex ---
