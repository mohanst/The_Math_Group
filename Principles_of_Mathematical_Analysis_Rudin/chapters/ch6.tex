% --- chapters/chapter6.tex ---
% Regenerated by Gemini (Google AI) on 2025-04-05.
% Based *only* on principles_of_mathematical_analysis_walter_rudin_ch_6.pdf
% Assumes macros from user-provided macros.tex (incl. Ch 6 additions)
% are defined.
% Proofs are omitted as requested. Using \norm{} for vector norms if applicable.

\chapter{The Riemann-Stieltjes Integral}
\label{chap:rudin6}

% The present chapter is based on a definition of the Riemann integral which
% depends very explicitly on the order structure of the real line. Accordingly,
% we begin by discussing integration of real-valued functions on intervals. Ex-
% tensions to complex- and vector-valued functions on intervals follow in later
% sections. Integration over sets other than intervals is discussed in Chaps. 10
% and 11.

\section{Definition and Existence of the Integral}
\label{sec:chap6:def_existence_rs_integral}

\begin{definition}[Partition, Riemann Integral] % Definition 6.1 from Ch 6 PDF
  \label{def:chap6:partition_riemann}
  Let $[a, b]$ be a given interval. By a partition P of $[a, b]$ we
  mean a finite set of points $x_0, x_1, \dots, x_n$, where
  \[ a = x_0 \le x_1 \le \dots \le x_{n-1} \le x_n = b. \]
  We write
  \[ \Delta x_i = x_i - x_{i-1} \quad (i=1, \dots, n). \]
  Now suppose f is a bounded real function defined on $[a, b]$.
  Corresponding to each partition P of $[a, b]$ we put
  \begin{align*}
    M_i &= \sup \set{f(x)}{x_{i-1} \le x \le x_i}, \\
    m_i &= \inf \set{f(x)}{x_{i-1} \le x \le x_i},
  \end{align*}
  and define the upper and lower Riemann sums by (Note: Rudin Ch 6
  PDF uses U(P,f) and L(P,f) here, aligning with alpha(x)=x)
  \begin{align*}
    \Usum{P}{f}{} &= \sum_{i=1}^n M_i \Delta x_i, \\
    \Lsum{P}{f}{} &= \sum_{i=1}^n m_i \Delta x_i.
  \end{align*}
  and finally
  \begin{align}
    \overline{\int_{a}^{b}} f \, dx &= \inf \Usum{P}{f}{},
    \label{eq:chap6:upper_riemann_int} \\
    \underline{\int_{a}^{b}} f \, dx &= \sup \Lsum{P}{f}{},
    \label{eq:chap6:lower_riemann_int}
  \end{align}
  where the inf and the sup are taken over all partitions P of $[a,
  b]$. The left members of \eqref{eq:chap6:upper_riemann_int} and
  \eqref{eq:chap6:lower_riemann_int} are called the upper and lower
  Riemann integrals of f over $[a, b]$, respectively. If the upper
  and lower integrals are equal, we say that f is Riemann-integrable
  on $[a, b]$, we write $f \in \mathcal{R}$ (that is, $\mathcal{R}$
  denotes the set of Riemann-integrable functions), and we denote the
  common value of \eqref{eq:chap6:upper_riemann_int} and
  \eqref{eq:chap6:lower_riemann_int} by
  \begin{equation} \label{eq:chap6:riemann_int_1}
    \int_{a}^{b} f \, dx,
  \end{equation}
  or by
  \begin{equation} \label{eq:chap6:riemann_int_2}
    \int_{a}^{b} f(x) \, dx.
  \end{equation}
  This is the Riemann integral of f over $[a, b]$.
\end{definition}

\begin{definition}[Riemann-Stieltjes Integral] % Definition 6.2 from Ch 6 PDF
  \label{def:chap6:rs_integral}
  Let $\alpha$ be a monotonically increasing function on $[a, b]$.
  Corresponding to each partition P of $[a, b]$ (as in
  \autoref{def:chap6:partition_riemann}), we write
  \[ \Delta \alpha_i = \alpha(x_i) - \alpha(x_{i-1}). \]
  It is clear that $\Delta \alpha_i \ge 0$. For any real function f
  which is bounded on $[a, b]$ we put
  \begin{align*}
    \Usum{P}{f}{\alpha} &= \sum_{i=1}^n M_i \Delta \alpha_i, \\
    \Lsum{P}{f}{\alpha} &= \sum_{i=1}^n m_i \Delta \alpha_i,
  \end{align*}
  where $M_i, m_i$ have the same meaning as in
  \autoref{def:chap6:partition_riemann}, and we define
  \begin{align}
    \overline{\int_{a}^{b}} f \, d\alpha &= \inf \Usum{P}{f}{\alpha},
    \label{eq:chap6:upper_rs_int} \\
    \underline{\int_{a}^{b}} f \, d\alpha &= \sup
    \Lsum{P}{f}{\alpha}, \label{eq:chap6:lower_rs_int}
  \end{align}
  the inf and sup again being taken over all partitions. If the left
  members of \eqref{eq:chap6:upper_rs_int} and
  \eqref{eq:chap6:lower_rs_int} are equal, we denote their common value by
  \begin{equation} \label{eq:chap6:rs_int_1}
    \int_{a}^{b} f \, d\alpha
  \end{equation}
  or sometimes by
  \begin{equation} \label{eq:chap6:rs_int_2}
    \int_{a}^{b} f(x) \, d\alpha(x).
  \end{equation}
  This is the Riemann-Stieltjes integral (or simply the Stieltjes
  integral) of f with respect to $\alpha$, over $[a, b]$. If
  \eqref{eq:chap6:rs_int_1} exists, i.e., if
  \eqref{eq:chap6:upper_rs_int} and \eqref{eq:chap6:lower_rs_int} are
  equal, we say that f is integrable with respect to $\alpha$, in the
  Riemann sense, and write $f \in \RSintegrable{\alpha}$.
  By taking $\alpha(x) = x$, the Riemann integral is seen to be a
  special case of the Riemann-Stieltjes integral.
\end{definition}

\begin{definition}[Refinement] % Definition 6.3 from Ch 6 PDF
  \label{def:chap6:refinement}
  We say that the partition $P^*$ is a refinement of P if $P^*
  \supset P$ (that is, if every point of P is a point of $P^*$).
  Given two partitions, $P_1$ and $P_2$, we say that $P^* = P_1 \cup
  P_2$ is their common refinement. $P^*$ is finer than P.
\end{definition}

\begin{theorem}[Refinement Effect on Sums] % Theorem 6.4 from Ch 6 PDF
  \label{thm:chap6:refinement_sums}
  If $P^*$ is a refinement of P (\autoref{def:chap6:refinement}), then
  \[ \Lsum{P}{f}{\alpha} \le \Lsum{P^*}{f}{\alpha} \]
  and
  \[ \Usum{P^*}{f}{\alpha} \le \Usum{P}{f}{\alpha}. \]
  % Proof Omitted
\end{theorem}

\begin{theorem}[Lower Integral <= Upper Integral] % Theorem 6.5 from Ch 6 PDF
  \label{thm:chap6:lower_le_upper_integral}
  $\underline{\int_a^b} f \, d\alpha \le \overline{\int_a^b} f \, d\alpha$.
  % Proof Omitted
\end{theorem}

\begin{theorem}[Integrability Condition] % Theorem 6.6 from Ch 6 PDF
  \label{thm:chap6:integrability_condition}
  $f \in \RSintegrable{\alpha}$ on $[a, b]$ if and only if for every
  $\epsilon > 0$ there exists a partition P such that
  \[ \Usum{P}{f}{\alpha} - \Lsum{P}{f}{\alpha} < \epsilon. \]
  % Proof Omitted
\end{theorem}

% --- End of content chunk (Restart 6.1-6.6) ---

% --- Content from Thm 6.7 to Thm 6.11 (Proofs Omitted) to append ---

\begin{theorem}[Properties Related to Integrability Condition] %
  % Theorem 6.7 from Ch 6 PDF
  \label{thm:chap6:integrability_props}
  (a) If $\Usum{P}{f}{\alpha} - \Lsum{P}{f}{\alpha} < \epsilon$ holds
  for some P and some $\epsilon$, then this inequality holds (with
  the same $\epsilon$) for every refinement of P.
  (b) If $\Usum{P}{f}{\alpha} - \Lsum{P}{f}{\alpha} < \epsilon$ holds
  for $P=\{x_0, \dots, x_n\}$ and if $s_i, t_i$ are arbitrary points
  in $[x_{i-1}, x_i]$, then
  \[ \sum_{i=1}^n \abs{f(s_i) - f(t_i)} \Delta \alpha_i < \epsilon. \]
  (c) If $f \in \RSintegrable{\alpha}$ and the hypotheses of (b) hold, then
  \[ \abs{\sum_{i=1}^n f(t_i) \Delta \alpha_i - \int_a^b f \,
  d\alpha} < \epsilon. \]
  % Proof Omitted (Uses Thm 6.4)
\end{theorem}

\begin{theorem}[Continuous Function implies Integrable] % Theorem 6.8
  % from Ch 6 PDF
  \label{thm:chap6:continuous_implies_integrable}
  If f is continuous on $[a, b]$ then $f \in \RSintegrable{\alpha}$ on $[a, b]$.
  % Proof Omitted (Uses uniform continuity Thm 4.19 and Thm 6.6)
\end{theorem}

\begin{theorem}[Monotonic Function implies Integrable (if alpha
  continuous)] % Theorem 6.9 from Ch 6 PDF
  \label{thm:chap6:monotonic_implies_integrable}
  If f is monotonic on $[a, b]$, and if $\alpha$ is continuous on
  $[a, b]$, then $f \in \RSintegrable{\alpha}$. (We still assume, of
  course, that $\alpha$ is monotonic.)
  % Proof Omitted (Uses continuity of alpha Thm 4.23 and Thm 6.6)
\end{theorem}

\begin{theorem}[Bounded Function with Finite Discontinuities] %
  % Theorem 6.10 from Ch 6 PDF
  \label{thm:chap6:finite_discont_integrable}
  Suppose f is bounded on $[a, b]$, f has only finitely many points
  of discontinuity on $[a, b]$, and $\alpha$ is continuous at every
  point at which f is discontinuous. Then $f \in \RSintegrable{\alpha}$.
  % Proof Omitted
\end{theorem}

\begin{theorem}[Continuous Function of Integrable Function] % Theorem
  % 6.11 from Ch 6 PDF
  \label{thm:chap6:continuous_fn_of_integrable}
  Suppose $f \in \RSintegrable{\alpha}$ on $[a, b]$, $m \le f \le M$,
  $\phi$ is continuous on $[m, M]$, and $h(x) = \phi(f(x))$ on $[a,
  b]$. Then $h \in \RSintegrable{\alpha}$ on $[a, b]$.
  % Proof Omitted
\end{theorem}

% --- End of content chunk ---

% --- Content from Thm 6.12 to Thm 6.15 (Proofs Omitted) to append ---

\section{Properties of the Integral}
\label{sec:chap6:integral_properties}

\begin{theorem}[Properties of the Integral] % Theorem 6.12 from Ch 6 PDF
  \label{thm:chap6:integral_properties}
  (a) If $f_1 \in \RSintegrable{\alpha}$ and $f_2 \in
  \RSintegrable{\alpha}$ on $[a, b]$, then $f_1+f_2 \in
  \RSintegrable{\alpha}$, $c f \in \RSintegrable{\alpha}$ for every
  constant c, and
  \begin{align*}
    \int_a^b (f_1 + f_2) \, d\alpha &= \int_a^b f_1 \, d\alpha +
    \int_a^b f_2 \, d\alpha, \\
    \int_a^b c f \, d\alpha &= c \int_a^b f \, d\alpha.
  \end{align*}
  (b) If $f_1(x) \le f_2(x)$ on $[a, b]$, then
  \[ \int_a^b f_1 \, d\alpha \le \int_a^b f_2 \, d\alpha. \]
  (c) If $f \in \RSintegrable{\alpha}$ on $[a, b]$ and if $a < c <
  b$, then $f \in \RSintegrable{\alpha}$ on $[a, c]$ and on $[c, b]$, and
  \[ \int_a^c f \, d\alpha + \int_c^b f \, d\alpha = \int_a^b f \, d\alpha. \]
  (d) If $f \in \RSintegrable{\alpha}$ on $[a, b]$ and if $\abs{f(x)}
  \le M$ on $[a, b]$, then
  \[ \abs{\int_a^b f \, d\alpha} \le M[\alpha(b) - \alpha(a)]. \]
  (e) If $f \in \RSintegrable{\alpha_1}$ and $f \in
  \RSintegrable{\alpha_2}$, then $f \in \RSintegrable{\alpha_1 + \alpha_2}$ and
  \[ \int_a^b f \, d(\alpha_1 + \alpha_2) = \int_a^b f \, d\alpha_1 +
  \int_a^b f \, d\alpha_2; \]
  if $f \in \RSintegrable{\alpha}$ and c is a positive constant, then
  $f \in \RSintegrable{c\alpha}$ and
  \[ \int_a^b f \, d(c\alpha) = c \int_a^b f \, d\alpha. \]
  % Proof Omitted
\end{theorem}

\begin{theorem}[Integrability of Product and Absolute Value] %
  % Theorem 6.13 from Ch 6 PDF
  \label{thm:chap6:product_abs_integrability}
  If $f \in \RSintegrable{\alpha}$ and $g \in \RSintegrable{\alpha}$
  on $[a, b]$, then
  (a) $fg \in \RSintegrable{\alpha}$;
  (b) $\abs{f} \in \RSintegrable{\alpha}$ and $\abs{\int_a^b f \,
  d\alpha} \le \int_a^b \abs{f} \, d\alpha$.
  % Proof Omitted (Uses Thm 6.11 for part (b))
\end{theorem}

\begin{definition}[Unit Step Function] % Definition 6.14 from Ch 6 PDF
  \label{def:chap6:unit_step_function}
  The unit step function I is defined by
  \[ I(x) =
    \begin{cases} 0 & (x \le 0), \\ 1 & (x > 0).
  \end{cases} \]
\end{definition}

\begin{theorem}[Integral with respect to Unit Step Function] %
  % Theorem 6.15 from Ch 6 PDF
  \label{thm:chap6:integral_unit_step}
  If $a < s < b$, f is bounded on $[a, b]$, f is continuous at s, and
  $\alpha(x) = I(x-s)$, then
  \[ \int_a^b f \, d\alpha = f(s). \]
  % Proof Omitted
\end{theorem}

% --- End of content chunk ---

% --- Content from Thm 6.16 to Thm 6.21 (Proofs Omitted) to append ---

\begin{theorem}[Integral with respect to Step Function Sum] % Theorem
  % 6.16 from Ch 6 PDF
  \label{thm:chap6:integral_step_sum}
  Suppose $c_n \ge 0$ for $n=1, 2, 3, \dots$, $\sum c_n$ converges,
  $\sequence{s}$ is a sequence of distinct points in $(a, b)$, and
  \begin{equation} \label{eq:chap6:alpha_step_sum}
    \alpha(x) = \sum_{n=1}^\infty c_n I(x - s_n).
  \end{equation}
  Let f be continuous on $[a, b]$. Then
  \begin{equation} \label{eq:chap6:integral_step_sum_result}
    \int_a^b f \, d\alpha = \sum_{n=1}^\infty c_n f(s_n).
  \end{equation}
  % Proof Omitted (Uses Thm 6.12 and Thm 6.15)
\end{theorem}

\begin{theorem}[Relation to Riemann Integral] % Theorem 6.17 from Ch 6 PDF
  \label{thm:chap6:rs_vs_riemann}
  Assume $\alpha$ increases monotonically and $\alpha' \in
  \mathcal{R}$ on $[a, b]$. Let f be a bounded real function on $[a, b]$.
  Then $f \in \RSintegrable{\alpha}$ if and only if $f\alpha' \in
  \mathcal{R}$. In that case
  \[ \int_a^b f \, d\alpha = \int_a^b f(x)\alpha'(x) \, dx. \]
  % Proof Omitted (Uses Mean Value Theorem, Thm 6.6, Thm 6.7(b))
\end{theorem}

\begin{remark}[Generality of Stieltjes Integral] % Remark 6.18 from Ch 6 PDF
  \label{rem:chap6:stieltjes_generality}
  The two preceding theorems illustrate the generality and
  flexibility which are inherent in the Stieltjes process of
  integration. If $\alpha$ is a pure step function [this is the name
    often given to functions of the form
  \eqref{eq:chap6:alpha_step_sum}], the integral reduces to a finite
  or infinite series. If $\alpha$ has an integrable derivative, the
  integral reduces to an ordinary Riemann integral. This makes it
  possible in many cases to study series and integrals
  simultaneously, rather than separately.
  To illustrate this point, consider a physical example. The moment
  of inertia of a straight wire of unit length, about an axis through
  an endpoint, at right angles to the wire, is
  \begin{equation} \label{eq:chap6:moment_inertia}
    \int_0^1 x^2 \, dm
  \end{equation}
  where $m(x)$ is the mass contained in the interval $[0, x]$. If the
  wire is regarded as having a continuous density $\rho$, that is, if
  $m'(x) = \rho(x)$, then \eqref{eq:chap6:moment_inertia} turns into
  \begin{equation} \label{eq:chap6:moment_inertia_cont}
    \int_0^1 x^2 \rho(x) \, dx.
  \end{equation}
  On the other hand, if the wire is composed of masses $m_i$
  concentrated at points $x_i$, \eqref{eq:chap6:moment_inertia} becomes
  \begin{equation} \label{eq:chap6:moment_inertia_discrete}
    \sum_i x_i^2 m_i.
  \end{equation}
  Thus \eqref{eq:chap6:moment_inertia} contains
  \eqref{eq:chap6:moment_inertia_cont} and
  \eqref{eq:chap6:moment_inertia_discrete} as special cases, but it
  contains much more; for instance, the case in which m is continuous
  but not everywhere differentiable.
\end{remark}

\begin{theorem}[Change of Variable] % Theorem 6.19 from Ch 6 PDF
  \label{thm:chap6:change_variable}
  Suppose $\psi$ is a strictly increasing continuous function that
  maps an interval $[A, B]$ onto $[a, b]$. Suppose $\alpha$ is
  monotonically increasing on $[a, b]$ and $f \in
  \RSintegrable{\alpha}$ on $[a, b]$. Define $\beta$ and g on $[A, B]$ by
  \begin{align}
    \beta(y) &= \alpha(\psi(y)), \label{eq:chap6:beta_change_var} \\
    g(y) &= f(\psi(y)). \label{eq:chap6:g_change_var}
  \end{align}
  Then $g \in \RSintegrable{\beta}$ and
  \begin{equation} \label{eq:chap6:change_var_result}
    \int_A^B g \, d\beta = \int_a^b f \, d\alpha.
  \end{equation}
  % Proof Omitted (Relates partitions P of [a,b] to partitions Q of
  % [A,B] via phi)
  (Special case: If $\alpha(x)=x$ and $\psi' \in \mathcal{R}$ on $[A,
    B]$, applying \autoref{thm:chap6:rs_vs_riemann} gives the standard
    change of variables formula: $\int_a^b f(x) \, dx = \int_A^B
  f(\psi(y))\psi'(y) \, dy$.)
\end{theorem}

\section{Integration and Differentiation}
\label{sec:chap6:integration_differentiation}

% We still confine ourselves to real functions in this section. We
% shall show that
% integration and differentiation are, in a certain sense, inverse operations.

\begin{theorem}[Fundamental Theorem of Calculus, Part 1] % Theorem
  % 6.20 from Ch 6 PDF
  \label{thm:chap6:ftc1}
  Let $f \in \mathcal{R}$ on $[a, b]$. For $a \le x \le b$, put
  \[ F(x) = \int_a^x f(t) \, dt. \]
  Then F is continuous on $[a, b]$; furthermore, if f is continuous
  at a point $x_0$ of $[a, b]$, then F is differentiable at $x_0$,
  and $F'(x_0) = f(x_0)$.
  % Proof Omitted
\end{theorem}

\begin{theorem}[Fundamental Theorem of Calculus, Part 2] % Theorem
  % 6.21 from Ch 6 PDF
  \label{thm:chap6:ftc2}
  If $f \in \mathcal{R}$ on $[a, b]$ and if there is a differentiable
  function F on $[a, b]$ such that $F' = f$, then
  \[ \int_a^b f(x) \, dx = F(b) - F(a). \]
  % Proof Omitted (Uses Mean Value Theorem and Thm 6.7(c))
\end{theorem}

% --- End of content chunk ---

% --- Content from Thm 6.22 to Thm 6.27 (Proofs Omitted) to append ---

\begin{theorem}[Integration by Parts] % Theorem 6.22 from Ch 6 PDF
  \label{thm:chap6:integration_by_parts}
  Suppose F and G are differentiable functions on $[a, b]$, $F' = f
  \in \mathcal{R}$, and $G' = g \in \mathcal{R}$. Then
  \[ \int_a^b F(x) g(x) \, dx = F(b)G(b) - F(a)G(a) - \int_a^b f(x)
  G(x) \, dx. \]
  % Proof Omitted (Applies Thm 6.21 to H=FG)
\end{theorem}

\section{Integration of Vector-valued Functions}
\label{sec:chap6:integration_vector_valued}

\begin{definition}[Integral of Vector-valued Function] % Definition
  % 6.23 from Ch 6 PDF
  \label{def:chap6:integral_vector_valued}
  Let $f_1, \dots, f_k$ be real functions on $[a, b]$, and let
  $\vect{f} = (f_1, \dots, f_k)$ be the corresponding mapping of $[a,
  b]$ into $\R^k$. If $\alpha$ increases monotonically on $[a, b]$,
  to say that $\vect{f} \in \RSintegrable{\alpha}$ means that $f_j
  \in \RSintegrable{\alpha}$ for $j=1, \dots, k$. If this is the case, we define
  \[ \int_a^b \vect{f} \, d\alpha = \left( \int_a^b f_1 \, d\alpha,
  \dots, \int_a^b f_k \, d\alpha \right). \]
  In other words, $\int_a^b \vect{f} \, d\alpha$ is the point in
  $\R^k$ whose jth coordinate is $\int_a^b f_j \, d\alpha$.
\end{definition}

% Note: Rudin mentions properties like linearity (Thm 6.12(a,c,e))
% and connection to Riemann (Thm 6.17)
% and FTC Part 1 (Thm 6.20) also hold for vector integrals.

\begin{theorem}[FTC for Vector-valued Functions] % Theorem 6.24 from Ch 6 PDF
  \label{thm:chap6:ftc_vector}
  If $\vect{f}$ and $\vect{F}$ map $[a, b]$ into $\R^k$, if $\vect{f}
  \in \mathcal{R}$ on $[a, b]$, and if $\vect{F}' = \vect{f}$, then
  \[ \int_a^b \vect{f}(t) \, dt = \vect{F}(b) - \vect{F}(a). \]
  % Proof Omitted (Applies Thm 6.21 to each component)
\end{theorem}

\begin{theorem}[Inequality for Vector Integral] % Theorem 6.25 from Ch 6 PDF
  \label{thm:chap6:vector_integral_inequality}
  If $\vect{f}$ maps $[a, b]$ into $\R^k$ and if $\vect{f} \in
  \RSintegrable{\alpha}$ for some monotonically increasing function
  $\alpha$ on $[a, b]$, then $\abs{\vect{f}} \in
  \RSintegrable{\alpha}$ (using standard magnitude for
  $\abs{\vect{f}}$) and (using $\norm{}$ for vector norm)
  \[ \norm{\int_a^b \vect{f} \, d\alpha} \le \int_a^b \abs{\vect{f}}
  \, d\alpha. \]
  % Proof Omitted (Uses properties of components, Schwarz inequality,
  % Thm 6.11, Thm 6.12(b))
\end{theorem}

\section{Rectifiable Curves}
\label{sec:chap6:rectifiable_curves}

% We conclude this chapter with a topic of geometric interest which provides an
% application of some of the preceding theory. The case k=2 (i.e., the case of
% plane curves) is of considerable importance in the study of analytic functions
% of a complex variable.

\begin{definition}[Curve, Arc, Length] % Definition 6.26 from Ch 6 PDF
  \label{def:chap6:curve_arc_length}
  A continuous mapping $\gamma$ of an interval $[a, b]$ into $\R^k$
  is called a curve in $\R^k$. To emphasize the parameter interval
  $[a, b]$, we may also say that $\gamma$ is a curve on $[a, b]$.
  If $\gamma$ is one-to-one, $\gamma$ is called an arc.
  If $\gamma(a) = \gamma(b)$, $\gamma$ is said to be a closed curve.
  It should be noted that we define a curve to be a mapping, not a
  point set. Of course, with each curve $\gamma$ in $\R^k$ there is
  associated a subset of $\R^k$, namely the range of $\gamma$, but
  different curves may have the same range.
  We associate to each partition $P = \{x_0, \dots, x_n\}$ of $[a,
  b]$ and to each curve $\gamma$ on $[a, b]$ the number
  \[ \arclength{P, \gamma} = \sum_{i=1}^n \norm{\gamma(x_i) -
  \gamma(x_{i-1})}. \]
  (Using `$\norm{}$` for the norm in $\R^k$).
  The ith term in this sum is the distance (in $\R^k$) between the
  points $\gamma(x_{i-1})$ and $\gamma(x_i)$. Hence $\arclength{P,
  \gamma}$ is the length of a polygonal path with vertices at
  $\gamma(x_0), \gamma(x_1), \dots, \gamma(x_n)$, in this order. As
  our partition becomes finer and finer, this polygon approaches the
  range of $\gamma$ more and more closely. This makes it seem
  reasonable to define the length of $\gamma$ as
  \[ \arclength{\gamma} = \sup \arclength{P, \gamma}, \]
  where the supremum is taken over all partitions P of $[a, b]$.
  If $\arclength{\gamma} < \infty$ we say that $\gamma$ is rectifiable.
\end{definition}

\begin{theorem}[Arc Length Formula] % Theorem 6.27 from Ch 6 PDF
  \label{thm:chap6:arc_length_formula}
  If $\gamma'$ is continuous on $[a, b]$, then $\gamma$ is rectifiable, and
  \[ \arclength{\gamma} = \int_a^b \norm{\gamma'(t)} \, dt. \]
  (Using `$\norm{}$` for the norm in $\R^k$).
  % Proof Omitted (Uses FTC for vector functions Thm 6.24, integral
  % inequality Thm 6.25, and uniform continuity of gamma')
\end{theorem}

% --- End of Chapter 6 content chunk (Proofs Omitted) ---
% --- End of chapters/chapter6.tex ---
