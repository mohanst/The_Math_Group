% --- chapters/chapter2.tex ---
% Generated by Gemini (Google AI) on 2025-04-04.
% Contains extracted items from W. Rudin, PMA, Chapter 2.

\chapter{Basic Topology}
\label{chap:rudin2}

\section{Finite, Countable, and Uncountable Sets}
\label{sec:chap2:countability}

% Note: Apply macros from macros.tex where applicable.
% Content filled for items 2.1 to 2.8.

\begin{definition}[Function/Mapping] % Definition 2.1
  \label{def:chap2:function}
  Consider two sets A and B, whose elements may be any objects
  whatsoever, and suppose that with each element x of A there is
  associated, in some manner, an element of B, which we denote by
  $f(x)$. Then f is said to be a function from A to B (or a mapping
  of A into B). The set A is called the domain of f (we also say f is
  defined on A), and the elements $f(x)$ are called the values of f.
  The set of all values of f is called the range of f.
\end{definition}

\begin{definition}[Image, Range, Onto, Inverse Image, 1-1] % Definition 2.2
  \label{def:chap2:image_inverse_onto_1-1}
  Let A and B be two sets and let f be a mapping of A into B.
  If $E \subset A$, $f(E)$ is defined to be the set of all elements
  $f(x)$, for $x \in E$. We call $f(E)$ the image of E under f. In
  this notation, $f(A)$ is the range of f. It is clear that $f(A)
  \subset B$. If $f(A) = B$, we say that f maps A onto B. (Note that,
  according to this usage, onto is more specific than into.)

  If $E \subset B$, $f^{-1}(E)$ denotes the set of all $x \in A$ such
  that $f(x) \in E$. We call $f^{-1}(E)$ the inverse image of E under
  f. If $y \in B$, $f^{-1}(y)$ is the set of all $x \in A$ such that
  $f(x)=y$. If, for each $y \in B$, $f^{-1}(y)$ consists of at most
  one element of A, then f is said to be a 1-1 (one-to-one) mapping
  of A into B. This may also be expressed as follows: f is a 1-1
  mapping of A into B provided that $f(x_1) \ne f(x_2)$ whenever $x_1
  \ne x_2$, $x_1 \in A$, $x_2 \in A$.
  (The notation $x_1 \ne x_2$ means that $x_1$ and $x_2$ are distinct
  elements; otherwise we write $x_1 = x_2$.)
\end{definition}

\begin{definition}[1-1 Correspondence, Equivalence Relation] % Definition 2.3
  \label{def:chap2:equivalence}
  If there exists a 1-1 mapping of A onto B, we say that A and B can
  be put in 1-1 correspondence, or that A and B have the same
  cardinal number, or, briefly, that A and B are equivalent, and we
  write $A \sim B$. This relation clearly has the following properties:
  It is reflexive: $A \sim A$.
  It is symmetric: If $A \sim B$, then $B \sim A$.
  It is transitive: If $A \sim B$ and $B \sim C$, then $A \sim C$.
  Any relation with these three properties is called an equivalence relation.
\end{definition}

\begin{definition}[Finite, Infinite, Countable, Uncountable, At most
  countable] % Definition 2.4
  \label{def:chap2:countability_types}
  For any positive integer n, let $J_n$ be the set whose elements are
  the integers $1, 2, \dots, n$; let J be the set consisting of all
  positive integers. For any set A, we say:
  (a) A is finite if $A \sim J_n$ for some n (the empty set is also
  considered to be finite).
  (b) A is infinite if A is not finite.
  (c) A is countable if $A \sim J$.
  (d) A is uncountable if A is neither finite nor countable.
  (e) A is at most countable if A is finite or countable.
  Countable sets are sometimes called enumerable, or denumerable.
\end{definition}

\begin{example}[Integers are countable] % Example 2.5
  \label{ex:chap2:integers_countable}
  Let A be the set of all integers. Then A is countable. For,
  consider the following arrangement of the sets A and J:
  A: 0, 1, -1, 2, -2, 3, -3, ...
  J: 1, 2,  3, 4,  5, 6,  7, ...
  We can, in this example, even give an explicit formula for a
  function f from J to A which sets up a 1-1 correspondence:
  \[ f(n) =
    \begin{cases} n/2 & \text{(n even)}, \\ -(n-1)/2 & \text{(n odd)}.
  \end{cases} \]
\end{example}

\begin{remark}[Infinite sets equivalent to proper subsets] % Remark 2.6
  \label{rem:chap2:infinite_subsets}
  A finite set cannot be equivalent to one of its proper subsets.
  That this is, however, possible for infinite sets, is shown by
  Example 2.5, in which J is a proper subset of A.
  In fact, we could replace Definition 2.4(b) by the statement: A is
  infinite if A is equivalent to one of its proper subsets.
\end{remark}

\begin{definition}[Sequence] % Definition 2.7
  \label{def:chap2:sequence}
  By a sequence, we mean a function f defined on the set J of all
  positive integers. If $f(n) = x_n$, for $n \in J$, it is customary
  to denote the sequence f by the symbol $\{x_n\}$, or sometimes by
  $x_1, x_2, x_3, \dots$. The values of f, that is, the elements
  $x_n$, are called the terms of the sequence. If A is a set and if
  $x_n \in A$ for all $n \in J$, then $\{x_n\}$ is said to be a
  sequence in A, or a sequence of elements of A.
  Note that the terms $x_1, x_2, x_3, \dots$ of a sequence need not be distinct.
  Since every countable set is the range of a 1-1 function defined on
  J, we may regard every countable set as the range of a sequence of
  distinct terms. Speaking more loosely, we may say that the elements
  of any countable set can be "arranged in a sequence."
  Sometimes it is convenient to replace J in this definition by the
  set of all nonnegative integers, i.e., to start with 0 rather than with 1.
\end{definition}

\begin{theorem}[Infinite subset of countable set is countable] % Theorem 2.8
  \label{thm:chap2:infinite_subset_countable}
  Every infinite subset of a countable set A is countable.
  \begin{proof}
    Suppose $E \subset A$, and E is infinite. Arrange the elements x
    of A in a sequence $\{x_n\}$ of distinct elements. Construct a
    sequence $\{n_k\}$ as follows:
    Let $n_1$ be the smallest positive integer such that $x_{n_1} \in
    E$. Having chosen $n_1, \dots, n_{k-1}$ ($k=2, 3, 4, \dots$), let
    $n_k$ be the smallest integer greater than $n_{k-1}$ such that
    $x_{n_k} \in E$.
    Putting $f(k) = x_{n_k}$ ($k=1, 2, 3, \dots$) we obtain a 1-1
    correspondence between E and J.
    The theorem shows that, roughly speaking, countable sets
    represent the "smallest" infinity: No uncountable set can be a
    subset of a countable set.
  \end{proof}
\end{theorem}

% Add remaining definitions, theorems, examples etc. from Ch 2 here later...

% --- Regenerated content from Def 2.9 to Thm 2.14 using suggested macros ---
% --- Replace the corresponding section in chapters/chapter2.tex ---

\begin{definition}[Union, Intersection of Collections] % Definition 2.9
  \label{def:chap2:set_collections}
  Let A and $\Omega$ be sets, and suppose that with each element
  $\alpha$ of A there is associated a subset of $\Omega$ which we
  denote by $E_\alpha$.
  The set whose elements are the sets $E_\alpha$ will be denoted by
  $\{E_\alpha\}$. Instead of speaking of sets of sets, we shall
  sometimes speak of a collection of sets, or a family of sets.

  The union of the sets $E_\alpha$ is defined to be the set S such
  that $x \in S$ if and only if $x \in E_\alpha$ for at least one
  $\alpha \in A$. We use the notation
  \begin{equation} \label{eq:chap2:union_def}
    S = \bigunion{\alpha \in A} E_\alpha.
  \end{equation}
  If A consists of the integers $1, 2, \dots, n$, one usually writes
  \begin{equation} \label{eq:chap2:finite_union}
    S = \bigcup_{m=1}^{n} E_m \quad \text{or} \quad S = E_1 \cup E_2
    \cup \dots \cup E_n.
  \end{equation}
  If A is the set of all positive integers, the usual notation is
  \begin{equation} \label{eq:chap2:countable_union}
    S = \bigcup_{m=1}^{\infty} E_m.
  \end{equation}
  The symbol $\infty$ in (4) merely indicates that the union of a
  countable collection of sets is taken, and should not be confused
  with the symbols $+\infty$, $-\infty$, introduced in Definition 1.23.

  The intersection of the sets $E_\alpha$ is defined to be the set P
  such that $x \in P$ if and only if $x \in E_\alpha$ for every
  $\alpha \in A$. We use the notation
  \begin{equation} \label{eq:chap2:intersect_def}
    P = \bigintersect{\alpha \in A} E_\alpha.
  \end{equation}
  or
  \begin{equation} \label{eq:chap2:finite_intersect}
    P = \bigcap_{m=1}^{n} E_m = E_1 \cap E_2 \cap \dots \cap E_n,
  \end{equation}
  or
  \begin{equation} \label{eq:chap2:countable_intersect}
    P = \bigcap_{m=1}^{\infty} E_m,
  \end{equation}
  as for unions. If $A \cap B$ is not empty, we say that A and B
  intersect; otherwise they are disjoint.
\end{definition}

\begin{example}[Union/Intersection Examples] % Example 2.10
  \label{ex:chap2:union_intersect_examples}
  (a) Suppose $E_1$ consists of 1, 2, 3 and $E_2$ consists of 2, 3,
  4. Then $E_1 \cup E_2$ consists of 1, 2, 3, 4, whereas $E_1 \cap
  E_2$ consists of 2, 3.

  (b) Let A be the set of real numbers x such that $0 < x \le 1$. For
  every $x \in A$, let $E_x$ be the set of real numbers y such that
  $0 < y < x$. Then
  (i) $E_x \subset E_z$ if and only if $0 < x \le z \le 1$;
  (ii) $\bigunion{x \in A} E_x = E_1$;
  (iii) $\bigintersect{x \in A} E_x$ is empty;
  (i) and (ii) are clear. To prove (iii), we note that for every $y >
  0$, $y \notin E_x$ if $x < y$. Hence $y \notin \bigintersect{x \in A} E_x$.
\end{example}

\begin{remark}[Properties of Union/Intersection] % Remark 2.11
  \label{rem:chap2:union_intersect_props}
  Many properties of unions and intersections are quite similar to
  those of sums and products; in fact, the words sum and product were
  sometimes used in this connection, and the symbols $\sum$ and
  $\prod$ were written in place of $\cup$ and $\cap$.
  The commutative and associative laws are trivial:
  \begin{gather}
    A \cup B = B \cup A; \quad A \cap B = B \cap A. \label{eq:chap2:set_comm} \\
    (A \cup B) \cup C = A \cup (B \cup C); \quad (A \cap B) \cap C =
    A \cap (B \cap C). \label{eq:chap2:set_assoc}
  \end{gather}
  Thus the omission of parentheses in (3) and (6) is justified.
  The distributive law also holds:
  \begin{equation} \label{eq:chap2:set_distrib}
    A \cap (B \cup C) = (A \cap B) \cup (A \cap C).
  \end{equation}
  To prove this, let the left and right members of (10) be denoted by
  E and F, respectively. Suppose $x \in E$. Then $x \in A$ and $x \in
  B \cup C$, that is, $x \in B$ or $x \in C$ (possibly both). Hence
  $x \in A \cap B$ or $x \in A \cap C$, so that $x \in F$. Thus $E \subset F$.
  Next, suppose $x \in F$. Then $x \in A \cap B$ or $x \in A \cap C$.
  That is, $x \in A$, and ($x \in B$ or $x \in C$). Hence $x \in A$
  and $x \in B \cup C$. Hence $x \in A \cap (B \cup C)$, so that $F \subset E$.
  It follows that $E=F$.
  We list a few more relations which are easily verified:
  \begin{gather}
    A \subset A \cup B. \label{eq:chap2:set_subset_union} \\
    A \cap B \subset A. \label{eq:chap2:set_intersect_subset}
  \end{gather}
  If $\emptyset$ denotes the empty set, then
  \begin{equation} \label{eq:chap2:set_empty}
    A \cup \emptyset = A, \quad A \cap \emptyset = \emptyset.
  \end{equation}
  If $A \subset B$, then
  \begin{equation} \label{eq:chap2:set_subset_ops}
    A \cup B = B, \quad A \cap B = A.
  \end{equation}
\end{remark}

\begin{theorem}[Countable Union of Countable Sets] % Theorem 2.12
  \label{thm:chap2:countable_union_countable}
  Let $\{E_n\}$, $n=1, 2, 3, \dots$ be a sequence of countable sets, and put
  \begin{equation}
    S = \bigcup_{n=1}^{\infty} E_n.
  \end{equation}
  Then S is countable.
  \begin{proof}
    Let every set $E_n$ be arranged in a sequence $\{x_{nk}\}$, $k=1,
    2, 3, \dots$, and consider the infinite array
    \[
      \begin{matrix}
        x_{11} & x_{12} & x_{13} & x_{14} & \dots \\
        x_{21} & x_{22} & x_{23} & x_{24} & \dots \\
        x_{31} & x_{32} & x_{33} & x_{34} & \dots \\
        x_{41} & x_{42} & x_{43} & x_{44} & \dots \\
        \vdots & \vdots & \vdots & \vdots & \ddots
      \end{matrix}
    \]
    in which the elements of $E_n$ form the nth row. The array
    contains all elements of S. As indicated by the arrows [diagonal
    traversal], these elements can be arranged in a sequence
    \begin{equation}
      x_{11}; x_{21}, x_{12}; x_{31}, x_{22}, x_{13}; x_{41}, x_{32},
      x_{23}, x_{14}; \dots
    \end{equation}
    If any two of the sets $E_n$ have elements in common, these will
    appear more than once in (17). Hence there is a subset T of the
    set of all positive integers such that $S \sim T$, which shows
    that S is at most countable (Theorem 2.8). Since $E_1 \subset S$,
    and $E_1$ is infinite [assuming at least one $E_n$ is infinite,
    otherwise S is finite or empty], S is infinite, and thus
    countable. [Refinement: If all $E_n$ are finite, the union is at
      most countable. If at least one $E_n$ is countably infinite, the
    union S is countably infinite.]
  \end{proof}
\end{theorem}

\begin{corollary} % Corollary to 2.12
  \label{cor:chap2:at_most_countable_union}
  Suppose A is at most countable, and, for every $\alpha \in A$,
  $B_\alpha$ is at most countable. Put
  \[ T = \bigunion{\alpha \in A} B_\alpha. \]
  Then T is at most countable.
  (For T is equivalent to a subset of (15).)
\end{corollary}

\begin{theorem}[Set of n-tuples] % Theorem 2.13
  \label{thm:chap2:n_tuples_countable}
  Let A be a countable set, and let $B_n$ be the set of all n-tuples
  $(a_1, \dots, a_n)$, where $a_k \in A$ ($k=1, \dots, n$), and the
  elements $a_1, \dots, a_n$ need not be distinct. Then $B_n$ is countable.
  \begin{proof}
    That $B_1$ is countable is evident, since $B_1 = A$. Suppose
    $B_{n-1}$ is countable ($n=2, 3, 4, \dots$). The elements of
    $B_n$ are of the form
    \[ (b, a) \quad (b \in B_{n-1}, a \in A). \]
    For every fixed b, the set of pairs (b, a) is equivalent to A,
    and hence countable. Thus $B_n$ is the union of a countable set
    of countable sets [i.e., $\bigunion{b \in B_{n-1}} \set{(b,a)}{a
    \in A}$]. By Theorem 2.12, $B_n$ is countable. The theorem
    follows by induction.
  \end{proof}
\end{theorem}

\begin{corollary}[Rationals are Countable] % Corollary to 2.13
  \label{cor:chap2:rationals_countable}
  The set of all rational numbers is countable.
  \begin{proof}
    We apply Theorem 2.13, with $n=2$, noting that every rational r
    is of the form $b/a$, where a and b are integers. The set of
    integers is countable (Example 2.5). The set of pairs (a, b), and
    therefore the set of fractions $b/a$, is countable.
    (In fact, even the set of all algebraic numbers is countable (see
    Exercise 2).)
  \end{proof}
\end{corollary}

\begin{theorem}[Set of 0/1 Sequences is Uncountable] % Theorem 2.14
  \label{thm:chap2:binary_sequences_uncountable}
  Let A be the set of all sequences whose elements are the digits 0
  and 1. This set A is uncountable.
  (The elements of A are sequences like $1, 0, 0, 1, 0, 1, 1, 1, \dots$)
  \begin{proof}
    Let E be a countable subset of A, and let E consist of the
    sequences $s_1, s_2, s_3, \dots$. We construct a sequence s as
    follows. If the nth digit in $s_n$ is 1, we let the nth digit of
    s be 0, and vice versa. Then the sequence s differs from every
    member of E in at least one place; hence $s \notin E$. But
    clearly $s \in A$, so that E is a proper subset of A.
    We have shown that every countable subset of A is a proper subset
    of A. It follows that A is uncountable (for otherwise A would be
    a proper subset of A, which is absurd).
    (The idea of the above proof was first used by Cantor, and is
      called Cantor's diagonal process; for, if the sequences $s_1,
      s_2, s_3, \dots$ are placed in an array like (16), it is the
      elements on the diagonal which are involved in the construction
      of the new sequence.
      Readers who are familiar with the binary representation of the
      real numbers (base 2 instead of 10) will notice that Theorem 2.14
      implies that the set of all real numbers is uncountable. We shall
    give a second proof of this fact in Theorem 2.43.)
  \end{proof}
\end{theorem}

% --- End of regenerated chunk (2.9 to 2.14) using macros ---

% --- Content from Sec "Metric Spaces" up to Ex 2.21 to append ---

\section{Metric Spaces}
\label{sec:chap2:metric_spaces}

\begin{definition}[Metric Space] % Definition 2.15
  \label{def:chap2:metric_space}
  A set X, whose elements we shall call points, is said to be a
  metric space if with any two points p and q of X there is
  associated a real number $d(p, q)$, called the distance from p to q, such that
  (a) $d(p, q) > 0$ if $p \ne q$; $d(p, p) = 0$;
  (b) $d(p, q) = d(q, p)$;
  (c) $d(p, q) \le d(p, r) + d(r, q)$, for any $r \in X$.
  Any function with these three properties is called a distance
  function, or a metric.
\end{definition}

\begin{example}[Examples of Metric Spaces] % Example 2.16
  \label{ex:chap2:metric_space_examples}
  The most important examples of metric spaces, from our standpoint,
  are the euclidean spaces $\R^k$, especially $\R^1$ (the real line)
  and $\R^2$ (the complex plane); the distance in $\R^k$ is defined by
  \begin{equation} \label{eq:chap2:euclidean_distance}
    d(\vect{x}, \vect{y}) = \abs{\vect{x} - \vect{y}} \quad
    (\vect{x}, \vect{y} \in \R^k).
  \end{equation}
  By \autoref{thm:chap1:norm_inner_product_properties}, the
  conditions of \autoref{def:chap2:metric_space} are satisfied by
  \eqref{eq:chap2:euclidean_distance}.
  It is important to observe that every subset Y of a metric space X
  is a metric space in its own right, with the same distance
  function. For it is clear that if conditions (a) to (c) of
  \autoref{def:chap2:metric_space} hold for p, q, $r \in X$, they
  also hold if we restrict p, q, r to lie in Y.
  Thus every subset of a euclidean space is a metric space. Other
  examples are the spaces $\mathcal{C}(K)$ and $\mathcal{L}^2(\mu)$,
  which are discussed in Chaps. 7 and 11, respectively.
\end{example}

\begin{definition}[Intervals, Cells, Balls, Convex Sets] % Definition 2.17
  \label{def:chap2:intervals_balls_convex}
  By the segment (a, b) we mean the set of all real numbers x such
  that $a < x < b$.
  By the interval [a, b] we mean the set of all real numbers x such
  that $a \le x \le b$.
  Occasionally we shall also encounter "half-open intervals" [a, b)
  and (a, b]; the first consists of all x such that $a \le x < b$,
  the second of all x such that $a < x \le b$.
  If $a_i < b_i$ for $i=1, \dots, k$, the set of all points $\vect{x}
  = (x_1, \dots, x_k)$ in $\R^k$ whose coordinates satisfy the
  inequalities $a_i \le x_i \le b_i$ ($1 \le i \le k$) is called a
  k-cell. Thus a 1-cell is an interval, a 2-cell is a rectangle, etc.

  If $\vect{x} \in \R^k$ and $r>0$, the open (or closed) ball B with
  center at $\vect{x}$ and radius r is defined to be the set of all
  $\vect{y} \in \R^k$ such that $\abs{\vect{y} - \vect{x}} < r$ (or
  $\abs{\vect{y} - \vect{x}} \le r$).
  We call a set $E \subset \R^k$ convex if
  \[ \lambda \vect{x} + (1-\lambda)\vect{y} \in E \]
  whenever $\vect{x} \in E$, $\vect{y} \in E$, and $0 < \lambda < 1$.
  For example, balls are convex. For if $\abs{\vect{y} - \vect{x}} <
  r$, $\abs{\vect{z} - \vect{x}} < r$, and $0 < \lambda < 1$, we have
  \begin{align*}
    \abs{\lambda \vect{y} + (1-\lambda)\vect{z} - \vect{x}} &=
    \abs{\lambda(\vect{y} - \vect{x}) + (1-\lambda)(\vect{z} - \vect{x})} \\
    &\le \lambda\abs{\vect{y} - \vect{x}} + (1-\lambda)\abs{\vect{z}
    - \vect{x}} \\
    &< \lambda r + (1-\lambda)r = r.
  \end{align*}
  The same proof applies to closed balls. It is also easy to see that
  k-cells are convex.
\end{definition}

\begin{definition}[Topological Concepts in Metric Spaces] % Definition 2.18
  \label{def:chap2:topology_defs}
  Let X be a metric space. All points and sets mentioned below are
  understood to be elements and subsets of X.
  (a) A \emph{neighborhood} of p is a set $N_r(p)$ consisting of all
  q such that $d(p, q) < r$, for some $r>0$. The number r is called
  the radius of $N_r(p)$. \label{def:chap2:neighborhood}
  (b) A point p is a \emph{limit point} of the set E if every
  neighborhood of p contains a point $q \ne p$ such that $q \in E$.
  \label{def:chap2:limit_point}
  (c) If $p \in E$ and p is not a limit point of E, then p is called
  an \emph{isolated point} of E. \label{def:chap2:isolated_point}
  (d) E is \emph{closed} if every limit point of E is a point of E.
  \label{def:chap2:closed_set}
  (e) A point p is an \emph{interior point} of E if there is a
  neighborhood N of p such that $N \subset E$. \label{def:chap2:interior_point}
  (f) E is \emph{open} if every point of E is an interior point of E.
  \label{def:chap2:open_set}
  (g) The \emph{complement} of E (denoted by $E^c$) is the set of all
  points $p \in X$ such that $p \notin E$. \label{def:chap2:complement}
  (h) E is \emph{perfect} if E is closed and if every point of E is a
  limit point of E. \label{def:chap2:perfect_set}
  (i) E is \emph{bounded} if there is a real number M and a point $q
  \in X$ such that $d(p, q) < M$ for all $p \in E$.
  \label{def:chap2:bounded_set}
  (j) E is \emph{dense} in X if every point of X is a limit point of
  E, or a point of E (or both). \label{def:chap2:dense_set}
  Let us note that in $\R^1$ neighborhoods are segments, whereas in
  $\R^2$ neighborhoods are interiors of circles.
\end{definition}

\begin{theorem}[Neighborhood is Open] % Theorem 2.19
  \label{thm:chap2:neighborhood_open}
  Every neighborhood is an open set.
  \begin{proof}
    Consider a neighborhood $E = N_r(p)$, and let q be any point of
    E. Then there is a positive real number h such that $d(p, q) = r - h$.
    For all points s such that $d(q, s) < h$, we have then
    \[ d(p, s) \le d(p, q) + d(q, s) < (r-h) + h = r, \]
    so that $s \in E$. Thus q is an interior point of E
    (\autoref{def:chap2:topology_defs}(e)). Since q was arbitrary, E
    is open (\autoref{def:chap2:topology_defs}(f)).
  \end{proof}
\end{theorem}

\begin{theorem}[Limit Point Neighborhood Content] % Theorem 2.20
  \label{thm:chap2:limit_point_infinite}
  If p is a limit point of a set E, then every neighborhood of p
  contains infinitely many points of E.
  \begin{proof}
    Suppose there is a neighborhood N of p which contains only a
    finite number of points of E. Let $q_1, \dots, q_n$ be those
    points of $N \cap E$ which are distinct from p, and put
    \[ r = \min_{1 \le m \le n} d(p, q_m). \]
    [We use this notation to denote the smallest of the numbers $d(p,
    q_1), \dots, d(p, q_n)$.] The minimum of a finite set of positive
    numbers is clearly positive, so that $r > 0$.
    The neighborhood $N_r(p)$ contains no point q of E such that $q
    \ne p$. This contradicts the assumption that p is a limit point
    of E (\autoref{def:chap2:topology_defs}(b)). The contradiction
    establishes the theorem.
  \end{proof}
\end{theorem}

\begin{corollary} % Corollary to 2.20
  \label{cor:chap2:finite_set_no_limit_points}
  A finite point set has no limit points.
\end{corollary}

\begin{example}[Examples of Set Properties in R2] % Example 2.21
  \label{ex:chap2:set_property_examples}
  Let us consider the following subsets of $\R^2$:
  (a) The set of all complex z such that $\abs{z} < 1$.
  (b) The set of all complex z such that $\abs{z} \le 1$.
  (c) A nonempty finite set.
  (d) The set of all integers. (Can be seen as subset of $\R^1$ or $\R^2$).
  (e) The set consisting of the numbers $1/n$ ($n=1, 2, 3, \dots$).
  (Can be seen as subset of $\R^1$ or $\R^2$). Let us note that this
  set E has a limit point (namely, $z=0$) but that no point of E is a
  limit point of E; we wish to stress the difference between having a
  limit point and containing one.
  (f) The set of all complex numbers (that is, $\R^2$).
  (g) The segment (a, b). (Can be seen as subset of $\R^1$ or $\R^2$).

  Some properties of these sets are tabulated below:
  % Note: Rudin uses a table here. Reproducing simply.
  % (a) Closed: No, Open: Yes, Perfect: No, Bounded: Yes
  % (b) Closed: Yes, Open: No, Perfect: Yes, Bounded: Yes
  % (c) Closed: Yes, Open: No, Perfect: No, Bounded: Yes
  % (d) Closed: Yes, Open: No, Perfect: No, Bounded: No
  % (e) Closed: No, Open: No, Perfect: No, Bounded: Yes
  % (f) Closed: Yes, Open: Yes, Perfect: Yes, Bounded: No
  % (g) Closed: No, Open: (Depends on space - Yes in R1, No in R2),
  % Perfect: No, Bounded: Yes
  In (g), we left the second entry blank. The reason is that the
  segment (a, b) is not open if we regard it as a subset of $\R^2$,
  but it is an open subset of $\R^1$.
\end{example}

% --- End of content chunk ---

% --- Content from Thm 2.22 to Thm 2.30 (Proofs Omitted) to append ---

\begin{theorem}[De Morgan's Laws for Sets] % Theorem 2.22
  \label{thm:chap2:de_morgan_complements}
  Let $\{E_\alpha\}$ be a (finite or infinite) collection of sets
  $E_\alpha$. Then
  \begin{equation} \label{eq:chap2:de_morgan}
    \left( \bigunion{\alpha} E_\alpha \right)^c =
    \bigintersect{\alpha} (E_\alpha^c).
  \end{equation}
  % Proof Omitted
\end{theorem}

\begin{theorem}[Open iff Complement Closed] % Theorem 2.23
  \label{thm:chap2:open_iff_complement_closed}
  A set E is open if and only if its complement is closed.
  % Proof Omitted
\end{theorem}

\begin{corollary} % Corollary to 2.23
  \label{cor:chap2:closed_iff_complement_open}
  A set F is closed if and only if its complement is open.
\end{corollary}

\begin{theorem}[Union/Intersection of Open/Closed Sets] % Theorem 2.24
  \label{thm:chap2:union_intersect_open_closed}
  (a) For any collection $\{G_\alpha\}$ of open sets,
  $\bigunion{\alpha} G_\alpha$ is open. \label{thm:chap2:union_open}
  (b) For any collection $\{F_\alpha\}$ of closed sets,
  $\bigintersect{\alpha} F_\alpha$ is closed. \label{thm:chap2:intersect_closed}
  (c) For any finite collection $G_1, \dots, G_n$ of open sets,
  $\bigcap_{i=1}^n G_i$ is open. \label{thm:chap2:finite_intersect_open}
  (d) For any finite collection $F_1, \dots, F_n$ of closed sets,
  $\bigcup_{i=1}^n F_i$ is closed. \label{thm:chap2:finite_union_closed}
  % Proof Omitted
\end{theorem}

\begin{example}[Infinite Intersections/Unions] % Example 2.25
  \label{ex:chap2:infinite_intersect_union_counterex}
  In parts (c) and (d) of
  \autoref{thm:chap2:union_intersect_open_closed}, the finiteness of
  the collections is essential. For let $G_n$ be the segment $(-1/n,
  1/n)$ for $n=1, 2, 3, \dots$. Then $G_n$ is an open subset of
  $\R^1$. Put $G = \bigcap_{n=1}^\infty G_n$. Then G consists of a
  single point (namely, $x=0$) and is therefore not an open subset of
  $\R^1$. Thus the intersection of an infinite collection of open
  sets need not be open.
  Similarly, let $F_n = [1/n, 1]$. Each $F_n$ is closed in $\R^1$.
  But $\bigcup_{n=1}^\infty F_n = (0, 1]$, which is not closed (since
  0 is a limit point but not in the set). Thus the union of an
  infinite collection of closed sets need not be closed.
\end{example}

\begin{definition}[Closure of a Set] % Definition 2.26
  \label{def:chap2:closure}
  If X is a metric space, if $E \subset X$, and if $E'$ denotes the
  set of all limit points of E in X, then the \emph{closure} of E is
  the set $\overline{E} = E \cup E'$.
\end{definition}

\begin{theorem}[Properties of Closure] % Theorem 2.27
  \label{thm:chap2:closure_properties}
  If X is a metric space and $E \subset X$, then
  (a) $\overline{E}$ is closed.
  (b) $E = \overline{E}$ if and only if E is closed.
  (c) $\overline{E} \subset F$ for every closed set $F \subset X$
  such that $E \subset F$.
  By (a) and (c), $\overline{E}$ is the smallest closed subset of X
  that contains E.
  % Proof Omitted
\end{theorem}

\begin{theorem}[Supremum is in Closure] % Theorem 2.28
  \label{thm:chap2:sup_in_closure}
  Let E be a nonempty set of real numbers which is bounded above. Let
  $y = \sup E$. Then $y \in \overline{E}$. Hence $y \in E$ if E is closed.
  (Compare this with the examples in \autoref{ex:chap1:sup_inf_examples}.)
  % Proof Omitted
\end{theorem}

\begin{remark}[Open Relative to a Subset] % Remark 2.29
  \label{rem:chap2:relative_topology}
  Suppose $E \subset Y \subset X$, where X is a metric space. To say
  that E is an open subset of X means that to each point $p \in E$
  there is associated a positive number r such that the conditions
  $d(p, q) < r$, $q \in X$ imply that $q \in E$. But we have already
  observed (\autoref{ex:chap2:metric_space_examples}) that Y is also
  a metric space, so that our definitions may equally well be made
  within Y. To be quite explicit, let us say that E is open relative
  to Y if to each $p \in E$ there is associated an $r>0$ such that $q
  \in E$ whenever $d(p, q) < r$ and $q \in Y$.
  \autoref{ex:chap2:set_property_examples}(g) showed that a set may
  be open relative to Y without being an open subset of X. However,
  there is a simple relation between these concepts, which we now state.
\end{remark}

\begin{theorem}[Characterization of Relatively Open Sets] % Theorem 2.30
  \label{thm:chap2:relatively_open}
  Suppose $Y \subset X$. A subset E of Y is open relative to Y if and
  only if $E = Y \cap G$ for some open subset G of X.
  % Proof Omitted
\end{theorem}

% --- End of content chunk (Proofs Omitted) ---

% --- Content from Sec "Compact Sets" up to Thm 2.40 (Proofs Omitted)
% to append ---

\section{Compact Sets}
\label{sec:chap2:compact_sets}

\begin{definition}[Open Cover] % Definition 2.31
  \label{def:chap2:open_cover}
  By an open cover of a set E in a metric space X we mean a
  collection $\{G_\alpha\}$ of open subsets of X such that $E \subset
  \bigunion{\alpha} G_\alpha$.
\end{definition}

\begin{definition}[Compact Set] % Definition 2.32
  \label{def:chap2:compact_set}
  A subset K of a metric space X is said to be compact if every open
  cover of K contains a finite subcover.
  More explicitly, the requirement is that if $\{G_\alpha\}$ is an
  open cover of K, then there are finitely many indices $\alpha_1,
  \dots, \alpha_n$ such that
  \[ K \subset G_{\alpha_1} \cup \dots \cup G_{\alpha_n}. \]
  The notion of compactness is of great importance in analysis,
  especially in connection with continuity (Chap. 4).
  It is clear that every finite set is compact. The existence of a
  large class of infinite compact sets in $\R^k$ will follow from
  \autoref{thm:chap2:heine_borel}. % Theorem 2.41
  We observed earlier (in \autoref{rem:chap2:relative_topology}) that
  if $E \subset Y \subset X$, then E may be open relative to Y
  without being open relative to X. The property of being open thus
  depends on the space in which E is embedded. The same is true of
  the property of being closed.
  Compactness, however, behaves better, as we shall now see. To
  formulate the next theorem, let us say, temporarily, that K is
  compact relative to X if the requirements of
  \autoref{def:chap2:compact_set} are met.
\end{definition}

\begin{theorem}[Compactness is Intrinsic] % Theorem 2.33
  \label{thm:chap2:compactness_intrinsic}
  Suppose $K \subset Y \subset X$. Then K is compact relative to X if
  and only if K is compact relative to Y.
  By virtue of this theorem we are able, in many situations, to
  regard compact sets as metric spaces in their own right, without
  paying any attention to any embedding space. In particular,
  although it makes little sense to talk of open spaces, or of closed
  spaces (every metric space X is an open subset of itself, and is a
  closed subset of itself), it does make sense to talk of compact metric spaces.
  % Proof Omitted
\end{theorem}

\begin{theorem}[Compact Sets are Closed] % Theorem 2.34
  \label{thm:chap2:compact_implies_closed}
  Compact subsets of metric spaces are closed.
  % Proof Omitted
\end{theorem}

\begin{theorem}[Closed Subsets of Compact Sets are Compact] % Theorem 2.35
  \label{thm:chap2:closed_subset_compact}
  Closed subsets of compact sets are compact.
  % Proof Omitted
\end{theorem}

\begin{corollary} % Corollary to 2.35
  \label{cor:chap2:closed_intersect_compact}
  If F is closed and K is compact, then $F \cap K$ is compact.
  % Proof uses Thm 2.24(b), Thm 2.34, Thm 2.35
\end{corollary}

\begin{theorem}[Intersection Property of Compact Sets] % Theorem 2.36
  \label{thm:chap2:compact_intersection_property}
  If $\{K_\alpha\}$ is a collection of compact subsets of a metric
  space X such that the intersection of every finite subcollection of
  $\{K_\alpha\}$ is nonempty, then $\bigintersect{\alpha} K_\alpha$ is nonempty.
  % Proof Omitted
\end{theorem}

\begin{corollary}[Nested Compact Sets] % Corollary to 2.36
  \label{cor:chap2:nested_compact_nonempty}
  If $\{K_n\}$ is a sequence of nonempty compact sets such that $K_n
  \supset K_{n+1}$ ($n=1, 2, 3, \dots$), then $\bigcap_{n=1}^\infty
  K_n$ is not empty.
\end{corollary}

\begin{theorem}[Limit Point Property of Compact Sets] % Theorem 2.37
  \label{thm:chap2:compact_limit_point_property}
  If E is an infinite subset of a compact set K, then E has a limit point in K.
  % Proof Omitted
\end{theorem}

\begin{theorem}[Nested Intervals Theorem for R1] % Theorem 2.38
  \label{thm:chap2:nested_intervals_R1}
  If $\{I_n\}$ is a sequence of intervals in $\R^1$, such that $I_n =
  [a_n, b_n]$ and $I_n \supset I_{n+1}$ ($n=1, 2, 3, \dots$), then
  $\bigcap_{n=1}^\infty I_n$ is not empty.
  % Proof Omitted
\end{theorem}

\begin{theorem}[Nested k-cells Theorem] % Theorem 2.39
  \label{thm:chap2:nested_k_cells}
  Let k be a positive integer. If $\{I_n\}$ is a sequence of k-cells
  such that $I_n \supset I_{n+1}$ ($n=1, 2, 3, \dots$), then
  $\bigcap_{n=1}^\infty I_n$ is not empty.
  % Proof Omitted
\end{theorem}

\begin{theorem}[k-cells are Compact] % Theorem 2.40
  \label{thm:chap2:k_cell_compact}
  Every k-cell is compact.
  % Proof Omitted
\end{theorem}

% --- End of content chunk ---

% --- Content from Thm 2.41 to Thm 2.47 (Proofs Omitted) to append ---

\begin{theorem}[Heine-Borel] % Theorem 2.41
  \label{thm:chap2:heine_borel}
  If a set E in $\R^k$ has one of the following three properties,
  then it has the other two:
  (a) E is closed and bounded.
  (b) E is compact.
  (c) Every infinite subset of E has a limit point in E.
  % Proof Omitted (relies on Thm 2.40, 2.35, 2.37)
\end{theorem}

\begin{theorem}[Weierstrass] % Theorem 2.42
  \label{thm:chap2:weierstrass}
  Every bounded infinite subset of $\R^k$ has a limit point in $\R^k$.
  % Proof Omitted (relies on Thm 2.40, 2.37)
\end{theorem}

\section{Perfect Sets}
\label{sec:chap2:perfect_sets}

\begin{theorem}[Perfect Sets are Uncountable] % Theorem 2.43
  \label{thm:chap2:perfect_sets_uncountable}
  Let P be a nonempty perfect set in $\R^k$. Then P is uncountable.
  % Proof Omitted (uses nested neighborhoods and Thm 2.36 Corollary)
\end{theorem}

\begin{corollary} % Corollary to 2.43
  \label{cor:chap2:intervals_uncountable}
  Every interval [a, b] ($a<b$) is uncountable. In particular, the
  set of all real numbers is uncountable.
\end{corollary}

\begin{example}[The Cantor Set] % Example 2.44
  \label{ex:chap2:cantor_set}
  The set which we are now going to construct shows that there exist
  perfect sets in $\R^1$ which contain no segment.
  Let $E_0$ be the interval $[0, 1]$. Remove the segment $(1/3,
  2/3)$, and let $E_1$ be the union of the intervals
  \[ [0, 1/3] \cup [2/3, 1]. \]
  Remove the middle thirds of these intervals, and let $E_2$ be the
  union of the intervals
  \[ [0, 1/9] \cup [2/9, 1/3] \cup [2/3, 7/9] \cup [8/9, 1]. \]
  Continuing in this way, we obtain a sequence of compact sets $E_n$ such that
  (a) $E_1 \supset E_2 \supset E_3 \supset \dots$;
  (b) $E_n$ is the union of $2^n$ intervals, each of length $3^{-n}$.
  The set
  \[ P = \bigcap_{n=1}^\infty E_n \]
  is called the Cantor set. P is clearly compact
  (\autoref{thm:chap2:intersect_closed},
  \autoref{thm:chap2:closed_subset_compact}), and
  \autoref{cor:chap2:nested_compact_nonempty} shows that P is not empty.
  No segment of the form $(\frac{3k+1}{3^m}, \frac{3k+2}{3^m})$ can
  be a subset of P. Since every segment $(\alpha, \beta)$ contains
  such a segment if $3^{-m} < \beta - \alpha$, P contains no segment.
  To show that P is perfect, note that P is closed (being an
  intersection of closed sets). Let $x \in P$. Let $N$ be any
  neighborhood of x. Let $I_n$ be that interval of $E_n$ which
  contains x. Choose n large enough so that $I_n \subset N$. Let
  $x_n$ be an endpoint of $I_n$ such that $x_n \ne x$. It follows
  from the construction of the set P that $x_n \in P$. Hence x is a
  limit point of P, and P is perfect.
  (The Cantor set has many other curious properties.)
\end{example}

\section{Connected Sets}
\label{sec:chap2:connected_sets}

\begin{definition}[Separated Sets] % Definition 2.45
  \label{def:chap2:separated_sets}
  Two subsets A and B of a metric space X are said to be separated if
  both $A \cap \overline{B}$ and $\overline{A} \cap B$ are empty,
  i.e., if no point of A lies in the closure of B and no point of B
  lies in the closure of A.
\end{definition}

\begin{definition}[Connected Set] % Definition 2.46
  \label{def:chap2:connected_set}
  A set $E \subset X$ is said to be connected if E is not a union of
  two nonempty separated sets.
\end{definition}

\begin{theorem}[Connected Subsets of R1] % Theorem 2.47
  \label{thm:chap2:connected_R1_iff_interval}
  A subset E of the real line $\R^1$ is connected if and only if it
  has the following property: If $x \in E$, $y \in E$, and $x < z <
  y$, then $z \in E$.
  % Proof Omitted
  (The property described in this theorem is possessed by intervals
    and segments (and rays, and $\R^1$ itself). It is usually stated by
  saying that E is an interval.)
\end{theorem}

% --- End of Chapter 2 content chunk (Proofs Omitted) ---

% --- End of chapters/chapter2.tex ---
