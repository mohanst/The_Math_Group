% --- chapters/chapter5.tex ---
% Generated by Gemini (Google AI) on 2025-04-05.
% Contains extracted items from W. Rudin, PMA, Chapter 5.
% Assumes macros from user-provided macros.tex are defined.
% Proofs are omitted as requested.

\chapter{Differentiation}
\label{chap:rudin5}

\section{The Derivative of a Real Function}
\label{sec:chap5:derivative_real}

\begin{definition}[Derivative] % Definition 5.1
  \label{def:chap5:derivative}
  Let $\map{f}{(a, b)}{\R}$. For any $x \in (a, b)$, form the quotient
  \[ \phi(t) = \frac{f(t) - f(x)}{t - x} \quad (a < t < b, t \ne x). \]
  If the limit
  \[ f'(x) = \lim_{t \to x} \phi(t) \]
  exists, we say that f is differentiable at x, and $f'(x)$ is called
  the derivative of f at x. If f is differentiable at every point of
  a set $E \subset (a, b)$, we say f is differentiable on E.
\end{definition}

\begin{theorem}[Differentiability implies Continuity] % Theorem 5.2
  \label{thm:chap5:diff_implies_cont}
  Let $\map{f}{(a, b)}{\R}$. If f is differentiable at a point $x \in
  (a, b)$, then f is continuous at x (\autoref{def:chap4:continuity}).
  % Proof Omitted
\end{theorem}

\begin{theorem}[Algebra of Derivatives] % Theorem 5.3
  \label{thm:chap5:derivative_algebra}
  Suppose f and g are defined on $[a, b]$ and are differentiable at a
  point $x \in [a, b]$. Then $f+g$, $fg$, and $f/g$ (if $g(x) \ne 0$)
  are differentiable at x, and
  (a) $(f+g)'(x) = f'(x) + g'(x)$;
  (b) $(fg)'(x) = f'(x)g(x) + f(x)g'(x)$;
  (c) $\left( \frac{f}{g} \right)'(x) = \frac{g(x)f'(x) -
  g'(x)f(x)}{g(x)^2}$, if $g(x) \ne 0$.
  % Proof Omitted
\end{theorem}

% Skipping Example 5.4 text here for brevity, as it's descriptive.
% Example 5.4 discusses derivatives of x^n.

% --- MISSING ITEM (Insert between Thm 5.3 and Thm 5.5) ---

\begin{example}[Derivative of $x^n$] % Example 5.4
  \label{ex:chap5:derivative_xn}
  If $f(x) = x^n$ where n is a positive integer, then $f'(x) =
  nx^{n-1}$. This can be shown by induction using
  \autoref{thm:chap5:derivative_algebra}(b) starting from $f(x)=x$
  (where $f'(x)=1$) or directly from the definition. For $n=0$,
  $f(x)=1$, $f'(x)=0$. For negative integers n, the result follows
  from \autoref{thm:chap5:derivative_algebra}(c). For rational n=p/q,
  $f(x)=x^{p/q}$, we can use implicit differentiation or the chain
  rule (if $y=x^{p/q}$, $y^q=x^p$).
\end{example}

% --- End of missing item ---

\begin{theorem}[Chain Rule] % Theorem 5.5
  \label{thm:chap5:chain_rule}
  Suppose f is continuous on $[a, b]$, $f'(x)$ exists at some point
  $x \in [a, b]$, g is defined on an interval I which contains the
  range of f, and g is differentiable at the point $f(x)$. If $h(t) =
  g(f(t))$ for $t \in [a, b]$, then h is differentiable at x, and
  \[ h'(x) = g'(f(x)) f'(x). \]
  % Proof Omitted
\end{theorem}

% --- End of content chunk ---

% --- Content from Ex 5.6 to Thm 5.11 (Proofs Omitted) to append ---

\begin{example}[Derivative of $x^2 \sin(1/x)$] % Example 5.6
  \label{ex:chap5:x2sin1x_example}
  Let f be defined by
  \[ f(x) =
    \begin{cases} x^2 \sin(1/x) & (x \ne 0) \\ 0 & (x=0)
  \end{cases} \]
  Then $f'(x)$ exists for all x, but $f'$ is not continuous at $x=0$.
  For $x \ne 0$, $f'(x) = 2x \sin(1/x) - \cos(1/x)$ by
  \autoref{thm:chap5:derivative_algebra} and \autoref{thm:chap5:chain_rule}.
  At $x=0$, we look at the limit of $\frac{f(t)-f(0)}{t-0} =
  \frac{t^2 \sin(1/t)}{t} = t \sin(1/t)$. Since $\abs{t \sin(1/t)}
  \le \abs{t}$, this limit is 0 as $t \to 0$. Thus $f'(0)=0$.
  However, as $x \to 0$, $\cos(1/x)$ does not approach a limit, so
  $\lim_{x \to 0} f'(x)$ does not exist. Thus $f'$ is discontinuous at $x=0$.
\end{example}

\section{Mean Value Theorems}
\label{sec:chap5:mvt}

\begin{definition}[Local Extrema] % Definition 5.7
  \label{def:chap5:local_extrema}
  Let $\map{f}{E}{\R}$, where $E \subset \R$. We say that f has a
  local maximum at a point $p \in E$ if there exists $\delta > 0$
  such that $f(q) \le f(p)$ for all $q \in E$ with $d(p, q) <
  \delta$. Local minimum is defined likewise with $f(q) \ge f(p)$.
\end{definition}

\begin{theorem}[Local Extremum Implies Zero Derivative] % Theorem 5.8
  \label{thm:chap5:local_extremum_zero_deriv}
  Let $\map{f}{(a, b)}{\R}$. If f has a local maximum or minimum at
  $x \in (a, b)$, and if $f'(x)$ exists, then $f'(x) = 0$.
  % Proof Omitted
\end{theorem}

\begin{theorem}[Generalized Mean Value Theorem - Cauchy MVT] % Theorem 5.9
  \label{thm:chap5:cauchy_mvt}
  If f and g are continuous real functions on $[a, b]$ which are
  differentiable in $(a, b)$, then there is a point $x \in (a, b)$ at which
  \[ [f(b) - f(a)]g'(x) = [g(b) - g(a)]f'(x). \]
  (Note that differentiability is not required at the endpoints.)
  % Proof Omitted (Applies Rolle's Thm, which is Thm 5.8 applied, to
  % h(t)=[f(b)-f(a)]g(t)-[g(b)-g(a)]f(t))
\end{theorem}

\begin{theorem}[Mean Value Theorem - Lagrange MVT] % Theorem 5.10
  \label{thm:chap5:lagrange_mvt}
  If f is a real continuous function on $[a, b]$ which is
  differentiable in $(a, b)$, then there is a point $x \in (a, b)$ at which
  \[ f(b) - f(a) = (b - a)f'(x). \]
  % Proof Omitted (Special case of Thm 5.9 with g(x)=x)
\end{theorem}

\begin{theorem}[Derivative Sign Implies Monotonicity] % Theorem 5.11
  \label{thm:chap5:deriv_sign_monotonicity}
  Suppose f is differentiable in $(a, b)$.
  (a) If $f'(x) \ge 0$ for all $x \in (a, b)$, then f is
  monotonically increasing.
  (b) If $f'(x) = 0$ for all $x \in (a, b)$, then f is constant.
  (c) If $f'(x) \le 0$ for all $x \in (a, b)$, then f is
  monotonically decreasing.
  % Proof Omitted (Uses Thm 5.10)
\end{theorem}

% --- End of content chunk ---

% --- Content from Thm 5.12 to Thm 5.15 (Proofs Omitted) to append ---

\section{The Continuity of Derivatives}
\label{sec:chap5:continuity_derivatives}

\begin{theorem}[Intermediate Value Theorem for Derivatives -
  Darboux's Thm] % Theorem 5.12
  \label{thm:chap5:darboux_ivt_derivative}
  Suppose f is a real differentiable function on $[a, b]$ and suppose
  $f'(a) < \lambda < f'(b)$. Then there is a point $x \in (a, b)$
  such that $f'(x) = \lambda$.
  A similar result holds if $f'(a) > f'(b)$.
  (Corollary: If f is differentiable on $[a, b]$, then $f'$ cannot
    have any simple discontinuities
  (\autoref{def:chap4:discontinuity_types}) on $[a, b]$.)
  % Proof Omitted
\end{theorem}

\section{L'Hospital's Rule}
\label{sec:chap5:lhospitals_rule}

\begin{theorem}[L'Hospital's Rule] % Theorem 5.13
  \label{thm:chap5:lhospitals_rule}
  Suppose f and g are real and differentiable in $(a, b)$, and $g'(x)
  \ne 0$ for all $x \in (a, b)$, where $-\infty \le a < b \le +\infty$. Suppose
  \[ \frac{f'(x)}{g'(x)} \to A \quad \text{as } x \to a. \]
  If $f(x) \to 0$ and $g(x) \to 0$ as $x \to a$, or if $g(x) \to
  +\infty$ as $x \to a$, then
  \[ \frac{f(x)}{g(x)} \to A \quad \text{as } x \to a. \]
  The analogous statement is also true if $x \to b$, or if $g(x) \to -\infty$.
  % Proof Omitted (Uses Generalized Mean Value Theorem, Thm 5.9)
\end{theorem}

\section{Derivatives of Higher Order}
\label{sec:chap5:higher_derivatives}

\begin{definition}[Higher Order Derivatives] % Definition 5.14
  \label{def:chap5:higher_deriv_notation}
  If f has a derivative $f'$ on an interval, and if $f'$ is itself
  differentiable, we denote the derivative of $f'$ by $f''$ and call
  $f''$ the second derivative of f. Continuing in this manner, we
  obtain functions
  \[ f, f', f'', f^{(3)}, \dots, f^{(n)}, \]
  each of which is the derivative of the preceding one. $f^{(n)}$ is
  called the n-th derivative, or the derivative of order n, of f.
  In order for $f^{(n)}(x)$ to exist at a point x, $f^{(n-1)}(t)$
  must exist in a neighborhood of x (or in a one-sided neighborhood,
  if x is an endpoint of the interval on which f is defined), and
  $f^{(n-1)}$ must be differentiable at x. Since $f^{(k-1)}$ must be
  continuous in order to be differentiable, existence of $f^{(n)}(x)$
  implies continuity of $f, f', \dots, f^{(n-1)}$ in a neighborhood of x.
\end{definition}

\section{Taylor's Theorem}
\label{sec:chap5:taylors_theorem}

\begin{theorem}[Taylor's Theorem] % Theorem 5.15
  \label{thm:chap5:taylors_theorem}
  Suppose f is a real function on $[a, b]$, n is a positive integer,
  $f^{(n-1)}$ is continuous on $[a, b]$, $f^{(n)}(t)$ exists for
  every $t \in (a, b)$. Let $\alpha, \beta$ be distinct points of
  $[a, b]$, and define
  \[ P(t) = \sum_{k=0}^{n-1} \frac{f^{(k)}(\alpha)}{k!} (t - \alpha)^k. \]
  Then there exists a point x between $\alpha$ and $\beta$ such that
  \[ f(\beta) = P(\beta) + \frac{f^{(n)}(x)}{n!} (\beta - \alpha)^n. \]
  (For n=1, this is just the Mean Value Theorem
    \autoref{thm:chap5:lagrange_mvt}. P(t) is the Taylor polynomial of
  degree n-1 for f about $\alpha$.)
  % Proof Omitted (Applies Rolle's Thm / Thm 5.8 to a helper function)
\end{theorem}

% --- End of content chunk ---

% --- Regenerated content from Def 5.16 to Thm 5.19 (Proofs Omitted) ---
% --- Replace the previous incorrect block for 5.16-5.19 in
% chapters/chapter5.tex ---

\section{Differentiation of Vector-valued Functions}
\label{sec:chap5:diff_vector_valued}

\begin{definition}[Derivative of Vector-valued Function] % Definition 5.16
  \label{def:chap5:deriv_vector_valued}
  Let $\map{\vect{f}}{[a, b]}{\R^k}$. For any $x \in [a, b]$, the
  derivative $\vect{f}'(x)$ is defined by
  \[ \vect{f}'(x) = \lim_{t \to x} \frac{\vect{f}(t) - \vect{f}(x)}{t - x}, \]
  provided this limit exists. (The limit is taken in $\R^k$). If
  $\vect{f}'(x)$ exists, we say $\vect{f}$ is differentiable at x. If
  $\vect{f}'(x)$ exists for every $x \in [a, b]$, we say $\vect{f}$
  is differentiable on $[a, b]$.
  (If $\vect{f}(t) = (f_1(t), \dots, f_k(t))$, it follows from
    \autoref{thm:chap3:convergence_components} and
    \autoref{thm:chap4:cont_vector_coords} that $\vect{f}'(x)$ exists
  if and only if $f'_j(x)$ exists for $j=1, \dots, k$.)
\end{definition}

\begin{theorem}[Component-wise Differentiation] % Theorem 5.17
  \label{thm:chap5:component_wise_diff}
  Let $\vect{f}(t) = (f_1(t), \dots, f_k(t))$ map $[a, b]$ into
  $\R^k$. Then $\vect{f}$ is differentiable at x if and only if each
  $f_j$ is differentiable at x, and in this case
  \[ \vect{f}'(x) = (f'_1(x), \dots, f'_k(x)). \]
  % Proof Omitted
\end{theorem}

\begin{theorem}[Properties of Vector-valued Derivatives] % Theorem 5.18
  \label{thm:chap5:vector_deriv_props}
  Suppose $\vect{f}$ and $\vect{g}$ map $[a, b]$ into $\R^k$, $g$ is
  a real function defined on $[a, b]$, and $\vect{f}$, $\vect{g}$,
  and $g$ are differentiable at a point $x \in [a, b]$. Then
  $\vect{f}+\vect{g}$ and $g\vect{f}$ are differentiable at x, and
  (a) $(\vect{f}+\vect{g})'(x) = \vect{f}'(x) + \vect{g}'(x)$.
  (b) $(g\vect{f})'(x) = g(x)\vect{f}'(x) + g'(x)\vect{f}(x)$.
  (Note: Rudin's text uses 'f' and 'g' for vector functions and 'g'
    also for the real function in part (b). I've kept 'g' for the real
  function here for clarity matching the text structure.)
  % Proof Omitted
\end{theorem}

% Note: The chain rule discussion follows here in the text but is unnumbered.

\begin{theorem}[Vector MVT Inequality] % Theorem 5.19
  \label{thm:chap5:vector_mvt_inequality}
  Suppose $\map{\vect{f}}{[a, b]}{\R^k}$ and $\vect{f}$ is
  differentiable in $(a, b)$. Then there exists $x \in (a, b)$ such that
  \[ \norm{\vect{f}(b) - \vect{f}(a)} \le (b - a) \norm{\vect{f}'(x)}. \]
  (Note: This uses `$\abs{}$` for the norm in $\R^k$, matching
    Rudin's notation, assuming `$\abs{}$` handles vectors appropriately
  or you use `|...|`.)
  % Proof Omitted
\end{theorem}

% --- End of Chapter 5 main content ---
% --- End of regenerated chunk (5.16 to 5.19) ---
