% --- chapters/chapter1.tex ---
% Generated by Gemini (Google AI) on 2025-04-04.
% Contains extracted items from W. Rudin, PMA, Chapter 1, up to Def 1.8.

\chapter{The Real and Complex Number Systems}
\label{chap:rudin1}

\begin{example}[Proof that $p^2=2$ has no rational solution] % Example 1.1
  \label{ex:chap1:p_squared_eq_2_no_rational_sol}
  We now show that the equation
  \begin{equation}
    p^{2}=2
  \end{equation}
  is not satisfied by any rational p. If there were such a p, we
  could write $p=m/n$ where m and n are integers that are not both
  even. Let us assume this is done. Then (1) implies
  \begin{equation}
    m^{2}=2n^{2}.
  \end{equation}
  This shows that $m^{2}$ is even. Hence m is even (if m were odd,
  $m^{2}$ would be odd), and so $m^{2}$ is divisible by 4. It follows
  that the right side of (2) is divisible by 4, so that $n^{2}$ is
  even, which implies that n is even.
  The assumption that (1) holds thus leads to the conclusion that
  both m and n are even, contrary to our choice of m and n. Hence (1)
  is impossible for rational p.

  We now examine this situation a little more closely. Let A be the
  set of all positive rationals p such that $p^{2}<2$ and let B
  consist of all positive rationals p such that $p^{2}>2$. We shall
  show that A contains no largest number and B contains no smallest.
  More explicitly, for every p in A we can find a rational q in A
  such that $p<q$, and for every p in B we can find a rational q in B
  such that $q<p$.
  To do this, we associate with each rational $p>0$ the number
  \begin{equation}
    q=p-\frac{p^{2}-2}{p+2}=\frac{2p+2}{p+2}.
  \end{equation}
  Then
  \begin{equation}
    q^{2}-2=\frac{2(p^{2}-2)}{(p+2)^{2}}.
  \end{equation}
  If p is in A then $p^{2}-2<0$, (3) shows that $q>p$, and (4) shows
  that $q^{2}<2$. Thus q is in A.
  If p is in B then $p^{2}-2>0$, (3) shows that $0<q<p$, and (4)
  shows that $q^{2}>2$. Thus q is in B.
\end{example}

\begin{remark}[Gaps in the Rational Number System] % Remark 1.2
  \label{rem:chap1:gaps_in_Q}
  The purpose of the above discussion has been to show that the
  rational number system $\mathbb{Q}$ has certain gaps, in spite of
  the fact that between any two rationals there is another: if $r<s$
  then $r<(r+s)/2<s$. The real number system fills these gaps. This
  is the principal reason for the fundamental role which it plays in analysis.
\end{remark}

\begin{definition}[Set Notation] % Definition 1.3
  \label{def:chap1:set_notation}
  If A is any set (whose elements may be numbers or any other
  objects), we write $x \in A$ to indicate that x is a member (or an
  element) of A.
  If x is not a member of A, we write: $x \notin A$.
  The set which contains no element will be called the empty set. If
  a set has at least one element, it is called nonempty.
  If A and B are sets, and if every element of A is an element of B,
  we say that A is a subset of B, and write $A \subset B$, or $B
  \supset A$. If, in addition, there is an element of B which is not
  in A, then A is said to be a proper subset of B. Note that $A
  \subset A$ for every set A.
  If $A \subset B$ and $B \subset A$, we write $A=B$. Otherwise $A \ne B$.
\end{definition}

\begin{definition}[Notation for Rationals] % Definition 1.4
  \label{def:chap1:Q_notation}
  Throughout Chap. 1, the set of all rational numbers will be denoted
  by $\mathbb{Q}$.
\end{definition}

\begin{definition}[Order Relation] % Definition 1.5
  \label{def:chap1:order_relation}
  Let S be a set. An order on S is a relation, denoted by $<$, with
  the following two properties:
  (i) If $x \in S$ and $y \in S$ then one and only one of the statements
  \[ x<y, \quad x=y, \quad y<x \]
  is true.
  (ii) If x, y, $z \in S$, if $x<y$ and $y<z$, then $x<z$.

  The statement "$x<y$" may be read as "x is less than y" or "x is
  smaller than y" or "x precedes y".
  It is often convenient to write $y>x$ in place of $x<y$.
  The notation $x \le y$ indicates that $x<y$ or $x=y$, without
  specifying which of these two is to hold. In other words, $x \le y$
  is the negation of $x>y$.
\end{definition}

\begin{definition}[Ordered Set] % Definition 1.6
  \label{def:chap1:ordered_set}
  An ordered set is a set S in which an order is defined. For
  example, $\mathbb{Q}$ is an ordered set if $r<s$ is defined to mean
  that $s-r$ is a positive rational number.
\end{definition}

\begin{definition}[Bounds] % Definition 1.7
  \label{def:chap1:bounds}
  Suppose S is an ordered set, and $E \subset S$. If there exists a
  $\beta \in S$ such that $x \le \beta$ for every $x \in E$, we say
  that E is bounded above, and call $\beta$ an upper bound of E.
  Lower bounds are defined in the same way (with $\ge$ in place of $\le$).
\end{definition}

\begin{definition}[Supremum and Infimum] % Definition 1.8
  \label{def:chap1:supremum_infimum}
  Suppose S is an ordered set, $E \subset S$, and E is bounded above.
  Suppose there exists an $\alpha \in S$ with the following properties:
  (i) $\alpha$ is an upper bound of E.
  (ii) If $\gamma < \alpha$ then $\gamma$ is not an upper bound of E.
  Then $\alpha$ is called the least upper bound of E [that there is
  at most one such $\alpha$ is clear from (ii)] or the supremum of E,
  and we write
  \[ \alpha = \sup E. \]
  The greatest lower bound, or infimum, of a set E which is bounded
  below is defined in the same manner: The statement
  \[ \alpha = \inf E \]
  means that $\alpha$ is a lower bound of E and that no $\beta$ with
  $\beta > \alpha$ is a lower bound of E.
\end{definition}

% Add remaining definitions, theorems, examples etc. here later...

% --- Regenerated content from Ex 1.9 to Prop 1.16 using macros ---
% --- Replace the corresponding section in chapters/chapter1.tex ---

\begin{example}[Supremum/Infimum Examples] % Example 1.9
  \label{ex:chap1:sup_inf_examples}
  (a) Consider the sets A and B of Example 1.1 as subsets of the
  ordered set $\Q$. The set A is bounded above. In fact, the upper
  bounds of A are exactly the members of B. Since B contains no
  smallest member, A has no least upper bound in $\Q$.
  Similarly, B is bounded below: The set of all lower bounds of B
  consists of A and of all $r \in \Q$ with $r \le 0$. Since A has no
  largest member, B has no greatest lower bound in $\Q$.

  (b) If $\alpha = \sup E$ exists, then $\alpha$ may or may not be a
  member of E. For instance, let $E_1$ be the set of all $r \in \Q$
  with $r < 0$. Let $E_2$ be the set of all $r \in \Q$ with $r \le 0$. Then
  \[ \sup E_1 = \sup E_2 = 0, \]
  and $0 \notin E_1$, $0 \in E_2$.

  (c) Let E consist of all numbers $1/n$, where $n = 1, 2, 3, \dots$.
  Then $\sup E = 1$, which is in E, and $\inf E = 0$, which is not in E.
\end{example}

\begin{definition}[Least-Upper-Bound Property] % Definition 1.10
  \label{def:chap1:lub_property}
  An ordered set S is said to have the least-upper-bound property if
  the following is true:
  If $E \subset S$, E is not empty, and E is bounded above, then
  $\sup E$ exists in S.
  Example 1.9(a) shows that $\Q$ does not have the least-upper-bound property.
  We shall now show that there is a close relation between greatest
  lower bounds and least upper bounds, and that every ordered set
  with the least-upper-bound property also has the
  greatest-lower-bound property.
\end{definition}

\begin{theorem}[Existence of Infimum] % Theorem 1.11
  \label{thm:chap1:inf_exists_if_lub_property}
  Suppose S is an ordered set with the least-upper-bound property, $B
  \subset S$, B is not empty, and B is bounded below. Let L be the
  set of all lower bounds of B. Then
  \[ \alpha = \sup L \]
  exists in S, and $\alpha = \inf B$.
  In particular, $\inf B$ exists in S.
  \begin{proof}
    Since B is bounded below, L is not empty. Since L consists of
    exactly those $y \in S$ which satisfy the inequality $y \le x$
    for every $x \in B$, we see that every $x \in B$ is an upper
    bound of L. Thus L is bounded above. Our hypothesis about S
    implies therefore that L has a supremum in S; call it $\alpha$.
    If $\gamma < \alpha$ then (see Definition 1.8) $\gamma$ is not an
    upper bound of L, hence $\gamma \notin B$. It follows that
    $\alpha \le x$ for every $x \in B$. Thus $\alpha \in L$.
    If $\alpha < \beta$ then $\beta \notin L$, since $\alpha$ is an
    upper bound of L.
    We have shown that $\alpha \in L$ but $\beta \notin L$ if $\beta
    > \alpha$. In other words, $\alpha$ is a lower bound of B, but
    $\beta$ is not if $\beta > \alpha$. This means that $\alpha = \inf B$.
  \end{proof}
\end{theorem}

\begin{definition}[Field Axioms] % Definition 1.12
  \label{def:chap1:field_axioms}
  A field is a set F with two operations, called addition and
  multiplication, which satisfy the following so-called "field
  axioms" (A), (M), and (D):
  \begin{description}
    \item[(A)] Axioms for addition
      \begin{description}
        \item[(A1)] If $x \in F$ and $y \in F$, then their sum $x+y$ is in F.
        \item[(A2)] Addition is commutative: $x+y = y+x$ for all x, y $\in F$.
        \item[(A3)] Addition is associative: $(x+y)+z = x+(y+z)$ for
          all x, y, $z \in F$.
        \item[(A4)] F contains an element 0 such that $0+x = x$ for
          every $x \in F$.
        \item[(A5)] To every $x \in F$ corresponds an element $-x \in
          F$ such that $x+(-x) = 0$.
      \end{description}
    \item[(M)] Axioms for multiplication
      \begin{description}
        \item[(M1)] If $x \in F$ and $y \in F$, then their product $xy$ is in F.
        \item[(M2)] Multiplication is commutative: $xy = yx$ for all
          $x, y \in F$.
        \item[(M3)] Multiplication is associative: $(xy)z = x(yz)$
          for all x, y, $z \in F$.
        \item[(M4)] F contains an element $1 \ne 0$ such that $1x =
          x$ for every $x \in F$.
        \item[(M5)] If $x \in F$ and $x \ne 0$ then there exists an
          element $1/x \in F$ such that $x \cdot (1/x) = 1$.
      \end{description}
    \item[(D)] The distributive law
      \[ x(y+z) = xy + xz \]
      holds for all x, y, $z \in F$.
  \end{description}
\end{definition}

\begin{remark}[Field Remarks] % Remark 1.13
  \label{rem:chap1:field_remarks}
  (a) One usually writes (in any field)
  \[ x-y, \quad \frac{x}{y}, \quad x+y+z, \quad xyz, \quad x^{2},
  \quad x^{3}, \quad 2x, \quad 3x, \dots \]
  in place of
  \[ x+(-y), \quad x \cdot (\frac{1}{y}), \quad (x+y)+z, \quad (xy)z,
  \quad xx, \quad xxx, \quad x+x, \quad x+x+x, \dots \]
  (b) The field axioms clearly hold in $\Q$, the set of all rational
  numbers, if addition and multiplication have their customary
  meaning. Thus $\Q$ is a field.
  (c) Although it is not our purpose to study fields (or any other
  algebraic structures) in detail, it is worthwhile to prove that
  some familiar properties of $\Q$ are consequences of the field
  axioms; once we do this, we will not need to do it again for the
  real numbers and for the complex numbers.
\end{remark}

\begin{proposition}[Consequences of Addition Axioms] % Proposition 1.14
  \label{prop:chap1:addition_axiom_consequences}
  The axioms for addition imply the following statements.
  (a) If $x+y = x+z$ then $y=z$.
  (b) If $x+y = x$ then $y=0$.
  (c) If $x+y = 0$ then $y=-x$.
  (d) $-(-x) = x$.
  Statement (a) is a cancellation law. Note that (b) asserts the
  uniqueness of the element whose existence is assumed in (A4), and
  that (c) does the same for (A5).
  \begin{proof}
    If $x+y = x+z$, the axioms (A) give
    \begin{align*} y = 0+y &= (-x+x)+y = -x+(x+y) \\ &= -x+(x+z) =
      (-x+x)+z = 0+z = z.
    \end{align*}
    This proves (a). Take $z=0$ in (a) to obtain (b). Take $z=-x$ in
    (a) to obtain (c).
    Since $-x+x=0$, (c) (with x in place of x) gives (d).
  \end{proof}
\end{proposition}

\begin{proposition}[Consequences of Multiplication Axioms] % Proposition 1.15
  \label{prop:chap1:multiplication_axiom_consequences}
  The axioms for multiplication imply the following statements.
  (a) If $x \ne 0$ and $xy = xz$ then $y=z$.
  (b) If $x \ne 0$ and $xy = x$ then $y=1$.
  (c) If $x \ne 0$ and $xy = 1$ then $y=1/x$.
  (d) If $x \ne 0$ then $1/(1/x) = x$.
  (The proof is so similar to that of Proposition 1.14 that we omit it.)
\end{proposition}

\begin{proposition}[Consequences of Field Axioms] % Proposition 1.16
  \label{prop:chap1:field_axiom_consequences}
  The field axioms imply the following statements, for any x, y, $z \in F$.
  (a) $0x = 0$.
  (b) If $x \ne 0$ and $y \ne 0$ then $xy \ne 0$.
  (c) $(-x)y = -(xy) = x(-y)$.
  (d) $(-x)(-y) = xy$.
  \begin{proof}
    $0x + 0x = (0+0)x = 0x$. Hence 1.14(b) implies that $0x=0$, and (a) holds.
    Next, assume $x \ne 0$, $y \ne 0$, but $xy=0$. Then (a) gives
    \[ 1 = (\frac{1}{y})(\frac{1}{x})xy = (\frac{1}{y})(\frac{1}{x})0 = 0, \]
    a contradiction. Thus (b) holds.
    The first equality in (c) comes from
    \[ (-x)y + xy = (-x+x)y = 0y = 0, \]
    combined with 1.14(c); the other half of (c) is proved in the
    same way. Finally,
    \[ (-x)(-y) = -[x(-y)] = -[-(xy)] = xy \]
    by (c) and 1.14(d).
  \end{proof}
\end{proposition}

% --- End of regenerated chunk (1.9 to 1.16) ---

% --- Content from Def 1.17 to Def 1.23 to append ---

\begin{definition}[Ordered Field] % Definition 1.17
  \label{def:chap1:ordered_field}
  An ordered field is a field F which is also an ordered set, such that
  (i) $x+y < x+z$ if x, y, $z \in F$ and $y<z$.
  (ii) $xy > 0$ if $x \in F$, $y \in F$, $x>0$, and $y>0$.
  If $x>0$, we call x positive; if $x<0$, x is negative.
  For example, $\mathbb{Q}$ is an ordered field.
  All the familiar rules for working with inequalities apply in every
  ordered field: Multiplication by positive [negative] quantities
  preserves [reverses] inequalities, no square is negative, etc. The
  following proposition lists some of these.
\end{definition}

\begin{proposition}[Properties of Ordered Fields] % Proposition 1.18
  \label{prop:chap1:ordered_field_properties}
  The following statements are true in every ordered field.
  (a) If $x>0$ then $-x<0$, and vice versa.
  (b) If $x>0$ and $y<z$ then $xy<xz$.
  (c) If $x<0$ and $y<z$ then $xy>xz$.
  (d) If $x \ne 0$ then $x^{2}>0$. In particular, $1 > 0$.
  (e) If $0<x<y$ then $0<1/y<1/x$.
\end{proposition}

\begin{theorem}[Existence of R] % Theorem 1.19
  \label{thm:chap1:existence_of_R}
  There exists an ordered field $\R$ which has the least-upper-bound property.
  Moreover, $\R$ contains $\mathbb{Q}$ as a subfield.
  (The second statement means that $\mathbb{Q} \subset \R$ and that
    the operations of addition and multiplication in $\R$, when
    applied to members of $\mathbb{Q}$, coincide with the usual
    operations on rational numbers; also, the positive rational
    numbers are positive elements of $\R$.
    The members of $\R$ are called real numbers.
    The proof of Theorem 1.19 is rather long and a bit tedious and is
    therefore presented in an Appendix to Chap. 1. The proof actually
  constructs $\R$ from $\mathbb{Q}$.)
\end{theorem}

\begin{theorem}[Archimedean Property and Density] % Theorem 1.20
  \label{thm:chap1:archimedean_density}
  (a) If $x \in \R$, $y \in \R$, and $x>0$, then there is a positive
  integer n such that $nx > y$.
  (b) If $x \in \R$, $y \in \R$, and $x<y$, then there exists a $p
  \in \mathbb{Q}$ such that $x<p<y$.
  Part (a) is usually referred to as the archimedean property of
  $\R$. Part (b) may be stated by saying that $\mathbb{Q}$ is dense
  in $\R$: Between any two real numbers there is a rational one.
\end{theorem}

\begin{theorem}[Existence of nth Roots] % Theorem 1.21
  \label{thm:chap1:existence_of_nth_roots}
  For every real $x>0$ and every integer $n>0$ there is one and only
  one positive real y such that $y^n = x$.
  This number y is written $\sqrt[n]{x}$ or $x^{1/n}$.
\end{theorem}

\begin{corollary}[Product of nth Roots] % Corollary to 1.21
  \label{cor:chap1:product_of_nth_roots}
  If a and b are positive real numbers and n is a positive integer, then
  \[ (ab)^{1/n} = a^{1/n}b^{1/n}. \]
\end{corollary}

% Skipping 1.22 Decimals (descriptive text)

\begin{definition}[Extended Real Number System] % Definition 1.23
  \label{def:chap1:extended_real_numbers}
  The extended real number system consists of the real field $\R$ and
  two symbols, $+\infty$ and $-\infty$. We preserve the original
  order in $\R$, and define
  \[ -\infty < x < +\infty \]
  for every $x \in \R$.
  It is then clear that $+\infty$ is an upper bound of every subset
  of the extended real number system, and that every nonempty subset
  has a least upper bound. If, for example, E is a nonempty set of
  real numbers which is not bounded above in $\R$, then $\sup E =
  +\infty$ in the extended real number system.
  Exactly the same remarks apply to lower bounds.
  The extended real number system does not form a field, but it is
  customary to make the following conventions:
  (a) If x is real then
  \[ x + \infty = +\infty, \quad x - \infty = -\infty, \quad
  \frac{x}{+\infty} = \frac{x}{-\infty} = 0. \]
  (b) If $x > 0$ then $x \cdot (+\infty) = +\infty$, $x \cdot
  (-\infty) = -\infty$.
  (c) If $x < 0$ then $x \cdot (+\infty) = -\infty$, $x \cdot
  (-\infty) = +\infty$.
  When it is desired to make the distinction between real numbers on
  the one hand and the symbols $+\infty$ and $-\infty$ on the other
  quite explicit, the former are called finite.
\end{definition}

% --- End of content chunk ---

% --- Content from Def 1.24 to Thm 1.35 to append ---

\begin{definition}[Complex Number Operations] % Definition 1.24
  \label{def:chap1:complex_number_operations}
  A complex number is an ordered pair (a, b) of real numbers.
  "Ordered" means that (a, b) and (b, a) are regarded as distinct if $a \ne b$.
  Let $x = (a, b)$, $y = (c, d)$ be two complex numbers. We write
  $x=y$ if and only if $a=c$ and $b=d$. (Note that this definition is
    not entirely superfluous; think of equality of rational numbers,
  represented as quotients of integers.) We define
  \begin{align*}
    x+y &= (a+c, b+d), \\
    xy &= (ac-bd, ad+bc).
  \end{align*}
\end{definition}

\begin{theorem}[Complex Field] % Theorem 1.25
  \label{thm:chap1:complex_field}
  These definitions of addition and multiplication turn the set of
  all complex numbers into a field, with (0, 0) and (1, 0) in the
  role of $0$ and $1$.
\end{theorem}

\begin{theorem}[Real Subfield of Complex] % Theorem 1.26
  \label{thm:chap1:real_subfield_of_complex}
  For any real numbers a and b we have
  \[ (a, 0) + (b, 0) = (a+b, 0), \]
  \[ (a, 0)(b, 0) = (ab, 0). \]
  (The proof is trivial.)
  Theorem 1.26 shows that the complex numbers of the form (a, 0) have
  the same arithmetic properties as the corresponding real numbers a.
  We can therefore identify (a, 0) with a. This identification gives
  us the real field as a subfield of the complex field.
\end{theorem}

\begin{definition}[Definition of i] % Definition 1.27
  \label{def:chap1:i_definition}
  $i = (0, 1)$.
\end{definition}

\begin{theorem}[i squared] % Theorem 1.28
  \label{thm:chap1:i_squared}
  $i^2 = -1$.
\end{theorem}

\begin{theorem}[a+bi form] % Theorem 1.29
  \label{thm:chap1:a_plus_bi_form}
  If a and b are real, then $(a, b) = a+bi$.
\end{theorem}

\begin{definition}[Conjugate, Real, Imaginary Parts] % Definition 1.30
  \label{def:chap1:conjugate_real_imaginary}
  If a, b are real and $z=a+bi$, then the complex number $\bar{z} =
  a-bi$ is called the conjugate of z. The numbers a and b are the
  real part and the imaginary part of z, respectively.
  We shall occasionally write
  \[ a = \re(z), \quad b = \im(z). \]
\end{definition}

\begin{theorem}[Properties of Conjugates] % Theorem 1.31
  \label{thm:chap1:conjugate_properties}
  If z and w are complex, then
  (a) $\overline{z+w} = \bar{z} + \bar{w}$.
  (b) $\overline{zw} = \bar{z} \cdot \bar{w}$.
  (c) $z + \bar{z} = 2 \, \re(z)$, $z - \bar{z} = 2i \, \im(z)$.
  (d) $z\bar{z}$ is real and positive (except when $z=0$).
\end{theorem}

\begin{definition}[Complex Absolute Value] % Definition 1.32
  \label{def:chap1:complex_absolute_value}
  If z is a complex number, its absolute value $\abs{z}$ is the
  non-negative square root of $z\bar{z}$; that is, $\abs{z} = (z\bar{z})^{1/2}$.
  (The existence (and uniqueness) of $\abs{z}$ follows from Theorem
    1.21 and part (d) of Theorem 1.31. Note that when x is real, then
    $\bar{x}=x$, hence $\abs{x} = \sqrt{x^2}$. Thus $\abs{x}=x$ if $x
  \ge 0$, $\abs{x} = -x$ if $x < 0$.)
\end{definition}

\begin{theorem}[Properties of Absolute Value] % Theorem 1.33
  \label{thm:chap1:absolute_value_properties}
  Let z and w be complex numbers. Then
  (a) $\abs{z} > 0$ unless $z=0$; $\abs{0}=0$.
  (b) $|\bar{z}| = \abs{z}$.
  (c) $\abs{zw} = \abs{z}\abs{w}$.
  (d) $\abs{\re(z)} \le \abs{z}$.
  (e) $\abs{z+w} \le \abs{z} + \abs{w}$. (Triangle Inequality)
\end{theorem}

% Skipping 1.34 Notation (descriptive text)

\begin{theorem}[Schwarz Inequality] % Theorem 1.35
  \label{thm:chap1:schwarz_inequality}
  If $a_1, \dots, a_n$ and $b_1, \dots, b_n$ are complex numbers, then
  \[ \abs{\sum_{j=1}^{n} a_j \bar{b}_j}^2 \le \sum_{j=1}^{n}
  \abs{a_j}^2 \sum_{j=1}^{n} \abs{b_j}^2. \]
\end{theorem}

% --- End of content chunk ---

% --- Content from Def 1.36 to Remarks 1.38 to append ---

\begin{definition}[Euclidean k-Space] % Definition 1.36
  \label{def:chap1:euclidean_k_space}
  For each positive integer k, let $\R^k$ be the set of all ordered k-tuples
  \[ \vect{x} = (x_1, x_2, \dots, x_k), \]
  where $x_1, \dots, x_k$ are real numbers, called the coordinates of
  $\vect{x}$. The elements of $\R^k$ are called points, or vectors,
  especially when $k>1$. We shall denote vectors by boldfaced
  letters. If $\vect{y} = (y_1, \dots, y_k)$ and if $\alpha$ is a
  real number, put
  \begin{align*}
    \vect{x} + \vect{y} &= (x_1 + y_1, \dots, x_k + y_k), \\
    \alpha \vect{x} &= (\alpha x_1, \dots, \alpha x_k)
  \end{align*}
  so that $\vect{x} + \vect{y} \in \R^k$ and $\alpha \vect{x} \in
  \R^k$. This defines addition of vectors, as well as multiplication
  of a vector by a real number (a scalar). These two operations
  satisfy the commutative, associative, and distributive laws (the
    proof is trivial, in view of the analogous laws for the real
  numbers) and make $\R^k$ into a vector space over the real field.
  The zero element of $\R^k$ (sometimes called the origin or the null
  vector) is the point $\vect{0}$, all of whose coordinates are 0.

  We also define the so-called "inner product" (or scalar product) of
  $\vect{x}$ and $\vect{y}$ by
  \[ \vect{x} \cdot \vect{y} = \sum_{i=1}^{k} x_i y_i \]
  and the norm of $\vect{x}$ by
  \[ \norm{\vect{x}} = (\vect{x} \cdot \vect{x})^{1/2} = \left(
  \sum_{i=1}^{k} x_i^2 \right)^{1/2}. \]
  The structure now defined (the vector space $\R^k$ with the above
  inner product and norm) is called euclidean k-space.
\end{definition}

\begin{theorem}[Properties of Norm and Inner Product] % Theorem 1.37
  \label{thm:chap1:norm_inner_product_properties}
  Suppose $\vect{x}, \vect{y}, \vect{z} \in \R^k$ and $\alpha$ is real. Then
  (a) $\norm{\vect{x}} \ge 0$.
  (b) $\norm{\vect{x}} = 0$ if and only if $\vect{x} = \vect{0}$.
  (c) $|\alpha \vect{x}| = |\alpha| \norm{\vect{x}}$.
  (d) $|\vect{x} \cdot \vect{y}| \le \norm{\vect{x}}
  \norm{\vect{y}}$. (Schwarz Inequality)
  (e) $|\vect{x} + \vect{y}| \le \norm{\vect{x}} + \norm{\vect{y}}$.
  (Triangle Inequality)
  (f) $|\vect{x} - \vect{z}| \le |\vect{x} - \vect{y}| + |\vect{y} - \vect{z}|$.
\end{theorem}

\begin{remark}[Metric Space, R1, R2] % Remark 1.38
  \label{rem:chap1:metric_space_R1_R2}
  Theorem 1.37 (a), (b), and (f) will allow us (see Chap. 2) to
  regard $\R^k$ as a metric space.
  $\R^1$ (the set of all real numbers) is usually called the line, or
  the real line. Likewise, $\R^2$ is called the plane, or the complex
  plane (compare Definitions 1.24 and 1.36). In these two cases the
  norm is just the absolute value of the corresponding real or complex number.
\end{remark}

% --- End of Chapter 1 content ---
% --- End of chapters/chapter1.tex ---

% --- End of chapters/chapter1.tex ---
