% --- chapters/chapter10.tex ---
% Generated by Gemini (Google AI) on 2025-04-07.
% Contains extracted items from W. Rudin, PMA, Chapter 10.
% Assumes \supp, \dV, \dA, \ds macros are defined in macros.tex

\chapter{Integration of Differential Forms}
\label{chap:rudin10}

% Intro text condensed
This chapter develops aspects of integration theory related to the
geometry of euclidean spaces, including change of variables, line
integrals, differential forms, and Stokes' theorem (the n-dimensional
analogue of the fundamental theorem of calculus).

\section{Integration}

\begin{definition}[Iterated Integral over k-cell]
  \label{def:chap10:iterated_integral}
  Suppose $I^k$ is a k-cell in $\R^k$, consisting of all
  $\vect{x}=(x_1, \dots, x_k)$ such that $a_i \le x_i \le b_i$ for
  $i=1, \dots, k$. Let $I^j$ be the j-cell in $\R^j$ defined by the
  first j inequalities. Let f be a real continuous function on $I^k$.
  Put $f=f_k$, and define $f_{j-1}$ on $I^{j-1}$ recursively by
  \[
    f_{j-1}(x_1, \dots, x_{j-1}) = \int_{a_j}^{b_j} f_j(x_1, \dots,
    x_{j-1}, x_j) dx_j \quad (j=k, k-1, \dots, 1).
  \]
  Each $f_j$ is continuous on $I^j$ (since $f_{j+1}$ is uniformly
  continuous). After k steps we arrive at a number $f_0$, which we
  call the \textbf{integral} of f over $I^k$, denoted by
  \[
    \int_{I^k} f(\vect{x}) d\vect{x} \quad \text{or} \quad \int_{I^k} f.
  \]
\end{definition}

% A priori, this definition depends on the order of integration.

\begin{theorem}[Independence of Order of Integration]
  \label{thm:chap10:integration_order_independent}
  For every $f \in \mathcal{C}(I^k)$, the integral $\int_{I^k} f$
  defined in Def \ref{def:chap10:iterated_integral} is independent of
  the order in which the k integrations are carried out.
\end{theorem}
% Proof omitted. (Proof uses approximation by sums of products of
% 1-variable functions and Stone-Weierstrass theorem).

\begin{definition}[Integral over $\R^k$]
  \label{def:chap10:integral_Rk_compact_support}
  The \textbf{support} of a (real or complex) function f on $\R^k$ is
  the closure of the set $\{ \vect{x} \in \R^k \mid f(\vect{x}) \ne 0 \}$.
  If f is a continuous function with compact support, let $I^k$ be
  any k-cell which contains $\supp f$, and define
  \[
    \int_{\R^k} f = \int_{I^k} f.
  \]
  This definition is independent of the choice of $I^k$ (as long as
  it contains $\supp f$, since f is zero outside its support).
\end{definition}

% Remark on extending definition via limits (Lebesgue integral)
% omitted for conciseness.

\begin{example}[Integral over Standard k-simplex]
  \label{ex:chap10:integral_simplex}
  Let $Q^k$ be the \textbf{standard k-simplex}, consisting of all
  $\vect{x}=(x_1, \dots, x_k)$ in $\R^k$ for which $x_1 + \dots + x_k
  \le 1$ and $x_i \ge 0$ for $i=1, \dots, k$.
  If $f \in \mathcal{C}(Q^k)$, extend f to a function $\tilde{f}$ on
  the unit k-cell $I^k = [0, 1]^k$ by setting $\tilde{f}(\vect{x}) =
  0$ for $\vect{x} \in I^k \setminus Q^k$. Define
  \[
    \int_{Q^k} f = \int_{I^k} \tilde{f}.
  \]
  This integral exists and is independent of the order of
  integration. (This can be shown by approximating $\tilde{f}$ with
  continuous functions on $I^k$.)
\end{example}

% --- End of transcription chunk ---

% --- Previous content (up to Ex 10.4) from chapters/chapter10.tex above ---

\section{Primitive Mappings}

\begin{definition}[Primitive Mapping]
  \label{def:chap10:primitive_mapping}
  If G maps an open set $E \subset \R^n$ into $\R^n$ and if there is
  an integer $m \in \{1, \dots, n\}$ and a real function
  $\map{g}{E}{\R}$ such that
  \[
    G(\vect{x}) = \sum_{i \ne m} x_i \vect{e}_i + g(\vect{x})
    \vect{e}_m = \vect{x} + [g(\vect{x}) - x_m] \vect{e}_m \quad
    (\vect{x} \in E),
  \]
  then G is called \textbf{primitive}. (It changes at most the m-th coordinate).
  If $g \in \mathcal{C}'(E)$, then $G \in \mathcal{C}'(E)$, and its
  Jacobian determinant at $\vect{a} \in E$ is $J_G(\vect{a}) = (D_m
  g)(\vect{a})$. Thus $G'(\vect{a})$ is invertible if and only if
  $(D_m g)(\vect{a}) \ne 0$.
\end{definition}

\begin{definition}[Flip]
  \label{def:chap10:flip}
  A linear operator B on $\R^n$ that interchanges some pair of
  members of the standard basis $\{ \vect{e}_1, \dots, \vect{e}_n \}$
  and leaves the others fixed is called a \textbf{flip}.
  (Equivalently, it interchanges two coordinates).
\end{definition}

% Define Projections P_m needed for Theorem 10.7
\begin{definition}[Standard Projections in $\R^n$]
  \label{def:chap10:standard_projections}
  The projections $P_0, \dots, P_n$ in $\R^n$ are defined by $P_0
  \vect{x} = \vect{0}$ and
  \[
    P_m \vect{x} = x_1 \vect{e}_1 + \dots + x_m \vect{e}_m \quad (1
    \le m \le n).
  \]
\end{definition}

\begin{theorem}[Local Representation by Primitive Mappings]
  \label{thm:chap10:local_primitive_representation}
  Suppose F is a $\mathcal{C}'$-mapping of an open set $E \subset
  \R^n$ into $\R^n$, $\vect{0} \in E$, $F(\vect{0}) = \vect{0}$, and
  $F'(\vect{0})$ is invertible.
  Then there is a neighborhood of $\vect{0}$ in $\R^n$ in which F has
  a representation
  \[
    F(\vect{x}) = B_1 \cdots B_{n-1} G_n \circ \dots \circ G_1 (\vect{x}).
  \]
  Here, each $G_i$ is a primitive $\mathcal{C}'$-mapping in some
  neighborhood of $\vect{0}$, $G_i(\vect{0})=\vect{0}$,
  $G_i'(\vect{0})$ is invertible, and each $B_i$ is either a flip or
  the identity operator $\id$.
  (Briefly, F is locally a composition of primitive mappings and flips).
\end{theorem}
% Proof omitted. (Proof uses induction, applying the Inverse Function
% Theorem to intermediate mappings).

\section{Partitions of Unity}

\begin{theorem}[Existence of Partition of Unity]
  \label{thm:chap10:partition_of_unity}
  Suppose K is a compact subset of $\R^n$, and $\{ V_\alpha \}$ is an
  open cover of K. Then there exist functions $\psi_1, \dots, \psi_s
  \in \mathcal{C}(\R^n)$ such that
  \begin{enumerate}
    \item[(a)] $0 \le \psi_i(\vect{x}) \le 1$ for $1 \le i \le s$ and
      all $\vect{x} \in \R^n$;
    \item[(b)] each $\psi_i$ has its support in some $V_\alpha$; and
    \item[(c)] $\psi_1(\vect{x}) + \dots + \psi_s(\vect{x}) = 1$ for
      every $\vect{x} \in K$.
  \end{enumerate}
  (The collection $\{ \psi_i \}$ is called a \textbf{partition of
  unity} subordinate to the cover $\{ V_\alpha \}$.)
\end{theorem}
% Proof omitted. (Proof constructs functions $\varphi_i$ that are 1
% on smaller balls B(x_i) and 0 outside larger balls W(x_i) contained
% in some V_\alpha(x_i), then defines $\psi_i$ iteratively).

\begin{corollary}
  \label{cor:chap10:partition_unity_application}
  If $f \in \mathcal{C}(\R^n)$ and $\supp f \subset K$, then $f =
  \sum_{i=1}^s \psi_i f$. Each function $\psi_i f$ is continuous and
  has its support in some $V_\alpha$. (This allows representing f as
  a sum of functions with "small" supports).
\end{corollary}

% --- End of transcription chunk ---

% --- Previous content (up to Rem 9.32) from chapters/chapter10.tex above ---

\section{Change of Variables}

\begin{theorem}[Change of Variables Formula]
  \label{thm:chap10:change_of_variables}
  Suppose T is a 1-1 $\mathcal{C}'$-mapping of an open set $E \subset
  \R^k$ into $\R^k$ such that the Jacobian $J_T(\vect{x}) = \det
  T'(\vect{x}) \ne 0$ for all $\vect{x} \in E$. If f is a continuous
  function on $\R^k$ whose support is compact and lies in $T(E)$, then
  \[
    \int_{\R^k} f(\vect{y}) d\vect{y} = \int_{\R^k} f(T(\vect{x}))
    |J_T(\vect{x})| d\vect{x}.
  \]
  (The integral on the left is over $\R^k$ w.r.t the variable
  $\vect{y}$; the integral on the right is w.r.t $\vect{x}$).
\end{theorem}
% Proof omitted. (Proof relies on local representation Thm 10.7,
% properties for primitive maps/flips, and partition of unity Thm 10.8).

\begin{remark}[Absolute Value of Jacobian]
  The absolute value $|J_T(\vect{x})|$ appears because the integral
  $\int_{\R^k} f(\vect{y}) d\vect{y}$ is defined without orientation.
  For $k=1$, $\int f(y) dy = \int f(T(x)) |T'(x)| dx$. If $T'(x) < 0$
  (T is decreasing), the absolute value ensures the formula holds for
  positive f.
\end{remark}

\section{Differential Forms}

% Introductory text condensed.
The machinery of differential forms is developed for the
n-dimensional version of the fundamental theorem of calculus (Stokes' theorem).

% Convention on differentiability on compact sets.
\begin{remark}[Convention for $\mathcal{C}'$/$\mathcal{C}''$ on Compact Sets]
  \label{rem:chap10:diff_on_compact}
  To say that f is a $\mathcal{C}'$-mapping (or $\mathcal{C}''$) of a
  compact set $D \subset \R^k$ into $\R^n$ means that there exists a
  $\mathcal{C}'$-mapping (or $\mathcal{C}''$) g of an open set $W
  \subset \R^k$ into $\R^n$ such that $D \subset W$ and $g(\vect{x})
  = f(\vect{x})$ for all $\vect{x} \in D$.
\end{remark}

\begin{definition}[k-surface]
  \label{def:chap10:k-surface}
  Suppose E is an open set in $\R^n$. A \textbf{k-surface} in E is a
  $\mathcal{C}'$-mapping $\Phi$ from a compact set $D \subset \R^k$
  into E. D is called the \textbf{parameter domain} of $\Phi$. Points
  of D will be denoted by $\vect{u} = (u_1, \dots, u_k)$.
  We restrict D to be either a k-cell or the standard k-simplex
  $Q^k$. (A k-surface is a mapping, not necessarily a subset of E).
\end{definition}

\begin{definition}[Differential Form, Integration]
  \label{def:chap10:k-form_definition}
  Suppose E is an open set in $\R^n$. A \textbf{differential form of
  order k} $\ge 1$ (or \textbf{k-form}) in E is a function $\omega$,
  symbolically represented by the sum
  \[
    \omega = \sum a_{i_1 \dots i_k}(\vect{x}) dx_{i_1} \wedge \dots
    \wedge dx_{i_k}
  \]
  (the indices $i_1, \dots, i_k$ range independently from 1 to n),
  whose coefficients $a_{i_1 \dots i_k}$ are real continuous functions in E.
  $\omega$ assigns to each k-surface $\Phi$ in E (with parameter
  domain D) a number $\int_\Phi \omega$, defined by the rule
  \[
    \int_\Phi \omega = \int_D \sum a_{i_1 \dots i_k}(\Phi(\vect{u}))
    \frac{\partial(x_{i_1}, \dots, x_{i_k})}{\partial(u_1, \dots,
    u_k)} d\vect{u}.
  \]
  Here $(x_{i_1}, \dots, x_{i_k})$ are the components of $\vect{x} =
  \Phi(\vect{u})$, and the Jacobian is determined by the mapping
  $\vect{u} \mapsto ( \phi_{i_1}(\vect{u}), \dots,
  \phi_{i_k}(\vect{u}) )$, where $\phi_1, \dots, \phi_n$ are the
  components of $\Phi$. The integral over D is as in Def
  \ref{def:chap10:iterated_integral} or Ex \ref{ex:chap10:integral_simplex}.

  A k-form $\omega$ is of class $\mathcal{C}'$ or $\mathcal{C}''$ if
  its coefficient functions $a_{i_1 \dots i_k}$ are.
  A \textbf{0-form} in E is defined to be a continuous function in E.
\end{definition}

% --- End of transcription chunk ---

% --- Previous content (up to Def 10.11) from chapters/chapter10.tex above ---

\begin{example}[Integrals of Forms]
  \label{ex:chap10:form_integral_examples}
  ~ % Using itemize for parts (a) through (c)
  \begin{itemize}
    \item[(a)] Let $\gamma: [0, 1] \to \R^3$ be a
      $\mathcal{C}'$-curve (1-surface). Let $\omega = x dy + y dx$.
      Then by Definition \ref{def:chap10:k-form_definition},
      \[
        \int_\gamma \omega = \int_0^1 [ \gamma_1(t) \gamma'_2(t) +
        \gamma_2(t) \gamma'_1(t) ] dt = \int_0^1 ( \gamma_1(t)
        \gamma_2(t) )' dt = \gamma_1(1)\gamma_2(1) - \gamma_1(0)\gamma_2(0).
      \]
      The integral depends only on the endpoints. $\int_\gamma \omega
      = 0$ if $\gamma$ is closed. Integrals of 1-forms are called
      \textbf{line integrals}.

    \item[(b)] Let $\gamma(t) = (a \cos t, b \sin t)$ for $0 \le t
      \le 2\pi$ (ellipse in $\R^2$, $a>0, b>0$). Then
      \begin{align*}
        \int_\gamma x dy &= \int_0^{2\pi} (a \cos t) (b \cos t) dt =
        ab \int_0^{2\pi} \cos^2 t dt = \pi ab, \\
        \int_\gamma y dx &= \int_0^{2\pi} (b \sin t) (-a \sin t) dt =
        -ab \int_0^{2\pi} \sin^2 t dt = -\pi ab.
      \end{align*}
      Note that $\int_\gamma x dy$ is the area enclosed by $\gamma$.
      Also $\frac{1}{2} \int_\gamma (x dy - y dx) = \pi ab$.

    \item[(c)] Let D be the 3-cell $0 \le r \le 1, 0 \le \theta \le
      \pi, 0 \le \varphi \le 2\pi$. Define $\Phi(r, \theta, \varphi)
      = (x, y, z)$ by $x = r \sin\theta \cos\varphi$, $y = r
      \sin\theta \sin\varphi$, $z = r \cos\theta$. $\Phi$ maps D onto
      the unit ball in $\R^3$. The Jacobian is $J_\Phi(r, \theta,
      \varphi) = \frac{\partial(x,y,z)}{\partial(r,\theta,\varphi)} =
      r^2 \sin\theta$. Let $\omega = dx \wedge dy \wedge dz$. Then
      \[
        \int_\Phi \omega = \int_D J_\Phi dr d\theta d\varphi =
        \int_0^{2\pi} d\varphi \int_0^\pi \sin\theta d\theta \int_0^1
        r^2 dr = (2\pi)(2)(1/3) = \frac{4\pi}{3}.
      \]
      This is the volume of the unit ball $\Phi(D)$.
  \end{itemize}
\end{example}

\begin{remark}[Properties of Forms]
  \label{rem:chap10:form_properties}
  Let $\omega, \omega_1, \omega_2$ be k-forms, $\lambda$ an m-form, c a scalar.
  \begin{itemize}
    \item $\omega_1 = \omega_2$ iff $\int_\Phi \omega_1 = \int_\Phi
      \omega_2$ for every k-surface $\Phi$. $\omega=0$ iff $\int_\Phi
      \omega = 0$ for all $\Phi$.
    \item $c\omega$ and $\omega_1 + \omega_2$ are defined by
      $\int_\Phi c\omega = c \int_\Phi \omega$ and $\int_\Phi
      (\omega_1 + \omega_2) = \int_\Phi \omega_1 + \int_\Phi \omega_2$.
    \item Interchanging two indices in $a(\vect{x}) dx_{i_1} \wedge
      \dots \wedge dx_{i_k}$ multiplies the form by -1.
    \item Consequently: $dx_i \wedge dx_j = - dx_j \wedge dx_i$
      (anticommutativity).
    \item $dx_i \wedge dx_i = 0$.
    \item $dx_{i_1} \wedge \dots \wedge dx_{i_k} = 0$ if any two
      indices $i_r, i_s$ ($r \ne s$) are equal.
    \item Summands in $\omega = \sum a_{i_1 \dots i_k} dx_{i_1}
      \wedge \dots \wedge dx_{i_k}$ with repeated indices can be omitted.
    \item If $k > n$, the only k-form in an open set $E \subset \R^n$
      is $\omega=0$.
  \end{itemize}
\end{remark}

\begin{definition}[Basic Forms and Standard Presentation]
  \label{def:chap10:basic_forms_standard_presentation}
  If $I = \{ i_1, \dots, i_k \}$ is an ordered k-tuple with $1 \le
  i_1 < i_2 < \dots < i_k \le n$, we call I an \textbf{increasing
  k-index}. The corresponding \textbf{basic k-form} is $dx_I =
  dx_{i_1} \wedge \dots \wedge dx_{i_k}$.

  Any k-form $\omega$ can be written uniquely in the \textbf{standard
  presentation}
  \[
    \omega = \sum_I b_I(\vect{x}) dx_I,
  \]
  where the summation extends over all increasing k-indices I. (This
    follows from the anticommutativity properties in Remark
    \ref{rem:chap10:form_properties}, which allow any $dx_{j_1} \wedge
  \dots \wedge dx_{j_k}$ to be rewritten as $\pm dx_I$ or 0).
\end{definition}

\begin{theorem}[Uniqueness of Standard Presentation]
  \label{thm:chap10:uniqueness_standard_presentation}
  Suppose $\omega = \sum_I b_I(\vect{x}) dx_I$ is the standard
  presentation of a k-form $\omega$ in an open set $E \subset \R^n$.
  If $\omega=0$ in E, then $b_I(\vect{x})=0$ for every increasing
  k-index I and for every $\vect{x} \in E$.
\end{theorem}
% Proof omitted. (Proof involves constructing a specific k-surface
% $\Phi$ for which $\int_\Phi \omega \ne 0$ if some $b_J(v) \ne 0$).

% --- End of transcription chunk ---

% --- Previous content (up to Thm 10.15) from chapters/chapter10.tex above ---

\begin{definition}[Product of Basic Forms]
  \label{def:chap10:product_basic_forms}
  Let $I = \{ i_1, \dots, i_p \}$ and $J = \{ j_1, \dots, j_q \}$ be
  increasing indices. The product of the basic forms $dx_I$ and
  $dx_J$ is the $(p+q)$-form $dx_I \wedge dx_J$ defined by
  \[
    dx_I \wedge dx_J = dx_{i_1} \wedge \dots \wedge dx_{i_p} \wedge
    dx_{j_1} \wedge \dots \wedge dx_{j_q}.
  \]
  If $I \cap J \ne \emptyset$, then $dx_I \wedge dx_J = 0$.
  If $I \cap J = \emptyset$, let $[I, J]$ be the increasing
  $(p+q)$-index obtained by ordering $I \cup J$. Then
  \[
    dx_I \wedge dx_J = (-1)^\alpha dx_{[I, J]},
  \]
  where $\alpha$ is the number of pairs $(s, t)$ such that $i_s >
  j_t$. (This ensures the terms are ordered correctly based on the
  anticommutativity $dx_a \wedge dx_b = - dx_b \wedge dx_a$).
  The associative law holds for basic forms: $(dx_I \wedge dx_J)
  \wedge dx_K = dx_I \wedge (dx_J \wedge dx_K)$.
\end{definition}

\begin{definition}[Product of Forms]
  \label{def:chap10:product_forms}
  Suppose $\omega = \sum_I b_I(\vect{x}) dx_I$ is a p-form and
  $\lambda = \sum_J c_J(\vect{x}) dx_J$ is a q-form (standard
  presentations). Their product $\omega \wedge \lambda$ is the
  $(p+q)$-form defined by
  \[
    \omega \wedge \lambda = \sum_{I, J} b_I(\vect{x}) c_J(\vect{x})
    dx_I \wedge dx_J.
  \]
  This multiplication is distributive over addition and is
  associative: $(\omega \wedge \lambda) \wedge \sigma = \omega \wedge
  (\lambda \wedge \sigma)$.
  If f is a 0-form, the product $f \wedge \omega$ is usually written
  $f \omega = \omega f = \sum_I f(\vect{x}) b_I(\vect{x}) dx_I$.
  The product of a k-form $\omega$ and an m-form $\lambda$ also
  satisfies $\omega \wedge \lambda = (-1)^{km} \lambda \wedge \omega$.
\end{definition}

\begin{definition}[Exterior Derivative]
  \label{def:chap10:exterior_derivative}
  Let $\omega$ be a k-form of class $\mathcal{C}'$ in an open set $E
  \subset \R^n$.
  If $k=0$, $\omega = f$ is a function $f \in \mathcal{C}'(E)$, and
  its \textbf{exterior derivative} $df$ is the 1-form
  \[
    df = \sum_{i=1}^n (D_i f)(\vect{x}) dx_i.
  \]
  If $k \ge 1$ and $\omega = \sum_I b_I(\vect{x}) dx_I$ is the
  standard presentation, where $b_I \in \mathcal{C}'(E)$, then
  $d\omega$ is the $(k+1)$-form defined by
  \[
    d\omega = \sum_I (db_I) \wedge dx_I.
  \]
\end{definition}

\begin{example}[Integral of Exact 1-form]
  \label{ex:chap10:integral_exact_1form}
  If $f \in \mathcal{C}'(E)$ and $\gamma: [0, 1] \to E$ is a
  $\mathcal{C}'$-curve, then $df = \sum (D_i f) dx_i$. Using Def
  \ref{def:chap10:k-form_definition} and the chain rule:
  \[
    \int_\gamma df = \int_0^1 \sum_{i=1}^n (D_i f)(\gamma(t))
    \gamma'_i(t) dt = \int_0^1 (f \circ \gamma)'(t) dt = f(\gamma(1))
    - f(\gamma(0)).
  \]
  The integral of an exact 1-form $df$ depends only on the endpoints
  of the curve $\gamma$. Therefore $\int_\gamma df = 0$ for any
  closed curve $\gamma$. Since $\int_\gamma x dy \ne 0$ for the
  ellipse in Example \ref{ex:chap10:form_integral_examples}(b), $x
  dy$ is not an exact form (not the derivative of any 0-form f).
\end{example}

\begin{theorem}[Properties of Exterior Derivative]
  \label{thm:chap10:properties_exterior_derivative}
  ~ % Using itemize for parts (a) and (b)
  \begin{enumerate}
    \item[(a)] (Leibniz Rule) If $\omega$ is a k-form and $\lambda$
      is an m-form, both of class $\mathcal{C}'$ in E, then
      \[
        d(\omega \wedge \lambda) = (d\omega) \wedge \lambda + (-1)^k
        \omega \wedge d\lambda.
      \]
    \item[(b)] ($d^2=0$) If $\omega$ is of class $\mathcal{C}''$ in
      E, then $d(d\omega) = 0$, abbreviated as $d^2\omega = 0$.
  \end{enumerate}
\end{theorem}
% Proof omitted. (Proof of (a) uses definition for basic forms. Proof
% of (b) uses $d^2f=0$ for 0-forms (from equality of mixed partials)
% and applies (a)).

% --- End of transcription chunk ---

% --- Previous content (up to Thm 10.20) from chapters/chapter10.tex above ---

\subsection*{Change of Variables (Pullback of Forms)}
\label{sec:chap10:pullback_forms}

Suppose E is an open set in $\R^n$, T is a $\mathcal{C}'$-mapping of
E into an open set $V \subset \R^m$. Let $\omega$ be a k-form in V,
with standard presentation $\omega = \sum_I b_I(\vect{y}) dy_I$. Let
$t_1, \dots, t_m$ be the components of T, so $y_i = t_i(\vect{x})$.
Define the 1-forms $dt_i$ in E by $dt_i = \sum_{j=1}^n (D_j
t_i)(\vect{x}) dx_j$.

The mapping T transforms $\omega$ into a k-form $\omega_T$ in E,
called the \textbf{pullback} of $\omega$ by T, defined by
\[
  \omega_T = \sum_I b_I(T(\vect{x})) dt_{i_1} \wedge \dots \wedge dt_{i_k}.
\]
(where $I = \{ i_1, \dots, i_k \}$ is an increasing k-index).

\begin{theorem}[Properties of Pullback]
  \label{thm:chap10:pullback_properties}
  With E, V, T as above, let $\omega$ and $\lambda$ be k- and m-forms
  in V, respectively.
  \begin{enumerate}
    \item[(a)] If $k=m$, $(\omega + \lambda)_T = \omega_T + \lambda_T$.
    \item[(b)] $(\omega \wedge \lambda)_T = \omega_T \wedge \lambda_T$.
    \item[(c)] If $\omega$ is of class $\mathcal{C}'$ and T is of
      class $\mathcal{C}''$, then $d(\omega_T) = (d\omega)_T$. (The
      exterior derivative commutes with pullback).
  \end{enumerate}
\end{theorem}
% Proof omitted. (Proof uses definitions, chain rule for (c) on
% 0-forms, and Leibniz rule).

\begin{theorem}[Pullback of Composition]
  \label{thm:chap10:pullback_composition}
  Suppose T maps $E \subset \R^n$ into $V \subset \R^m$ (class
  $\mathcal{C}'$) and S maps V into $W \subset \R^p$ (class
  $\mathcal{C}'$). Let $\omega$ be a k-form in W. Define $ST: E \to
  W$ by $(ST)(\vect{x}) = S(T(\vect{x}))$. Then
  \[
    (\omega_S)_T = \omega_{ST}.
  \]
\end{theorem}
% Proof omitted. (Proof uses properties from Thm 10.22 and verifies
% for 0-forms and 1-forms).

\begin{theorem}[Integration and Pullback I]
  \label{thm:chap10:integration_pullback1}
  Suppose $\omega$ is a k-form in an open set $E \subset \R^n$,
  $\Phi$ is a k-surface in E with parameter domain $D \subset \R^k$,
  and $\Delta$ is the k-surface in $\R^k$ with parameter domain D
  defined by $\Delta(\vect{u}) = \vect{u}$ for $\vect{u} \in D$. Then
  \[
    \int_\Phi \omega = \int_\Delta \omega_\Phi.
  \]
\end{theorem}
% Proof omitted. (Proof involves showing $d\phi_{i_1} \wedge \dots
% \wedge d\phi_{i_k} = J(u) du_1 \wedge \dots \wedge du_k$, where J
% is the relevant Jacobian).

\begin{theorem}[Integration and Pullback II - Change of Variables for
  Form Integrals]
  \label{thm:chap10:integration_pullback2}
  Suppose T is a $\mathcal{C}'$-mapping of an open set $E \subset
  \R^n$ into an open set $V \subset \R^m$, $\Phi$ is a k-surface in
  E, and $\omega$ is a k-form in V. Then $T\Phi = T \circ \Phi$ is a
  k-surface in V, and
  \[
    \int_{T\Phi} \omega = \int_\Phi \omega_T.
  \]
\end{theorem}
% Proof omitted. (Follows from Thm 10.24 and Thm 10.23).

% --- End of transcription chunk ---

% --- Previous content (up to Thm 10.25) from chapters/chapter10.tex above ---

\section{Simplexes and Chains}

\begin{definition}[Affine Simplex]
  \label{def:chap10:affine_simplex}
  A mapping $\map{f}{X}{Y}$ between vector spaces is \textbf{affine}
  if $f(\vect{x}) = f(\vect{0}) + A\vect{x}$ for some $A \in \mathcal{L}(X, Y)$.

  The \textbf{standard k-simplex} $Q^k$ is the set $\{ \vect{u} =
    \sum_{i=1}^k \alpha_i \vect{e}_i \in \R^k \mid \alpha_i \ge 0, \sum
  \alpha_i \le 1 \}$.

  Given points $\vect{p}_0, \vect{p}_1, \dots, \vect{p}_k \in \R^n$,
  the \textbf{oriented affine k-simplex} $\sigma = [\vect{p}_0,
  \vect{p}_1, \dots, \vect{p}_k]$ is the k-surface with parameter
  domain $Q^k$ given by the affine mapping $\sigma: Q^k \to \R^n$
  uniquely defined by $\sigma(\vect{0}) = \vect{p}_0$ and
  $\sigma(\vect{e}_i) = \vect{p}_i$ for $1 \le i \le k$. Explicitly,
  \[
    \sigma\left( \sum_{i=1}^k \alpha_i \vect{e}_i \right) =
    \vect{p}_0 + \sum_{i=1}^k \alpha_i (\vect{p}_i - \vect{p}_0).
  \]
  If $\{ i_0, \dots, i_k \}$ is a permutation of $\{ 0, \dots, k \}$,
  let $\overline{\sigma} = [\vect{p}_{i_0}, \dots, \vect{p}_{i_k}]$.
  We write $\overline{\sigma} = \epsilon \sigma$, where $\epsilon =
  s(i_0, \dots, i_k) = \pm 1$ is the sign of the permutation. If
  $\epsilon=1$, $\overline{\sigma}$ and $\sigma$ have the
  \textbf{same orientation}; if $\epsilon=-1$, they have
  \textbf{opposite orientations}.

  If $n=k$ and the vectors $\vect{p}_i - \vect{p}_0$ ($1 \le i \le
  k$) are independent, then $\sigma$ corresponds to an invertible
  linear map A via $\sigma(\vect{u}) = \vect{p}_0 + A\vect{u}$.
  $\sigma$ is \textbf{positively oriented} if $\det A > 0$ and
  \textbf{negatively oriented} if $\det A < 0$.

  An \textbf{oriented 0-simplex} is $\sigma = \epsilon \vect{p}_0$
  where $\epsilon = \pm 1$. If f is a 0-form, $\int_\sigma f =
  \epsilon f(\vect{p}_0)$.
\end{definition}

\begin{theorem}[Integration over Reordered Simplex]
  \label{thm:chap10:integral_reordered_simplex}
  If $\sigma$ is an oriented affine k-simplex in an open set $E
  \subset \R^n$, $\omega$ is a k-form in E, and $\overline{\sigma} =
  \epsilon \sigma$ (where $\epsilon = \pm 1$), then
  \[
    \int_{\overline{\sigma}} \omega = \epsilon \int_\sigma \omega.
  \]
\end{theorem}
% Proof omitted. (Proof considers interchanges $p_0 \leftrightarrow
% p_j$ and $p_i \leftrightarrow p_j$ and their effect on the
% Jacobians in the integral definition).

\begin{definition}[Affine Chain]
  \label{def:chap10:affine_chain}
  An \textbf{affine k-chain} $\Gamma$ in an open set $E \subset \R^n$
  is a formal sum $\Gamma = \sum_{i=1}^r \sigma_i$ of finitely many
  oriented affine k-simplexes $\sigma_1, \dots, \sigma_r$ in E.
  If $\omega$ is a k-form in E, the integral over $\Gamma$ is defined as
  \[
    \int_\Gamma \omega = \sum_{i=1}^r \int_{\sigma_i} \omega.
  \]
  Note: The sum $\sum \sigma_i$ represents the chain as a functional
  on k-forms. $\sigma_1 + (-\sigma_1) = 0$ means the integral over
  the chain is 0, not that the mappings cancel pointwise.
\end{definition}

\begin{definition}[Boundary of a Simplex]
  \label{def:chap10:boundary_simplex}
  For $k \ge 1$, the \textbf{boundary} of the oriented affine
  k-simplex $\sigma = [\vect{p}_0, \vect{p}_1, \dots, \vect{p}_k]$ is
  defined as the affine $(k-1)$-chain
  \[
    \partial \sigma = \sum_{j=0}^k (-1)^j [\vect{p}_0, \dots,
    \vect{p}_{j-1}, \vect{p}_{j+1}, \dots, \vect{p}_k].
  \]
  (The term for $j=0$ is $[\vect{p}_1, \dots, \vect{p}_k]$.)
  For $k=0$, $\sigma = \epsilon \vect{p}_0$, we define $\partial \sigma = 0$.
\end{definition}

% --- End of transcription chunk ---

% --- Previous content (up to Thm 10.15) from chapters/chapter10.tex above ---

\subsection*{Differentiable Simplexes and Chains}
\label{sec:chap10:diff_simplex_chain}

Let T be a $\mathcal{C}''$-mapping of an open set $E \subset \R^n$
into an open set $V \subset \R^m$.
If $\sigma$ is an oriented affine k-simplex in E, the composite
mapping $\Phi = T \circ \sigma$ is called an \textbf{oriented
k-simplex of class $\mathcal{C}''$} in V (with parameter domain $Q^k$).

A \textbf{k-chain of class $\mathcal{C}''$} in V is a formal sum
$\Psi = \sum_{i=1}^r \Phi_i$ of finitely many oriented k-simplexes
$\Phi_i$ of class $\mathcal{C}''$ in V.
If $\omega$ is a k-form in V, the integral over $\Psi$ is defined as
$\int_\Psi \omega = \sum_{i=1}^r \int_{\Phi_i} \omega$.
If $\Gamma = \sum \sigma_i$ is an affine chain in E and $\Phi_i = T
\circ \sigma_i$, we write $\Psi = T \circ \Gamma = T(\sum \sigma_i) =
\sum T\sigma_i$.

The \textbf{boundary} $\partial \Phi$ of the oriented k-simplex $\Phi
= T \circ \sigma$ (where $\sigma$ is affine) is defined as the
$(k-1)$-chain $\partial \Phi = T(\partial \sigma)$.
The \textbf{boundary} $\partial \Psi$ of the k-chain $\Psi = \sum
\Phi_i$ is defined as the $(k-1)$-chain $\partial \Psi = \sum \partial \Phi_i$.

\begin{remark}[Positively Oriented Boundaries of Sets]
  \label{rem:chap10:pos_oriented_boundary_set}
  Let $\sigma_0$ be the standard oriented n-simplex $[0, \vect{e}_1,
  \dots, \vect{e}_n]$ in $\R^n$. Its boundary $\partial \sigma_0$ is
  the positively oriented boundary of $Q^n$.
  If T is a 1-1 $\mathcal{C}''$-mapping of $Q^n$ into $\R^n$ with
  positive Jacobian $J_T > 0$, let $E = T(Q^n)$. The
  \textbf{positively oriented boundary} of the set E is defined as
  the $(n-1)$-chain $\partial E = T(\partial \sigma_0)$. This
  definition can be extended to sets $\Omega$ which are finite unions
  of such sets $E_i$ with disjoint interiors, by setting $\partial
  \Omega = \sum \partial E_i$. (Internal boundaries often cancel out
  in this sum).
\end{remark}

\begin{example}[Boundary of Sphere Parameterization]
  \label{ex:chap10:sphere_boundary}
  Let $\Sigma(u, v) = (\sin u \cos v, \sin u \sin v, \cos u)$ for $0
  \le u \le \pi, 0 \le v \le 2\pi$. $\Sigma$ is a 2-surface
  (parameter domain D is a rectangle) whose range is the unit sphere
  $S^2 \subset \R^3$. Its boundary chain $\partial \Sigma =
  \Sigma(\partial D)$ consists of four curves. Two correspond to
  $u=0$ and $u=\pi$ (constant maps to poles, integral is 0). The
  other two correspond to $v=0$ and $v=2\pi$ (same meridian traversed
  in opposite directions). Thus $\int_{\partial \Sigma} \omega = 0$
  for any 1-form $\omega$, which we denote as $\partial \Sigma = 0$.
\end{example}

\section{Stokes' Theorem}

\begin{theorem}[Stokes' Theorem]
  \label{thm:chap10:stokes_theorem}
  If $\Psi$ is a k-chain of class $\mathcal{C}''$ in an open set $V
  \subset \R^m$ and if $\omega$ is a $(k-1)$-form of class
  $\mathcal{C}'$ in V, then
  \[
    \int_\Psi d\omega = \int_{\partial \Psi} \omega.
  \]
\end{theorem}
% Proof omitted. (Proof reduces to the case $\Psi = T \circ \sigma$
% where $\sigma$ is the standard affine k-simplex $[0, e_1, ...,
% e_k]$. Using pullback properties, this reduces to proving
% $\int_\sigma d\lambda = \int_{\partial \sigma} \lambda$ for a
% (k-1)-form $\lambda$. This is verified by considering basic forms
% and using the fundamental theorem of calculus for the iterated integrals).

% --- End of transcription chunk ---

% --- Previous content (up to Thm 10.25) from chapters/chapter10.tex above ---

\section{Closed Forms and Exact Forms}

\begin{definition}[Exact and Closed Forms]
  \label{def:chap10:exact_closed_forms}
  Let $\omega$ be a k-form in an open set $E \subset \R^n$.
  If there is a $(k-1)$-form $\lambda$ in E such that $\omega =
  d\lambda$, then $\omega$ is said to be \textbf{exact} in E.
  If $\omega$ is of class $\mathcal{C}'$ and $d\omega = 0$, then
  $\omega$ is said to be \textbf{closed}.
  Note: Since $d^2=0$ (Theorem
  \ref{thm:chap10:properties_exterior_derivative}(b)), every exact
  form of class $\mathcal{C}'$ is closed. The converse is not always
  true but holds in convex sets (Theorem \ref{thm:chap10:poincare_lemma}).
\end{definition}

\begin{remark}[Properties and Consequences]
  \label{rem:chap10:exact_closed_properties}
  Let $\omega$ be a k-form and $\lambda$ be a $(k-1)$-form in an open set E.
  \begin{itemize}
    \item[(a)] Closedness ($d\omega=0$) is a local condition
      verifiable by differentiation of coefficients (e.g., for a
      1-form $\omega = \sum f_i dx_i$, it means $D_j f_i = D_i f_j$).
      Exactness ($\omega=d\lambda$) is global, requiring finding
      $\lambda$ throughout E (e.g., finding $g$ such that $D_i g =
      f_i$ for all i).
    \item[(b)] If $\omega$ is exact ($\omega=d\lambda$), then
      $\int_\Psi \omega = \int_\Psi d\lambda = \int_{\partial \Psi}
      \lambda$ for any k-chain $\Psi$ (by Stokes' Thm
      \ref{thm:chap10:stokes_theorem}). The integral of an exact form
      depends only on the boundary. In particular, $\int_\Psi \omega
      = 0$ if $\partial \Psi = 0$ (e.g., for closed curves if k=1).
    \item[(c)] If $\omega$ is closed ($d\omega=0$), then
      $\int_{\partial \Psi} \omega = \int_\Psi d\omega = 0$ for any
      $(k+1)$-chain $\Psi$. Integrals of closed forms over boundaries are zero.
    \item[(d)] For any $(k+1)$-chain $\Psi$ of class $\mathcal{C}''$,
      $\int_{\partial(\partial \Psi)} \lambda = \int_{\partial \Psi}
      d\lambda = \int_\Psi d^2\lambda = 0$. This implies
      $\partial(\partial \Psi) = 0$, i.e., the boundary of a boundary is zero.
  \end{itemize}
\end{remark}

\begin{example}[Closed but not Exact 1-form]
  \label{ex:chap10:closed_not_exact_1form}
  Let $E = \R^2 \setminus \{ \vect{0} \}$. The 1-form $\eta = \frac{x
  dy - y dx}{x^2 + y^2}$ is closed in E ($d\eta = 0$). Let $\gamma(t)
  = (r \cos t, r \sin t)$ for $0 \le t \le 2\pi$ ($r>0$). Then
  $\partial \gamma = 0$ (since $\gamma(0)=\gamma(2\pi)$). Direct
  computation yields $\int_\gamma \eta = 2\pi$. Since the integral
  over the closed curve $\gamma$ is non-zero, $\eta$ cannot be exact
  in E (by Remark \ref{rem:chap10:exact_closed_properties}(b)). Also,
  $\gamma$ cannot be the boundary of any 2-chain in E (by Remark
  \ref{rem:chap10:exact_closed_properties}(c)).
\end{example}

\begin{example}[Closed but not Exact 2-form]
  \label{ex:chap10:closed_not_exact_2form}
  Let $E = \R^3 \setminus \{ \vect{0} \}$. The 2-form $\zeta =
  \frac{x dy \wedge dz + y dz \wedge dx + z dx \wedge dy}{(x^2 + y^2
  + z^2)^{3/2}}$ is closed in E ($d\zeta = 0$). Let $\Sigma$ be the
  unit sphere, parameterized as in Example
  \ref{ex:chap10:sphere_boundary}. We know $\partial \Sigma = 0$.
  Direct computation yields $\int_\Sigma \zeta = 4\pi$. Since the
  integral over the boundaryless surface $\Sigma$ is non-zero,
  $\zeta$ cannot be exact in E (by Remark
  \ref{rem:chap10:exact_closed_properties}(b)). Also, $\Sigma$ cannot
  be the boundary of any 3-chain in E (by Remark
  \ref{rem:chap10:exact_closed_properties}(c)).
\end{example}

\begin{theorem}[Helper Lemma for Poincaré's Lemma]
  \label{thm:chap10:poincare_helper_lemma}
  Suppose E is a convex open set in $\R^n$, $f \in \mathcal{C}'(E)$,
  p is an integer, $1 \le p \le n$, and $(D_j f)(\vect{x}) = 0$ for
  $p < j \le n$ and all $\vect{x} \in E$.
  Then there exists an $F \in \mathcal{C}'(E)$ such that $(D_p
  F)(\vect{x}) = f(\vect{x})$ and $(D_j F)(\vect{x}) = 0$ for $p < j
  \le n$ and all $\vect{x} \in E$.
\end{theorem}
% Proof omitted. (Proof involves integrating f w.r.t. x_p from a
% suitably chosen lower limit).

% --- End of transcription chunk ---

% --- Previous content (up to Thm 10.38) from chapters/chapter10.tex above ---

\begin{theorem}[Poincaré Lemma]
  \label{thm:chap10:poincare_lemma}
  If $E \subset \R^n$ is convex and open, if $k \ge 1$, if $\omega$
  is a k-form of class $\mathcal{C}'$ in E, and if $d\omega = 0$,
  then there is a $(k-1)$-form $\lambda$ in E such that $\omega = d\lambda$.
  (Briefly: closed forms are exact in convex sets).
\end{theorem}
% Proof omitted. (Proof uses induction on the number of variables
% involved in the form, applying Thm 10.38).

\begin{theorem}[Exactness under Diffeomorphism]
  \label{thm:chap10:exactness_diffeomorphism}
  Fix k, $1 \le k \le n$. Let $E \subset \R^n$ be an open set in
  which every closed k-form is exact (e.g., convex E). Let T be a 1-1
  $\mathcal{C}''$-mapping of E onto an open set $U \subset \R^n$
  whose inverse S is also of class $\mathcal{C}''$.
  Then every closed k-form in U is exact in U.
\end{theorem}
% Proof omitted. (Proof uses pullback: $\omega = (\omega_T)_S =
% (d\lambda)_S = d(\lambda_S)$).

\begin{remark}[$\mathcal{C}''$-Equivalence]
  Sets E and U related as in Theorem
  \ref{thm:chap10:exactness_diffeomorphism} are called
  $\mathcal{C}''$-equivalent. The theorem states that the property
  "closed implies exact" is preserved under $\mathcal{C}''$-equivalence.
\end{remark}

\begin{remark}[Cells vs. Simplexes]
  \label{rem:chap10:cells_vs_simplexes}
  Cells could be used as parameter domains instead of simplexes,
  simplifying some calculations. Simplexes are often preferred due to
  the natural definition of their boundary and connections to
  topology (triangulation). Since cells can be triangulated (viewed
  as chains of simplexes), the theories are closely related.
  Alternative proofs of Poincaré's lemma exist.
\end{remark}

\section{Vector Analysis}

% Intro text condensed.
We now translate some results about differential forms in $\R^3$ into
the language of vector analysis (vector fields, gradient, curl, divergence).

\subsection*{Vector Fields and Associated Forms/Operators}
\label{sec:chap10:vector_fields}

Let $E \subset \R^3$ be open. A continuous mapping
$\map{\vect{F}}{E}{\R^3}$, $\vect{F} = F_1 \vect{e}_1 + F_2
\vect{e}_2 + F_3 \vect{e}_3$, is called a \textbf{vector field}.
Associated with $\vect{F}$ are the 1-form $\lambda_F$ and the 2-form $\omega_F$:
\begin{align*}
  \lambda_F &= F_1 dx + F_2 dy + F_3 dz \\
  \omega_F &= F_1 dy \wedge dz + F_2 dz \wedge dx + F_3 dx \wedge dy.
\end{align*}
Conversely, every 1-form (or 2-form) in E corresponds to some vector
field $\vect{F}$. (Using $(x, y, z)$ for $(x_1, x_2, x_3)$).

If $u \in \mathcal{C}'(E)$ is a scalar function (0-form), its
\textbf{gradient} is the vector field
\[
  \nabla u = (D_1 u) \vect{e}_1 + (D_2 u) \vect{e}_2 + (D_3 u) \vect{e}_3.
\]
If $\vect{F} \in \mathcal{C}'(E)$ is a vector field, its
\textbf{curl} is the vector field
\[
  \nabla \times \vect{F} = (D_2 F_3 - D_3 F_2)\vect{e}_1 + (D_3 F_1 -
  D_1 F_3)\vect{e}_2 + (D_1 F_2 - D_2 F_1)\vect{e}_3,
\]
and its \textbf{divergence} is the scalar function
\[
  \nabla \cdot \vect{F} = D_1 F_1 + D_2 F_2 + D_3 F_3.
\]

% --- End of transcription chunk ---

% --- Previous content (up to Sec 10.42) from chapters/chapter10.tex above ---

\begin{theorem}[Vector Analysis Identities and Forms]
  \label{thm:chap10:vector_analysis_identities}
  Suppose E is an open set in $\R^3$, $u \in \mathcal{C}''(E)$
  (scalar field), and $\vect{G}$ is a vector field in E of class
  $\mathcal{C}''$.
  \begin{itemize}
    \item[(a)] If $\vect{F} = \nabla u$, then $\nabla \times \vect{F}
      = \vect{0}$.
    \item[(b)] If $\vect{F} = \nabla \times \vect{G}$, then $\nabla
      \cdot \vect{F} = 0$.
  \end{itemize}
  Furthermore, if E is $\mathcal{C}''$-equivalent to a convex set
  (see Remark following Thm
  \ref{thm:chap10:exactness_diffeomorphism}), then the converses hold
  for vector fields $\vect{F}$ of class $\mathcal{C}'$ in E:
  \begin{itemize}
    \item[(a')] If $\nabla \times \vect{F} = \vect{0}$, then
      $\vect{F} = \nabla u$ for some $u \in \mathcal{C}''(E)$.
    \item[(b')] If $\nabla \cdot \vect{F} = 0$, then $\vect{F} =
      \nabla \times \vect{G}$ for some vector field $\vect{G}$ in E
      of class $\mathcal{C}''$.
  \end{itemize}
\end{theorem}
% Proof omitted. (Proof relies on the following equivalences and
% properties of forms:
% $\vect{F}=\nabla u \Leftrightarrow \lambda_F = du$;
% $\nabla \times \vect{F} = \vect{0} \Leftrightarrow d\lambda_F = 0$;
% $\vect{F}=\nabla \times \vect{G} \Leftrightarrow \omega_F = d\lambda_G$;
% $\nabla \cdot \vect{F} = 0 \Leftrightarrow d\omega_F = 0$.
% Parts (a),(b) follow from $d^2=0$. Parts (a'),(b') follow from Thm
% \ref{thm:chap10:poincare_lemma} and Thm
% \ref{thm:chap10:exactness_diffeomorphism}.)

\subsection*{Volume Elements}
\label{sec:chap10:volume_elements}

The k-form $dx_1 \wedge \dots \wedge dx_k$ is called the
\textbf{volume element} in $\R^k$, often denoted $dV$ (or $dV_k$). If
$\Phi$ is a positively oriented k-surface in $\R^k$ (e.g., a 1-1
  $\mathcal{C}'$ map from a parameter domain $D \subset \R^k$ with
$J_\Phi > 0$), and f is continuous on $\Phi(D)$, the integral
notation is often used as:
\[
  \int_\Phi f(\vect{x}) dx_1 \wedge \dots \wedge dx_k = \int_\Phi f dV.
\]
By Theorems \ref{thm:chap10:integration_pullback1} and
\ref{thm:chap10:change_of_variables}, this equals $\int_{\Phi(D)}
f(\vect{x}) d\vect{x}$. In particular, $\int_\Phi dV$ gives the
volume of $\Phi(D)$. The standard notation for $dV_2$ is $dA$.

\subsection*{Green's Theorem}
\label{sec:chap10:greens_theorem}

Suppose E is an open set in $\R^2$, $\alpha, \beta \in
\mathcal{C}'(E)$, and $\Omega$ is a closed subset of E with
positively oriented boundary $\partial \Omega$ (a 1-chain, see Remark
\ref{rem:chap10:pos_oriented_boundary_set}). Then
\[
  \int_{\partial \Omega} (\alpha dx + \beta dy) = \int_\Omega \left(
    \frac{\partial \beta}{\partial x} - \frac{\partial \alpha}{\partial
  y} \right) dA.
\]
(This is Stokes' Theorem \ref{thm:chap10:stokes_theorem} for $k=2$,
  $m=2$, with $\lambda = \alpha dx + \beta dy$, so $d\lambda =
(D_1\beta - D_2\alpha) dx \wedge dy = (D_1\beta - D_2\alpha) dA$).
Special cases yield the area $A(\Omega)$: $A(\Omega) = \int_{\partial
\Omega} x dy = - \int_{\partial \Omega} y dx = \frac{1}{2}
\int_{\partial \Omega} (x dy - y dx)$.

\subsection*{Area Elements in $\R^3$}
\label{sec:chap10:area_elements_R3}

Let $\Phi$ be a 2-surface in $\R^3$ of class $\mathcal{C}'$, with
parameter domain $D \subset \R^2$. Let $(x, y, z) = \Phi(u, v)$.
Associate with $(u, v) \in D$ the \textbf{normal vector}
\[
  \vect{N}(u, v) = \frac{\partial(y, z)}{\partial(u, v)} \vect{e}_1 +
  \frac{\partial(z, x)}{\partial(u, v)} \vect{e}_2 +
  \frac{\partial(x, y)}{\partial(u, v)} \vect{e}_3.
\]
If f is a continuous function on $\Phi(D)$, the \textbf{area
integral} of f over $\Phi$ is
\[
  \int_\Phi f dA = \int_D f(\Phi(u, v)) |\vect{N}(u, v)| du dv.
\]
The \textbf{area} of $\Phi$ is $A(\Phi) = \int_\Phi 1 dA = \int_D
|\vect{N}(u, v)| du dv$.

Geometric Interpretation: If $\Phi'(u_0, v_0)$ has rank 2, the vector
$\vect{N}(u_0, v_0)$ is orthogonal to the tangent plane to $\Phi$ at
$\Phi(u_0, v_0)$. The magnitude $|\vect{N}(u, v)|$ represents the
local scaling factor for area when mapping from the parameter domain
D to the surface $\Phi(D)$ in $\R^3$.

% --- End of transcription chunk ---

% --- Previous content (up to Thm 10.43) from chapters/chapter10.tex above ---

\begin{example}[Volume and Surface Area of a Torus]
  \label{ex:chap10:torus_volume_area}
  Let $0 < a < b$. The mapping $\Psi: [0, a] \times [0, 2\pi] \times
  [0, 2\pi] \to \R^3$ given by
  \begin{align*}
    x &= t \cos u \\
    y &= (b + t \sin u) \cos v \\
    z &= (b + t \sin u) \sin v
  \end{align*}
  maps the 3-cell $K$ onto a solid torus $\Psi(K)$. Its Jacobian is
  $J_\Psi = t(b + t \sin u)$. The volume is
  \[
    \text{vol}(\Psi(K)) = \int_K J_\Psi dt du dv = 2\pi^2 a^2 b.
  \]
  The boundary $\partial \Psi$ (considering cancellations)
  corresponds to the surface $\Phi$ obtained by setting $t=a$,
  parameterized by $(u, v) \in [0, 2\pi] \times [0, 2\pi]$. The
  magnitude of the normal vector (Sec
  \ref{sec:chap10:area_elements_R3}) is $|\vect{N}(u, v)| = a(b + a
  \sin u)$. The surface area is
  \[
    A(\Phi) = \int_0^{2\pi} \int_0^{2\pi} |\vect{N}(u, v)| du dv = 4\pi^2 ab.
  \]
\end{example}

\subsection*{Line Integrals in $\R^3$}
\label{sec:chap10:line_integrals_R3}
Let $\gamma: [0, 1] \to E \subset \R^3$ be a $\mathcal{C}'$-curve.
Let $\vect{F}$ be a vector field in E. The \textbf{tangent vector} is
$\gamma'(u)$. Let $\vect{t}(u)$ be the unit tangent vector
$\gamma'(u)/|\gamma'(u)|$ (if $\gamma'(u) \ne \vect{0}$). The
\textbf{element of arc length} is $\ds = |\gamma'(u)| du$. The
integral of the 1-form $\lambda_F = F_1 dx + F_2 dy + F_3 dz$ over
$\gamma$ can be written as:
\[
  \int_\gamma \lambda_F = \int_0^1 \vect{F}(\gamma(u)) \cdot
  \gamma'(u) du = \int_0^1 (\vect{F}(\gamma(u)) \cdot \vect{t}(u))
  |\gamma'(u)| du = \int_\gamma (\vect{F} \cdot \vect{t}) \ds.
\]
$\vect{F} \cdot \vect{t}$ is the tangential component of $\vect{F}$
along $\gamma$.

\subsection*{Surface Integrals in $\R^3$}
\label{sec:chap10:surface_integrals_R3}
Let $\Phi: D \subset \R^2 \to E \subset \R^3$ be a $\mathcal{C}'$
2-surface. Let $\vect{F}$ be a vector field in E. Let $\vect{N}(u,
v)$ be the normal vector defined in Sec
\ref{sec:chap10:area_elements_R3}, and let $\vect{n}(u, v) =
\vect{N}(u, v) / |\vect{N}(u, v)|$ be the unit normal vector (if
$\vect{N} \ne \vect{0}$). The element of area is $\dA = |\vect{N}(u,
v)| du dv$. The integral of the 2-form $\omega_F = F_1 dy \wedge dz +
F_2 dz \wedge dx + F_3 dx \wedge dy$ over $\Phi$ can be written as:
\[
  \int_\Phi \omega_F = \int_D \vect{F}(\Phi(u, v)) \cdot \vect{N}(u,
  v) du dv = \int_D (\vect{F}(\Phi(u, v)) \cdot \vect{n}(u, v))
  |\vect{N}(u, v)| du dv = \int_\Phi (\vect{F} \cdot \vect{n}) \dA.
\]
$\vect{F} \cdot \vect{n}$ is the normal component of $\vect{F}$ over $\Phi$.

\begin{theorem}[Stokes' Formula]
  \label{thm:chap10:stokes_formula_vector}
  If $\vect{F}$ is a vector field of class $\mathcal{C}'$ in an open
  set $E \subset \R^3$, and if $\Phi$ is a 2-surface of class
  $\mathcal{C}''$ in E (viewed as a 2-chain), then
  \[
    \int_\Phi (\nabla \times \vect{F}) \cdot \vect{n} \dA =
    \int_{\partial \Phi} \vect{F} \cdot \vect{t} \ds.
  \]
  (This follows from Thm \ref{thm:chap10:stokes_theorem} using
    $\omega_H = d\lambda_F$ where $H = \nabla \times F$, $\int_\Phi (H
    \cdot n) dA = \int_\Phi \omega_H$, and $\int_{\partial \Phi} (F
  \cdot t) ds = \int_{\partial \Phi} \lambda_F$.)
\end{theorem}

\begin{theorem}[Divergence Theorem]
  \label{thm:chap10:divergence_theorem}
  If $\vect{F}$ is a vector field of class $\mathcal{C}'$ in an open
  set $E \subset \R^3$, and if $\Omega$ is a closed subset of E with
  positively oriented boundary $\partial \Omega$ (a 2-chain), then
  \[
    \int_\Omega (\nabla \cdot \vect{F}) \dV = \int_{\partial \Omega}
    (\vect{F} \cdot \vect{n}) \dA.
  \]
  (This follows from Thm \ref{thm:chap10:stokes_theorem} for $k=3$,
    using $d\omega_F = (\nabla \cdot F) dV$ and $\int_{\partial \Omega}
  \omega_F = \int_{\partial \Omega} (F \cdot n) dA$.)
\end{theorem}

% --- End of transcription chunk ---
