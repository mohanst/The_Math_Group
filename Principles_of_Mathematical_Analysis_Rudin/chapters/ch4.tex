% --- chapters/chapter4.tex ---
% Generated by Gemini (Google AI) on 2025-04-04.
% Contains extracted items from W. Rudin, PMA, Chapter 4.
% Assumes macros like \R, \abs{}, \map{}{}{}, \nhood{}{} are defined.
% Proofs are omitted as requested.

\chapter{Continuity}
\label{chap:rudin4}

\section{Limits of Functions}
\label{sec:chap4:limits_functions}

\begin{definition}[Limit of a Function] % Definition 4.1
  \label{def:chap4:limit_function}
  Let X and Y be metric spaces; suppose $E \subset X$, $f$ maps E
  into Y, and p is a limit point of E
  (\autoref{def:chap2:limit_point}). We write $f(x) \to q$ as $x \to p$, or
  \[ \lim_{x \to p} f(x) = q \]
  if there is a point $q \in Y$ with the following property: For
  every $\epsilon > 0$ there exists a $\delta > 0$ such that
  \[ d_Y(f(x), q) < \epsilon \]
  for all points $x \in E$ for which $0 < d_X(x, p) < \delta$.
  (The symbols $d_X$ and $d_Y$ refer to the distances in X and Y, respectively.)
  The limit q is unique if it exists (\autoref{cor:chap4:limit_uniqueness}).
  It should be noted that p need not be a point of E in this
  definition. Moreover, even if $p \in E$, we may have $f(p) \ne
  \lim_{x \to p} f(x)$.
\end{definition}

\begin{theorem}[Limit via Sequences] % Theorem 4.2
  \label{thm:chap4:limit_via_sequences}
  Let X, Y, E, f, and p be as in \autoref{def:chap4:limit_function}. Then
  \[ \lim_{x \to p} f(x) = q \]
  if and only if
  \[ \lim_{n \to \infty} f(p_n) = q \]
  for every sequence $\sequence{p}$ in E such that $p_n \ne p$,
  $\lim_{n \to \infty} p_n = p$.
  % Proof Omitted
\end{theorem}

\begin{corollary}[Uniqueness of Limit] % Corollary to 4.2
  \label{cor:chap4:limit_uniqueness}
  If $f$ has a limit at p, this limit is unique.
  % Proof Omitted (Follows from Thm 4.2 and uniqueness of sequence
  % limits Thm 3.2(b))
\end{corollary}

\begin{theorem}[Limits of Sums, Products, Quotients] % Theorem 4.3
  \label{thm:chap4:limit_algebra}
  Suppose $E \subset X$, a metric space, p is a limit point of E,
  $\map{f}{E}{\C}$, $\map{g}{E}{\C}$, and
  \[ \lim_{x \to p} f(x) = A, \quad \lim_{x \to p} g(x) = B. \]
  Then
  (a) $\lim_{x \to p} (f+g)(x) = A+B$;
  (b) $\lim_{x \to p} (fg)(x) = AB$;
  (c) $\lim_{x \to p} (f/g)(x) = A/B$, if $B \ne 0$.
  (These results also hold in $\R^k$ if Y is $\R^k$; cf.
  \autoref{thm:chap3:convergence_C_Rk}.)
  % Proof Omitted (Uses Thm 4.2 and Thm 3.3)
\end{theorem}

\section{Continuous Functions}
\label{sec:chap4:continuous_functions}

\begin{definition}[Continuity] % Definition 4.4
  \label{def:chap4:continuity}
  Suppose X and Y are metric spaces, $E \subset X$, $p \in E$, and
  $\map{f}{E}{Y}$. Then f is said to be continuous at p if for every
  $\epsilon > 0$ there exists a $\delta > 0$ such that
  \[ d_Y(f(x), f(p)) < \epsilon \]
  for all points $x \in E$ for which $d_X(x, p) < \delta$.
  If f is continuous at every point of E, then f is said to be continuous on E.
  Note that f has to be defined at p in order to be continuous at p.
  If p is an isolated point of E
  (\autoref{def:chap2:isolated_point}), then f is automatically
  continuous at p. If p is a limit point of E, then f is continuous
  at p if and only if $\lim_{x \to p} f(x) = f(p)$.
\end{definition}

\begin{theorem}[Continuity via Sequences] % Theorem 4.5
  \label{thm:chap4:continuity_via_sequences}
  Let X, Y be metric spaces, $E \subset X$, $p \in E$, and
  $\map{f}{E}{Y}$. Then f is continuous at p if and only if $\lim_{n
  \to \infty} f(p_n) = f(p)$ for every sequence $\sequence{p}$ in E
  such that $p_n \to p$.
  % Proof Omitted (Similar logic to Thm 4.2)
\end{theorem}

\begin{theorem}[Continuity of Sums, Products, Quotients,
  Compositions] % Theorem 4.6
  \label{thm:chap4:continuity_algebra_composition}
  Suppose X, Y, Z are metric spaces, $E \subset X$, $\map{f}{E}{Y}$,
  $\map{g}{f(E)}{Z}$, and h is the mapping of E into Z defined by
  $h(x) = g(f(x))$ for $x \in E$. If f is continuous at a point $p
  \in E$ and if g is continuous at the point $f(p)$, then h is
  continuous at p. (This function h is the composition $g \circ f$.)
  Furthermore, if Y is $\C$ (or $\R^k$) and f, g map E into Y, and f,
  g are continuous at p, then $f+g$ and $fg$ are continuous at p. If
  Y is $\C$ (or $\R$) and $g(p) \ne 0$, then $f/g$ is continuous at p.
  % Proof Omitted (Uses Thm 4.5 and Thm 3.3 / limit definition)
\end{theorem}

% --- Content from Thm 4.7 to Thm 4.13 (Proofs Omitted) to append ---

\begin{theorem}[Continuity of Vector Functions/Coordinates] % Theorem 4.7
  \label{thm:chap4:cont_vector_coords}
  Let $\map{f}{E}{\R^k}$ (where $E \subset X$, a metric space). Let
  \[ \vect{f}(x) = (f_1(x), \dots, f_k(x)) \]
  for $x \in E$. Then $\vect{f}$ is continuous at $p \in E$ if and
  only if each of the functions $f_i$ ($i=1,\dots,k$) is continuous at p.
  The functions $f_i$ are called the components of $\vect{f}$.
  If $\vect{f}$ and $\vect{g}$ map E into $\R^k$, are continuous at
  p, then $\vect{f}+\vect{g}$ and $\vect{f} \cdot \vect{g}$ (inner
  product) are continuous at p.
  % Proof Omitted
\end{theorem}

\section{Continuity and Compactness}
\label{sec:chap4:cont_compact}

\begin{theorem}[Continuity via Open/Closed Sets] % Theorem 4.8
  \label{thm:chap4:cont_open_closed_sets}
  Let $\map{f}{X}{Y}$ be a mapping of a metric space X into a metric
  space Y. Then f is continuous on X if and only if $f^{-1}(V)$ is
  open in X for every open set V in Y.
  (Equivalently, f is continuous on X if and only if $f^{-1}(C)$ is
  closed in X for every closed set C in Y.)
  % Proof Omitted
\end{theorem}

\begin{theorem}[Continuous Image of Compact Set is Compact] % Theorem 4.9
  \label{thm:chap4:image_compact_is_compact}
  If $\map{f}{X}{Y}$ is a continuous mapping of a compact metric
  space X (\autoref{def:chap2:compact_set}) into a metric space Y,
  then the image $f(X)$ is compact.
  % Proof Omitted (Uses Thm 4.8 and definition of compactness)
\end{theorem}

\begin{corollary}[Extreme Value Theorem] % Corollary to 4.9
  \label{cor:chap4:extreme_value_thm}
  If $\map{f}{X}{\R^1}$ is a continuous mapping of a compact metric
  space X into $\R$, then $f(X)$ is closed and bounded
  (\autoref{thm:chap2:compact_implies_closed},
  \autoref{thm:chap2:heine_borel}). Thus, there exist points $p, q
  \in X$ such that $f(p) = \sup f(X)$ and $f(q) = \inf f(X)$.
  (That is, f attains its maximum and minimum values on X.)
  % Proof Omitted (Uses Thm 4.9, properties of sup/inf on
  % closed+bounded sets of R)
\end{corollary}

% Note: Rudin discusses that f: compact X -> Y makes f a closed map
% (maps closed sets to closed sets),
% using Thm 4.9 and Thm 2.35, but doesn't number it as a theorem.

\begin{theorem}[Continuous Inverse Theorem] % Theorem 4.11 (Labeled
  % 4.17 in Rudin 2nd Ed)
  \label{thm:chap4:continuous_inverse}
  Suppose $\map{f}{X}{Y}$ is a continuous 1-1 mapping
  (\autoref{def:chap2:image_inverse_onto_1-1}) of a compact metric
  space X onto a metric space Y. Then the inverse mapping $f^{-1}$
  defined on Y by
  \[ f^{-1}(f(x)) = x \quad (x \in X) \]
  is a continuous mapping of Y onto X.
  % Proof Omitted (Uses Thm 4.8 on f^{-1} and the fact that f maps
  % closed sets to closed sets)
\end{theorem}

\begin{definition}[Uniform Continuity] % Definition 4.12 (Labeled
  % 4.18 in Rudin 2nd Ed)
  \label{def:chap4:uniform_continuity}
  Let $\map{f}{X}{Y}$ be a mapping of a metric space X into a metric
  space Y. We say that f is uniformly continuous on X if for every
  $\epsilon > 0$ there exists $\delta > 0$ such that
  \[ d_Y(f(p), f(q)) < \epsilon \]
  for all p and q in X for which $d_X(p, q) < \delta$.
  (The key point is that $\delta$ depends only on $\epsilon$, not on p or q.)
\end{definition}

\begin{theorem}[Continuity on Compact Set implies Uniform Continuity]
  % Theorem 4.13 (Labeled 4.19 in Rudin 2nd Ed)
  \label{thm:chap4:compact_implies_uniform_cont}
  Let $\map{f}{X}{Y}$ be a continuous mapping of a compact metric
  space X into a metric space Y. Then f is uniformly continuous on X.
  % Proof Omitted
\end{theorem}

% --- End of content chunk ---

% --- Content from Thm 4.14 to Def 4.20 (Proofs Omitted) to append ---

\section{Continuity and Connectedness}
\label{sec:chap4:cont_connected}

\begin{theorem}[Continuous Image of Connected Set is Connected] %
  % Theorem 4.14 (Labeled 4.22 in Rudin 2nd Ed)
  \label{thm:chap4:image_connected_is_connected}
  If $\map{f}{X}{Y}$ is a continuous mapping of a connected metric
  space X (\autoref{def:chap2:connected_set}) into a metric space Y,
  then $f(X)$ is connected.
  % Proof Omitted
\end{theorem}

\begin{theorem}[Intermediate Value Theorem] % Theorem 4.15 (Labeled
  % 4.23 in Rudin 2nd Ed)
  \label{thm:chap4:intermediate_value_thm}
  Let $\map{f}{[a, b]}{\R}$ be continuous. If $f(a) < c < f(b)$, then
  there is a point $x \in (a, b)$ such that $f(x) = c$. (A similar
  result holds if $f(a) > f(b)$.)
  % Proof Omitted (Uses Thm 4.14 and Thm 2.47)
\end{theorem}

\section{Discontinuities}
\label{sec:chap4:discontinuities}

\begin{definition}[Types of Discontinuities] % Definition 4.16
  % (Labeled 4.25 in Rudin 2nd Ed)
  \label{def:chap4:discontinuity_types}
  Let $\map{f}{(a, b)}{\R}$. If f is discontinuous at a point x, and
  if the left-hand limit $f(x-)$ and the right-hand limit $f(x+)$
  exist, then f is said to have a discontinuity of the first kind, or
  a simple discontinuity, at x. Otherwise the discontinuity is said
  to be of the second kind.
  There are two possibilities for the first kind:
  (a) $f(x-) = f(x+)$. In this case, the discontinuity is removable;
  defining $f(x) = f(x-) = f(x+)$ makes f continuous at x.
  (b) $f(x-) \ne f(x+)$. This is often called a jump discontinuity.
\end{definition}

\begin{theorem}[Discontinuities of Monotonic Functions] % Theorem
  % 4.17 (Labeled 4.29 in Rudin 2nd Ed)
  \label{thm:chap4:monotonic_discont_limits}
  Let $\map{f}{(a, b)}{\R}$ be monotonically increasing. Then $f(x+)$
  and $f(x-)$ exist at every point $x \in (a, b)$. More precisely,
  \[ \sup_{a<t<x} f(t) = f(x-) \le f(x) \le f(x+) = \inf_{x<t<b} f(t). \]
  Furthermore, if $a < x < y < b$, then $f(x+) \le f(y-)$.
  Analogous results hold for monotonically decreasing functions.
  % Proof Omitted
\end{theorem}

\section{Monotonic Functions}
\label{sec:chap4:monotonic_functions}

\begin{definition}[Monotonic Function] % Definition 4.18 (Labeled
  % 4.28 in Rudin 2nd Ed)
  \label{def:chap4:monotonic_function}
  Let $\map{f}{(a, b)}{\R}$. We say that f is monotonically
  increasing on $(a, b)$ if $a < x < y < b$ implies $f(x) \le f(y)$.
  We say f is monotonically decreasing if $a < x < y < b$ implies
  $f(x) \ge f(y)$. A function is monotonic if it is either
  monotonically increasing or monotonically decreasing.
\end{definition}

\begin{theorem}[Monotonic Functions Discontinuities] % Theorem 4.19
  % (Labeled 4.30 in Rudin 2nd Ed)
  \label{thm:chap4:monotonic_discont_type}
  Let $\map{f}{(a, b)}{\R}$ be monotonic. Then the set of points of
  $(a, b)$ at which f is discontinuous is at most countable.
  (Specifically, monotonic functions have no discontinuities of the
  second kind, by \autoref{thm:chap4:monotonic_discont_limits}).
  % Proof Omitted
\end{theorem}

\section{Infinite Limits and Limits at Infinity}
\label{sec:chap4:infinite_limits}

\begin{definition}[Infinite Limits and Limits at Infinity] %
  % Definition 4.20 (Labeled 4.32 in Rudin 2nd Ed)
  \label{def:chap4:infinite_limit_defs}
  Let $\map{f}{E}{\R}$, where $E \subset \R$. We write $f(x) \to
  +\infty$ as $x \to p$ if for every real number M there exists
  $\delta > 0$ such that $f(x) > M$ for all $x \in E$ with $0 <
  \abs{x-p} < \delta$. Similarly for $f(x) \to -\infty$.
  For functions defined on $(a, +\infty)$, we write $\lim_{x \to
  +\infty} f(x) = q$ if $q \in \R$ and for every $\epsilon > 0$,
  there exists $M \in \R$ such that $\abs{f(x) - q} < \epsilon$ for
  all $x > M$. Similarly for $x \to -\infty$ (on $(-\infty, a)$) and
  for $f(x) \to \pm\infty$ as $x \to \pm\infty$.
\end{definition}

% --- End of Chapter 4 content chunk (Proofs Omitted) ---
% --- End of chapters/chapter4.tex ---

% --- End of content chunk ---
